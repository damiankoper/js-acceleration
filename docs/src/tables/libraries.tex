\begin{table}
    \caption{Popularne biblioteki do przetwarzania danych w~języku Python i~ich odpowiedniki w~języku JavaScript. Dane pochodzą z~serwisów kolejno PyPI Stats oraz NPM, a~w~nawiasach znajduje się tygodniowa liczba pobrań biblioteki.}
    \centering
    \renewcommand\arraystretch{1.2}
    \begin{tabularx}{\linewidth}[t]{p{5cm} p{4.5cm} X}
        \bfseries{Python} & \bfseries{JavaScript} & \bfseries{Zastosowanie} \\ \hline
        numpy (26.067.844) & numjs (533) & Operacje na macierzach \\ \hline
        pandas (19.778.648) & danfojs (1.029) & Operacje na strukturach danych \\ \hline
        scipy (9.986.376) & simple-statistics (87.882) \newline fft.js (8.027) & Operacje związane z~analizą numeryczną, przetwarzanie sygnałów, algebra liniowa. \\ \hline
        scikit-learn (7.705.438) & ml (156) & Uczenie maszynowe \\ \hline
        matplotlib (6.457.099) \newline plotly (1.632.246) & plotly.js (149.542) \newline c3 (83.564) & Wizualizacja danych \\ \hline
        tensorflow (3.348.986) & @tensorflow/tfjs (91.233) & Sieci neuronowe \\ \hline
        opencv-python (1.236.711) & OpenCV.js (b.d \cite{cv2-js}) \newline jimp (1.479.783) \newline image-js (4.103)        & Operacja na obrazach, computer vision \\ \hline

    \end{tabularx}
    \label{tab:py-js}
\end{table}
