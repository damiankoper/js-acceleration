\begin{figure}[ht]
    \groupBenchmark{
        \plotBenchmark{js-wasm_theta_SHT_Simple_node.csv}{nodeColor}{{}}
        \addlegendentry{Node}

        \plotBenchmark{js-wasm_theta_SHT_Simple_Firefox.csv}{firefoxColor}{{}}
        \addlegendentry{Firefox}

        \plotBenchmark{js-wasm_theta_SHT_Simple_Chrome.csv}{chromeColor}{{}}
        \addlegendentry{Chrome}

        \seqReference
    } {
        \plotBenchmark{js-wasm_theta_SHT_Simple_Lookup_node.csv}{nodeColor}{{}}
        \addlegendentry{Node}

        \plotBenchmark{js-wasm_theta_SHT_Simple_Lookup_Firefox.csv}{firefoxColor}{{}}
        \addlegendentry{Firefox}

        \plotBenchmark{js-wasm_theta_SHT_Simple_Lookup_Chrome.csv}{chromeColor}{{}}
        \addlegendentry{Chrome}

        \seqReferenceLookup
    } {
        \plotBenchmark{js-wasm_theta_CHT_Simple_node.csv}{nodeColor}{{}}
        \addlegendentry{Node}

        \plotBenchmark{js-wasm_theta_CHT_Simple_Firefox.csv}{firefoxColor}{{}}
        \addlegendentry{Firefox}

        \plotBenchmark{js-wasm_theta_CHT_Simple_Chrome.csv}{chromeColor}{{}}
        \addlegendentry{Chrome}

        \seqReferenceCircle
    }[2500][950][15]
    \caption{Wyniki pomiarów czasu wydajności dla wykonania SHT i CHT z wykorzystaniem kompilacji do WASM.}
    \label{plot:wasm}
\end{figure}
