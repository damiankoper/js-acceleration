\begin{abstract}
   W~tej pracy zaprezentowane zostały popularne metody akceleracji obliczeń w~środowiskach języka JavaScript. Zbadano wydajność w~środowiskach przeglądarek internetowych Google Chrome oraz Mozilla Firefox, a~także w~środowiskach serwerowych - NodeJS i~Deno. Metodę wykrywania kształtów z~użyciem algorytmów Standard Hough Transform i~Circle Hough Transform zaimplementowano z~użyciem metody poprawy wykonania sekwencyjnego, wykorzystania natywnych rozszerzeń NodeJS, kompilacji i~wykonania kodu asm.js i~WebAssembly, również w~wariancie SIMD. Zaimplementowano również metody współbieżne z~wykorzystaniem Worker'ów oraz GPU używając WebGL API. Najlepszym ze środowisk okazała się przeglądarka Google Chrome, a~najwolniejszym Mozilla Firefox. Najbardziej wydajną metodą sekwencyjną są natywne rozszerzenia NodeJS, a~współbieżną WebGL. Opracowane wyniki stanowić mogą podstawę do wyboru metody akceleracji w~implementowanych algorytmach, aby móc konkurować ze środowiskami innych języków i~tworzyć wydajne algorytmy intensywne obliczeniowo.
\end{abstract}
\mykeywords{javascript, akceleracja obliczeń, sht, standard hough transform
node, przeglądarka internetowa, deno, webgl, webpack, wasm, simd, workers}

{
    \selectlanguage{english}
    \begin{abstract}
        This paper presents popular acceleration methods in JavaScript execution environments. Performance of browser environments of Google Chrome and Mozilla Firefox was examined as well as the server ones - NodeJS and Deno. Algorithms used for benchmarking were Standard Hough Transform and Circle Hough Transform used for pattern detection in images. Acceleration method implemented are sequential execution improvement, NodeJS native addons, WebAssembly with asm.js and SIMD variants. Parallel implemented method are usage of Workers and GPU with WebGL API. It appears that the most performant envitonment is Google Chrome and the least one is Mozilla Firefox. The fastest sequential method is the usage of native addons in NodeJS and the parallel one is WebGL. Summarized results may help choosing the most suitable acceleration method for algoritm to be implemented, to be able to compete with other languages' environments and to create efficient compute-intensive algorytms.    \end{abstract}
    \mykeywords{javascript, acceleration, sht, standard hough transform
    node, web browser, deno, webgl, webpack, wasm, simd, workers}
}
