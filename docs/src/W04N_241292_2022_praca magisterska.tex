%%%%%%%%%%%%%%%%%%%%%%%%%%%%%%%%%%%%%%%%%%%%%%%%%%%%%%%%%%%%%%%%%%%%%%%%%
%  Content: Main file of thesis template (master/enginering).
%  Author: Tomasz Kubik <tomasz.kubik@pwr.edu.pl>
%  Data: 1 marca 2020
%  Wersja: 0.4
%%%%%%%%%%%%%%%%%%%%%%%%%%%%%%%%%%%%%%%%%%%%%%%%%%%%%%%%%%%%%%%%%%%%%%%%%

\documentclass[a4paper,onecolumn,12pt,twoside,extrafontsizes]{memoir}
% In order to prepare the manuscript for archives (2x1, double printed) you can:
% a) produce normal pdf, then convert it to pdf with two pages on one phisical page (solution suggested)
%
%   This can be done by:
%   - printing from Adobe Acrobat Reader with option "Multiple"
%   - using psutils
%
%      Windows (assuming that you have MiKTeX installed with the package pakiet miktex-psutils-bin-x64-2.9):
%        "c:\Program Files\MiKTeX 2.9\miktex\bin\x64\pdf2ps.exe" Dyplom.pdf Dyplom.ps 
%        "c:\Program Files\MiKTeX 2.9\miktex\bin\x64\psnup.exe" -2 Dyplom.ps Dyplom2.ps
%        "c:\Program Files\MiKTeX 2.9\miktex\bin\x64\ps2pdf.exe" Dyplom2.ps Dyplom2.pdf
%        Del Dyplom2.ps Dyplom.ps
%
%     Linux:
%        pdf2ps Dyplom.pdf - | psnup -2 | ps2pdf - Dyplom2.pdf
%
%
% b) produce 'reduced' pdf by settning smaller fonts in the document class definition (it changes the formating thus it is not suggested)
%
%   Use of the following commands instead of original one:
%   \documentclass[a4paper,onecolumn,twoside,10pt]{memoir} 
%   \renewcommand{\normalsize}{\fontsize{8pt}{10pt}\selectfont}

%\usepackage[cp1250]{inputenc} % for cp1250 file encoding



\usepackage{amssymb}
\usepackage{emptypage}

\usepackage[utf8]{inputenc} % for UTF8 file encoding
\usepackage[T1]{fontenc}
\usepackage[polish]{babel}
%\usepackage[english]{babel}
%\DisemulatePackage{setspace}
\usepackage{setspace}
\usepackage{tabularx}
\usepackage{color,calc}
\usepackage{threeparttable,tablefootnote}

%\usepackage{soul} % packege with commands for text highliting

%% Accorging to the rules the main font of the thesis should be Times.
%% In order to achieve it we use tgtermes font that offers shapes: normal, bold, italic, italic bold.
%% There is no slanted shape.
%% If you use slanted font in the text (typing \textsl{} commnand), then LaTeX will substitute it with a~standard font giving you a~warning.
%% Additionally tgtermes works for the running text. All maths (formulas, equations) will be rendered with a~default font for math.
%% If you with to change the font for math, you must to do it yourself.

%% After installation of tgtermes package there might be a~need to update font and mapping information 
%% This can be don by running the following commands (as administrator)
%% initexmf --admin --update-fndb
%% initexmf --admin --mkmaps

\usepackage{tgtermes}   

\renewcommand*\ttdefault{txtt}

% The settings regarding fonts were used in the earlier version of the template. It is kept for historical reasons
%\usepackage{mathptmx} 
%\usepackage{newtxtext,newtxmath} 
%\usepackage{newtxmath,tgtermes} 

\usepackage{listings} % used to render the source code 
% In UTF8 encoding there was a~need to define the following mapping (otherwise the national characters were not recognized correctly)
%%\lstset{literate=%-
%%{ą}{{\k{a}}}1 {ć}{{\'c}}1 {ę}{{\k{e}}}1 {ł}{{\l{}}}1 {ń}{{\'n}}1 {ó}{{\'o}}1 {ś}{{\'s}}1 {ż}{{\.z}}1 {ź}{{\'z}}1 {Ą}{{\k{A}}}1 {Ć}{{\'C}}1 {Ę}{{\k{E}}}1 {Ł}{{\L{}}}1 {Ń}{{\'N}}1 {Ó}{{\'O}}1 {Ś}{{\'S}}1 {Ż}{{\.Z}}1 {Ź}{{\'Z}}1 
    %%{Ö}{{\"O}}1
    %%{Ä}{{\"A}}1
    %%{Ü}{{\"U}}1
    %%{ß}{{\ss}}1
    %%{ü}{{\"u}}1
    %%{ä}{{\"a}}1
    %%{ö}{{\"o}}1
    %%{~}{{\textasciitilde}}1
		%%{—}{{{\textemdash} }}1
%%}%{\ \ }{{\ }}1}

\usepackage{pgfplots}

\usepackage{amsmath}
\usepackage{multicol}
\usepackage{placeins}
\usepackage{makecell}

\usepackage{siunitx}
\usepackage{physics}
\usepackage[f]{esvect}
\newcommand{\normalize}[1]{
  \ensuremath{\frac{#1}{\Vert#1\Vert}}
}
\newcommand{\length}[1]{
  \ensuremath{\Vert#1\Vert}
}
\usepackage{tikz}
\usetikzlibrary{quotes,babel,calc,patterns,angles, arrows.meta, intersections}
 \usepackage{pdfpages}
 \usepackage{wrapfig}
 \usepackage{pifont}
 \usepackage{xargs}

 \usepackage{float}
 \usepackage{longtable}
 \usepackage{xltabular}
 \usepackage{eqexpl}
 \usepackage{tkz-euclide}
 \usepackage{scalefnt}
 \usetikzlibrary{pgfplots.groupplots} % needs to be loaded exactly like this
 \usepgfplotslibrary{fillbetween}
 \usepgfplotslibrary{colorbrewer}
 \usetikzlibrary{patterns}

\usepackage{csvsimple}
\newcommand{\tms}[1]{\ensuremath{#1\times}}


 \newcommand{\cmark}{\ding{51}}%
\newcommand{\xmark}{\ding{55}}%

 \eqexplSetIntro{gdzie:}
% \usepackage{mathdots}
% \usepackage{yhmath}
% \usepackage{cancel}
% \usepackage{color}
% \usepackage{siunitx}
% \usepackage{array}
% \usepackage{multirow}
% \usepackage{amssymb}
% \usepackage{gensymb}
% \usepackage{tabularx}
% \usepackage{booktabs}
% \usetikzlibrary{patterns}
% \usetikzlibrary{shadows.blur}
% \usetikzlibrary{shapes}



\newcommand{\listingcaption}[1]% added to handle captions of listings in two columns 
{%
\vspace*{\abovecaptionskip}\small 
\refstepcounter{lstlisting}\hfill%
Listing \thelstlisting: #1\hfill%\hfill%
\addcontentsline{lol}{lstlisting}{\protect\numberline{\thelstlisting}#1}
}%

\lstdefinelanguage{JavaScript}{
  keywords={const, let, typeof, new, true, false, catch, function, return, null, catch, switch, var, if, in, while, do, else, case, break, type},
  keywordstyle=\color{blue}\bfseries,
  ndkeywords={class, export, boolean, default, extends throw, implements, import, this, void, number, string, public, Promise, private, protected, extends, from, constructor},
  ndkeywordstyle=\color{darkgray}\bfseries,
  identifierstyle=\color{black},
  sensitive=false,
  comment=[l]{//},
  morecomment=[s]{/*}{*/},
  commentstyle=\color{purple}\ttfamily,
  stringstyle=\color{red}\ttfamily,
  morestring=[b]',
  morestring=[b]"
}

\lstdefinelanguage{WASM}{
  keywords={typeof, new, true, false, catch, function, return, null, catch, switch, var, if, in, while, do, else, case, break, func, param, result},
  keywordstyle=\color{blue}\bfseries,
  ndkeywords={class, export, boolean, default, extends throw, implements, import, this, void, number, string, public, Promise, private, protected, extends, from, constructor},
  ndkeywordstyle=\color{darkgray}\bfseries,
  identifierstyle=\color{black},
  sensitive=false,
  comment=[l]{//},
  morecomment=[s]{/*}{*/},
  commentstyle=\color{purple}\ttfamily,
  stringstyle=\color{red}\ttfamily,
  morestring=[b]',
  morestring=[b]"
}

% code style with line numbering (but these are commented out)
\lstset{
  basicstyle=\footnotesize\ttfamily,
  %%columns=fullflexible,
	%%showstringspaces=false,
	%%showspaces=false,
  breaklines=true,
  postbreak=\mbox{\textcolor{red}{$\hookrightarrow$}\space},
  numbers=left,
  %%firstnumber=1,
  %%numberfirstline=true,
	%%xleftmargin=17pt,
  %%framexleftmargin=17pt,
  %%framexrightmargin=5pt,
  %%framexbottommargin=4pt,
	belowskip=.5\baselineskip
}

% code style without line numbering
%%\lstset{
  %%basicstyle=\footnotesize\ttfamily,
  %%columns=fullflexible,
	%%showstringspaces=false,
	%%showspaces=false,
  %%breaklines=true,
  %%postbreak=\mbox{\textcolor{red}{$\hookrightarrow$}\space},
%%}

%% Here you have another example of code styling
%%\lstloadlanguages{% Check Dokumentation for further languages ...
%%C,
%%C++,
%%csh,
%%Java
%%}
%%
%%\definecolor{red}{rgb}{0.6,0,0} % for strings
%%\definecolor{blue}{rgb}{0,0,0.6}
%%\definecolor{green}{rgb}{0,0.8,0}
%%\definecolor{cyan}{rgb}{0.0,0.6,0.6}
%%
%%\lstdefinestyle{sqlstyle}{
%%language=SQL,
%%basicstyle=\footnotesize\ttfamily, 
%%numbers=left, 
%%numberstyle=\tiny, 
%%numbersep=5pt, 
%%tabsize=2, 
%%extendedchars=true, 
%%breaklines=true, 
%%showspaces=false, 
%%showtabs=true, 
%%xleftmargin=17pt,
%%framexleftmargin=17pt,
%%framexrightmargin=5pt,
%%framexbottommargin=4pt,
%%keywordstyle=\color{blue}, 
%%commentstyle=\color{green}, 
%%stringstyle=\color{red}, 
%%}
%%
%%\lstdefinestyle{sharpcstyle}{
%%language=[Sharp]C,
%%basicstyle=\footnotesize\ttfamily, 
%%numbers=left, 
%%numberstyle=\tiny, 
%%numbersep=5pt, 
%%tabsize=2, 
%%extendedchars=true, 
%%breaklines=true, 
%%showspaces=false, 
%%showtabs=true, 
%%xleftmargin=17pt,
%%framexleftmargin=17pt,
%%framexrightmargin=5pt,
%%framexbottommargin=4pt,
%%morecomment=[l]{//}, %use comment-line-style!
%%morecomment=[s]{/*}{*/}, %for multiline comments
%%showstringspaces=false, 
%%morekeywords={  abstract, event, new, struct,
                %%as, explicit, null, switch,
                %%base, extern, object, this,
                %%bool, false, operator, throw,
                %%break, finally, out, true,
                %%byte, fixed, override, try,
                %%case, float, params, typeof,
                %%catch, for, private, uint,
                %%char, foreach, protected, ulong,
                %%checked, goto, public, unchecked,
                %%class, if, readonly, unsafe,
                %%const, implicit, ref, ushort,
                %%continue, in, return, using,
                %%decimal, int, sbyte, virtual,
                %%default, interface, sealed, volatile,
                %%delegate, internal, short, void,
                %%do, is, sizeof, while,
                %%double, lock, stackalloc,
                %%else, long, static,
                %%enum, namespace, string},
%%keywordstyle=\color{cyan},
%%identifierstyle=\color{red},
%%stringstyle=\color{blue}, 
%%commentstyle=\color{green},
%%}


\renewcommand\lstlistlistingname{Spis listingów}
\makeatletter

%\renewcommand*{\l@lstlisting}[2]{\@dottedtocline{1}{0em}{2.3em}{#1}{#2}}
\g@addto@macro\insertchapterspace{\addtocontents{lol}{\protect\addvspace{10pt}}}
\renewcommand*{\l@lstlisting}{\@dottedtocline{1}{0em}{2.3em}}
\makeatother
\usepackage{subcaption}

\renewcommand*{\lstlistlistingname}{Spis listingów} \newlistof{lstlistoflistings}{lol}{\lstlistlistingname}



% It is possible to use packages mentioned below for making various tables but it is suggested to left them unused
%\usepackage{longtable}
%\usepackage{ltxtable}
%\usepackage{tabulary}

%%%%%%%%%%%%%%%%%%%%%%%%%%%%%%%%%%%%%%%%%%%%%%%%%%%
%% Settings related to the autmatic document typesetting
%% and floats placements
%%%%%%%%%%%%%%%%%%%%%%%%%%%%%%%%%%%%%%%%%%%%%%%%%%%
%\hyphenpenalty=10000		% do not break words too often
\clubpenalty=10000      % penalty for orphans
\widowpenalty=10000  % do not left widows
%\brokenpenalty=10000		% dont break words between pages - commented out because interfere with line breaking in lstlisting
%\exhyphenpenalty=999999		% dont break word with dash - commented out because interfere with line breaking in lstlisting
\righthyphenmin=3			% break at min 3 characters

%\tolerance=4500
%\pretolerance=250
%\hfuzz=1.5pt
%\hbadness=1450

\renewcommand{\topfraction}{0.95}
\renewcommand{\bottomfraction}{0.95}
\renewcommand{\textfraction}{0.05}
\renewcommand{\floatpagefraction}{0.35}

%%%%%%%%%%%%%%%%%%%%%%%%%%%%%%%%%%%%%%%%%%%%%%%%%%%
%%  Size settings: text, header and footer, marigins
%%  for documents based on memoir class
%%%%%%%%%%%%%%%%%%%%%%%%%%%%%%%%%%%%%%%%%%%%%%%%%%%
\setlength{\headsep}{10pt} 
\setlength{\headheight}{13.6pt} % baselineskip for 11pt font, i.e. \small, equals 13.6pt
\setlength{\footskip}{\headsep+\headheight}
\setlength{\uppermargin}{\headheight+\headsep+1cm}
\setlength{\textheight}{\paperheight-\uppermargin-\footskip-1.5cm}
\setlength{\textwidth}{\paperwidth-5cm}
\setlength{\spinemargin}{2.5cm}
\setlength{\foremargin}{2.5cm}
\setlength{\marginparsep}{2mm}
\setlength{\marginparwidth}{2.3mm}
%\settrimmedsize{297mm}{210mm}{*}
%\settrims{0mm}{0mm}	
\checkandfixthelayout[fixed] % needed to fix the layout
%%%%%%%%%%%%%%%%%%%%%%%%%%%%%%%%%%%%%%%%%%%%%%%%
%%  Settings related to the interline spaces, indentations, distances
%%%%%%%%%%%%%%%%%%%%%%%%%%%%%%%%%%%%%%%%%%%%%%%%
\linespread{1}
%\linespread{1.241}
\setlength{\parindent}{14.5pt}
%\setlength{\cftbeforechapterskip}{0.3em} % spaces in the table of contents
%\setbeforesecskip{10pt plus 0.5ex}%{-3.5ex \@plus -1ex \@minus -.2ex}
%\setaftersecskip{10pt plus 0.5ex}%\onelineskip}
%\setbeforesubsecskip{8pt plus 0.5ex}%{-3.5ex \@plus -1ex \@minus -.2ex}
%\setaftersubsecskip{8pt plus 0.5ex}%\onelineskip}
%\setlength\floatsep{6pt plus 2pt minus 2pt} 
%\setlength\intextsep{12pt plus 2pt minus 2pt} 
%\setlength\textfloatsep{12pt plus 2pt minus 2pt} 

%%%%%%%%%%%%%%%%%%%%%%%%%%%%%%%%%%%%%%%%%%%%%%%%%%%
%%  Pakiety i~komendy zastosowane tylko do zamieszczenia informacji o~użytych komendach i~fontach
%%  Normalnie nie są potrzebne, można je zamarkować podczas redakcji pracy
%%%%%%%%%%%%%%%%%%%%%%%%%%%%%%%%%%%%%%%%%%%%%%%%%%%
\usepackage{./packages/memlays}     % extra layout diagrams, used only in template for 'debugging'. Is uses layouts package. Comment out them both if you edit your thesis
%\usepackage{layouts}
\usepackage{printlen} % allows displeying the values of defined lengths, used onlu in template for 'debugging'. Comment out it if you edit your thesis
\uselengthunit{pt}
\makeatletter
\newcommand{\showFontSize}{\f@size pt} % makro wypisujące wielkość bieżącej czcionki
\makeatother
% if you wish to show the frames:
%\usepackage{showframe} 


%%%%%%%%%%%%%%%%%%%%%%%%%%%%%%%%%%%%%%%%%%%%%%%%%%%
%%  Enumerated lists definitions
%%%%%%%%%%%%%%%%%%%%%%%%%%%%%%%%%%%%%%%%%%%%%%%%%%%

% Item lists have, by default, the bullets which are charactes not existing in the set of tgtermes fonts
% Therefore these are substituted by LaTeX with characters from standard set of fonts. In order to change this
% behavior you cad declare substitutions as below
%    \DeclareTextCommandDefault{\textbullet}{\ensuremath{\bullet}}
%    \DeclareTextCommandDefault{\textasteriskcentered}{\ensuremath{\ast}}
%    \DeclareTextCommandDefault{\textperiodcentered}{\ensuremath{\cdot}}
% But the better way is to redefine enumitem environment from enumitem package
\usepackage{enumitem}
\setlist{noitemsep,topsep=4pt,parsep=0pt,partopsep=4pt,leftmargin=*} % this makes list more compact
\setenumerate{labelindent=0pt,itemindent=0pt,leftmargin=!,label=\arabic*.} % it is possible to use \arabic or \alph, if the enumerations are supposed to be rendered with different numbers
\setlistdepth{4} % limits the depth of nested enumeration
\setlist[itemize,1]{label=$\bullet$}  % here we define the bullet at each of the levels
\setlist[itemize,2]{label=\normalfont\bfseries\textendash}
\setlist[itemize,3]{label=$\ast$}
\setlist[itemize,4]{label=$\cdot$}
\renewlist{itemize}{itemize}{4}

%%%http://tex.stackexchange.com/questions/29322/how-to-make-enumerate-items-align-at-left-margin
%\renewenvironment{enumerate}
%{
%\begin{list}{\arabic{enumi}.}
%{
%\usecounter{enumi}
%%\setlength{\itemindent}{0pt}
%%\setlength{\leftmargin}{1.8em}%{2zw} % 
%%\setlength{\rightmargin}{0zw} %
%%\setlength{\labelsep}{1zw} %
%%\setlength{\labelwidth}{3zw} % 
%\setlength{\topsep}{6pt}%
%\setlength{\partopsep}{0pt}%
%\setlength{\parskip}{0pt}%
%\setlength{\parsep}{0em} % 
%\setlength{\itemsep}{0em} % 
%%\setlength{\listparindent}{1zw} % 
%}
%}{
%\end{list}
%}

\makeatletter
\renewenvironment{quote}{
	\begin{list}{}
	{
	\setlength{\leftmargin}{1em}
	\setlength{\topsep}{0pt}%
	\setlength{\partopsep}{0pt}%
	\setlength{\parskip}{0pt}%
	\setlength{\parsep}{0pt}%
	\setlength{\itemsep}{0pt}
	}
	}{
	\end{list}}
\makeatother

%%%%%%%%%%%%%%%%%%%%%%%%%%%%%%%%%%%%%%%%%
%%  Package for index generation (must be set before hyperref)
%%%%%%%%%%%%%%%%%%%%%%%%%%%%%%%%%%%%%%%%%
%%\DisemulatePackage{imakeidx} % uncomment it out if you wish to generate an index
%%\usepackage[makeindex,noautomatic]{imakeidx} % here we say that the index can not be generated automatically

\makeatletter
%%%\renewenvironment{theindex}
							 %%%{\vskip 10pt\@makeschapterhead{\indexname}\vskip -3pt%
								%%%\@mkboth{\MakeUppercase\indexname}%
												%%%{\MakeUppercase\indexname}%
								%%%\vspace{-3.2mm}\parindent\z@%
								%%%\renewcommand\subitem{\par\hangindent 16\p@ \hspace*{0\p@}}%%
								%%%\phantomsection%
								%%%\begin{multicols}{2}
								%%%%\thispagestyle{plain}
								%%%\parindent\z@                
								%%%%\parskip\z@ \@plus .3\p@\relax
								%%%\let\item\@idxitem}
							 %%%{\end{multicols}\clearpage}
%%%
\makeatother


\usepackage{ifpdf}
%\newif\ifpdf \ifx\pdfoutput\undefined
%\pdffalse % we are not running PDFLaTeX
%\else
%\pdfoutput=1 % we are running PDFLaTeX
%\pdftrue \fi
\ifpdf
 \usepackage[pdftex,bookmarks,breaklinks,unicode]{hyperref}
% \usepackage[pdftex]{graphicx}
 \DeclareGraphicsExtensions{.pdf,.jpg,.mps,.png}
\pdfcompresslevel=9
\pdfoutput=1
\makeatletter
\AtBeginDocument{  % Here are the metadata that will be embedded in the resulting pdf. Please fill in them correctly
  \hypersetup{
	pdfinfo={
    Title = {\@title},
    Author = {\@author},
    Keywords={javascript, akceleracja obliczeń, sht, standard hough transform
    node, przeglądarka internetowa, deno, webgl, webpack, wasm, simd, workers},  
	}}
}
\pdftrailerid{} %Remove ID
\pdfsuppressptexinfo15 %Suppress PTEX.Fullbanner and info of imported PDFs

\makeatother
\else
\usepackage{graphicx}
\DeclareGraphicsExtensions{.eps,.ps,.jpg,.mps,.png}
\fi
\sloppy

%\graphicspath{{figures01/}{figures02/}} %% if you with to have set the paths to figures  


% Depth of numbering
\setcounter{secnumdepth}{2}
\setcounter{tocdepth}{2}
\setsecnumdepth{subsection} % activating subsubsec numbering in doc


% dots aftes sections numbers
\makeatletter
\def\@seccntformat#1{\csname the#1\endcsname.\quad}
\def\numberline#1{\hb@xt@\@tempdima{#1\if&#1&\else.\fi\hfil}}
\makeatother

\renewcommand{\chapternumberline}[1]{#1.\quad}
\renewcommand{\cftchapterdotsep}{\cftdotsep}

%\definecolor{niceblue}{rgb}{.168,.234,.671}

% Fonts in figures and tables captions
\captionnamefont{\small}
\captiontitlefont{\small}
% macro adjusting the way the chapter title is rendered
%\def\printchaptertitle##1{\fonttitle \space \thechapter.\space ##1} 

%\usepackage{ltcaption}
% The ltcaption package supports \CaptionLabelFont & \CaptionTextFont introduced by the NTG document classes
%\renewcommand\CaptionLabelFont{\small}
%\renewcommand\CaptionTextFont{\small}

% Redefinitions of labels for tables, figures and bibliography 
%\AtBeginDocument{% 
        \addto\captionspolish{% 
        \renewcommand{\tablename}{Tab.}% 
}%} 

%\AtBeginDocument{% 
%        \addto\captionspolish{% 
%        \renewcommand{\chaptername}{Rozdział}% 
%}} 

%\AtBeginDocument{% 
        \addto\captionspolish{% 
        \renewcommand{\figurename}{Rys.}% 
}%}


%\AtBeginDocument{% 
        \addto\captionspolish{% 
        \renewcommand{\bibname}{Literatura}% 
}%}

%\AtBeginDocument{% 
        \addto\captionspolish{% 
        \renewcommand{\listfigurename}{Spis rysunków}% 
}%}

%\AtBeginDocument{% 
        \addto\captionspolish{% 
        \renewcommand{\listtablename}{Spis tabel}% 
}%}

%\AtBeginDocument{% 
        \addto\captionspolish

%%%%%%%%%%%%%%%%%%%%%%%%%%%%%%%%%%%%%%%%%%%%%%%%%%%%%%%%%%%%%%%%%%                  
%% Definition of headers and footers appearing on pages
%%%%%%%%%%%%%%%%%%%%%%%%%%%%%%%%%%%%%%%%%%%%%%%%%%%%%%%%%%%%%%%%%%                  
\addtopsmarks{headings}{%
\nouppercaseheads % added at the beginning
}{%
\createmark{chapter}{both}{shownumber}{}{. \space}
%\createmark{chapter}{left}{shownumber}{}{. \space}
\createmark{section}{right}{shownumber}{}{. \space}
}%use the new settings

\makeatletter
\copypagestyle{outer}{headings}
\makeoddhead{outer}{}{}{\small\itshape\rightmark}
\makeevenhead{outer}{\small\itshape\leftmark}{}{}
\makeoddfoot{outer}{\small\@author:~\@titleShort}{}{\small\thepage}
\makeevenfoot{outer}{\small\thepage}{}{\small\@author:~\@titleShort}
\makeheadrule{outer}{\linewidth}{\normalrulethickness}
\makefootrule{outer}{\linewidth}{\normalrulethickness}{2pt}
\makeatother

% fix plain
\copypagestyle{plain}{headings} % overwrite plain with outer
\makeoddhead{plain}{}{}{} % remove right header
\makeevenhead{plain}{}{}{} % remove left header
\makeevenfoot{plain}{}{}{}
\makeoddfoot{plain}{}{}{}

\copypagestyle{empty}{headings} % overwrite plain with outer
\makeoddhead{empty}{}{}{} % remove right header
\makeevenhead{empty}{}{}{} % remove left header
\makeevenfoot{empty}{}{}{}
\makeoddfoot{empty}{}{}{}


%%%%%%%%%%%%%%%%%%%%%%%%%%%%%%%%%%%%%%%
%% Definition of title page
%%%%%%%%%%%%%%%%%%%%%%%%%%%%%%%%%%%%%%%
\makeatletter
% University
\newcommand\uczelnia[1]{\renewcommand\@uczelnia{#1}}
\newcommand\@uczelnia{}
% Faculty
\newcommand\wydzial[1]{\renewcommand\@wydzial{#1}}
\newcommand\@wydzial{}
% Field
\newcommand\kierunek[1]{\renewcommand\@kierunek{#1}}
\newcommand\@kierunek{}
% Speciality
\newcommand\specjalnosc[1]{\renewcommand\@specjalnosc{#1}}
\newcommand\@specjalnosc{}
% Title in english
\newcommand\titleEN[1]{\renewcommand\@titleEN{#1}}
\newcommand\@titleEN{}
% Short title (used in headers/footers
\newcommand\titleShort[1]{\renewcommand\@titleShort{#1}}
\newcommand\@titleShort{}
% Supervisor
\newcommand\promotor[1]{\renewcommand\@promotor{#1}}
\newcommand\@promotor{}

%\usepackage[absolute]{textpos} % not used because picture environment was applied


\makeatother
%%%%%%%%%%%%%%%%%%%%%%%%%%%%%%%%%%%%%%%%%

%\AtBeginDocument{\addtocontents{toc}{\protect\thispagestyle{empty}}}




%%%%%%%%%%%%%%%%%%%%%%%%%%%%%%%%%%%%%%%%%
%%  Metadane dokumentu 
%%%%%%%%%%%%%%%%%%%%%%%%%%%%%%%%%%%%%%%%%
\title{Autorska implementacja algorytmów transformacji Hough'a dla wybranych metod akceleracji obliczeń w~środowiskach języka JavaScript}
\titleShort{Implementacja algorytmów transformacji Hough'a}
\titleEN{Propietary implementation of Hough transform algorithms for selected calculation acceleration methods in JavaScript language environments}
\author{Damian Koper}
\uczelnia{POLITECHNIKA WROCŁAWSKA}
\wydzial{WYDZIAŁ INFORMATYKI I TELEKOM.}
\kierunek{INFORMATYKA}
\specjalnosc{GRAFIKA I SYSTEMY MULTIMEDIALNE}
\promotor{Dr inż. Marek Woda}
\date{WROCŁAW, 2022}

% Setting the space above nonnumbered chapters and list: ToC, LoT, LoF, Index
% Spis treści, Spis tabel, Spis rysunków, Indeks rzeczowy

%\newlength{\linespace}
%\setlength{\linespace}{-\beforechapskip-\topskip+\headheight+\topsep}
%\makechapterstyle{noNumbered}{%
%\renewcommand\chapterheadstart{\vspace*{\linespace}}
%}

%% powyższa komenda załatwia to, co robią komendy poniższe dla spisów
%\renewcommand*{\tocheadstart}{\vspace*{\linespace}}
%\renewcommand*{\lotheadstart}{\vspace*{\linespace}}
%\renewcommand*{\lofheadstart}{\vspace*{\linespace}}

%%%%%%%%%%%%%%%%%%%%%%%%%%%%%%%%%%%%%%%%%
%                  Beginning of the document 
%%%%%%%%%%%%%%%%%%%%%%%%%%%%%%%%%%%%%%%%%
%\includeonly{abbreviations,chapter01} % uncomment it if you with to complile only selected latex files (this will speed up compilation, especially if you focuse on a~specific part of the thesis)

\renewcommand{\abstractnamefont}{\normalfont\Large\bfseries}
\renewcommand{\abstracttextfont}{\normalfont}
%\AtBeginDocument{% 
\addto\captionspolish{
\renewcommand\abstractname{Streszczenie} % niepotrzebne, bo przy polskich ustawieniach jêzykowych jest 'Streszczenie'
}%}

%\AtBeginDocument{% 
\addto\captionsenglish{
\renewcommand\englishabstractname{Abstract} % niepotrzebne, bo przy polskich ustawieniach jêzykowych jest 'Streszczenie'
}%}

\makeatletter
\edef\kv{ }
\newcommand{\kvAdd}[1]{\edef\kv{\kv{}#1 }}
\newcommand\mykeywords[1]{\kvAdd{#1}\hspace{\absleftindent}%
\parbox{\linewidth-2.0\absleftindent}{
       \iflanguage{polish}{\textbf{Słowa kluczowe:} #1}{%
			  \iflanguage{english}{\textbf{Keywords:} #1}}{}}
				}

        \pgfplotsset{
    table/search path={./../../benchmark},
    every pin edge/.style={solid},
    compat=newest
}
\definecolor{cppColor}{rgb}{0,0,0}
\definecolor{nodeColor}{rgb}{0,0.6,0}
\definecolor{denoColor}{rgb}{0,0,1}
\definecolor{chromeColor}{rgb}{1,0,0}
\definecolor{firefoxColor}{rgb}{1,0.5,0}

\newcommand{\plotBenchmark}[3]{%
    \addplot+ [
        #2,
        #3,
        mark=*,
        mark options={solid,fill=#2, scale=0.5},
        error bars/.cd,
        error bar style={mark size=3pt, solid, #2},
        y dir=both,
        y explicit,
    ]
    table [
            x=sizeTheta,
            y=mean,
            y error=stdev,
            col sep=comma
        ] {#1};
}

\newenvironment{chartBenchmark}
{\begin{tikzpicture}
        \begin{axis} [
                width=\linewidth,
                height=0.45\linewidth,
                legend style={font=\tiny},
                grid,
                grid style=dashed,
                xlabel={$S_\theta$ [pixels per degree]},
                ylabel={Time [ms]},
                mark options={solid},
                legend pos=north west
            ] }
            %
            {
        \end{axis}
    \end{tikzpicture}
}

\newcommandx\groupBenchmark[4][3=1000, 4=800]{
    \begin{tikzpicture}
        \begin{groupplot}[
                group style={group size=2 by 1},
                width=0.52552\linewidth,
                grid,
                grid style=dashed,
                legend style={
                        legend columns=5,
                        font=\tiny,
                    },
                tick label style={font=\tiny},
                every axis title shift=0pt,
                max space between ticks=12,
                ymin=0
            ]

            \nextgroupplot[
                title={SHT Simple},
                ylabel={Time [ms]},
                legend to name={CommonLegend},
                xlabel={$S_\theta$ [pixels per degree]},
                ymax=#3
            ]
            #1

            \nextgroupplot[
                title={SHT Simple Lookup},
                xlabel={$S_\theta$ [pixels per degree]},
                legend to name={CommonLegend},
                ymax=#4
            ]
            #2
        \end{groupplot}
        \path (group c1r1.south east) -- node[below, yshift=-7ex] {\ref{CommonLegend}} (group c2r1.south west);
    \end{tikzpicture}
}


\newcommand\seqReference{
    \plotBenchmark{cpp_theta_SHT_Simple.csv}{cppColor}{dashed}
    \addlegendentry{C++ Sequential};
    \plotBenchmark
    {js-sequential_theta_SHT_Simple_Firefox.csv}
    {firefoxColor}
    {name path=Firefox_Seq,opacity=0}
    \plotBenchmark
    {js-sequential_theta_SHT_Simple_Chrome.csv}
    {chromeColor}
    {name path=Chrome_Seq,opacity=0}
    \addplot [black,opacity=0.1] fill between [of=Chrome_Seq and Firefox_Seq];
}

\newcommand\seqReferenceLookup{
    \plotBenchmark{cpp_theta_SHT_Simple_Lookup.csv}{cppColor}{dashed}
    \addlegendentry{C++ Sequential};
    \plotBenchmark
    {js-sequential_theta_SHT_Simple_Lookup_Firefox.csv}
    {firefoxColor}
    {name path=Firefox_Seq_Lookup,opacity=0}
    \plotBenchmark
    {js-sequential_theta_SHT_Simple_Lookup_Chrome.csv}
    {chromeColor}
    {name path=Chrome_Seq_Lookup,opacity=0}
    \addplot [black,opacity=0.1] fill between [of=Chrome_Seq_Lookup and Firefox_Seq_Lookup];
}

\begin{document}

% Here you have the commands for setting the spacying (do not change)
%\SingleSpacing
%\OnehalfSpacing
%\DoubleSpacing

%\settypeoutlayoutunit{cm} % for debbuging
%\typeoutstandardlayout    % prints on stdout the info about settings
\pdfbookmark[0]{Tytuł}{Tytul.1}
\includepdf[pages={1}]{../out/title}

\cleardoublepage
\thispagestyle{empty}
\begin{abstract}
   W~tej pracy zaprezentowane zostały popularne metody akceleracji obliczeń w~środowiskach języka JavaScript. Zbadano wydajność w~środowiskach przeglądarek internetowych Google Chrome oraz Mozilla Firefox, a~także w~środowiskach serwerowych - NodeJS i~Deno. Metodę wykrywania kształtów z~użyciem algorytmów Standard Hough Transform i~Circle Hough Transform zaimplementowano z~użyciem metody poprawy wykonania sekwencyjnego, wykorzystania natywnych rozszerzeń NodeJS, kompilacji i~wykonania kodu asm.js i~WebAssembly, również w~wariancie SIMD. Zaimplementowano również metody współbieżne z~wykorzystaniem Worker'ów oraz GPU używając WebGL API. Najlepszym ze środowisk okazała się przeglądarka Google Chrome, a~najwolniejszym Mozilla Firefox. Najbardziej wydajną metodą sekwencyjną są natywne rozszerzenia NodeJS, a~współbieżną WebGL. Opracowane wyniki stanowić mogą podstawę do wyboru metody akceleracji w~implementowanych algorytmach, aby móc konkurować ze środowiskami innych języków i~tworzyć wydajne algorytmy intensywne obliczeniowo.
\end{abstract}
\mykeywords{javascript, akceleracja obliczeń, sht, standard hough transform
node, przeglądarka internetowa, deno, webgl, webpack, wasm, simd, workers}

{
    \selectlanguage{english}
    \begin{abstract}
        This paper presents popular acceleration methods in JavaScript execution environments. Performance of browser environments of Google Chrome and Mozilla Firefox was examined as well as the server ones - NodeJS and Deno. Algorithms used for benchmarking were Standard Hough Transform and Circle Hough Transform used for pattern detection in images. Acceleration method implemented are sequential execution improvement, NodeJS native addons, WebAssembly with asm.js and SIMD variants. Parallel implemented method are usage of Workers and GPU with WebGL API. It appears that the most performant envitonment is Google Chrome and the least one is Mozilla Firefox. The fastest sequential method is the usage of native addons in NodeJS and the parallel one is WebGL. Summarized results may help choosing the most suitable acceleration method for algoritm to be implemented, to be able to compete with other languages' environments and to create efficient compute-intensive algorytms.    \end{abstract}
    \mykeywords{javascript, acceleration, sht, standard hough transform
    node, web browser, deno, webgl, webpack, wasm, simd, workers}
}
 
\pagestyle{outer}
\clearpage

\chapterstyle{noNumbered}  % Below there are declarations of various lists. Please comment out those which are too short (the list should contain at least 5 items)
\pagestyle{outer}
\mbox{}\pdfbookmark[0]{Spis treści}{spisTresci.1}
\tableofcontents*

\newpage
\mbox{}\pdfbookmark[0]{Spis rysunków}{spisRysunkow.1}
%\addcontentsline{toc}{chapter}{Spis rysunków}
\listoffigures*

\newpage
\mbox{}\pdfbookmark[0]{Spis tabel}{spisTabel.1}
%\addcontentsline{toc}{chapter}{Spis tabel}
\listoftables*

\newpage
\mbox{}\pdfbookmark[0]{Spis listingów}{spisListingow.1}
%\addcontentsline{toc}{chapter}{Spis listingów}
\lstlistoflistings*

% Below there are inclusions of latex files
\chapter*{Skróty}\mbox{}\pdfbookmark[0]{Skróty}{skroty.1}
\label{sec:skroty}
\noindent
\begin{description}[labelwidth=*]
  \item [EXAMPLE] (ang.\ \emph{Example})
\end{description}
 % If abbreviations list is short, you can comment it out
\chapterstyle{default}{
\section{Introduction}\label{sec:introduction}

TODO: JS in general, JS in numerical computing.

TODO: Environments briefly.

TODO: Acceleration methods briefly.

TODO: Building methods and complex architectures.

TODO: Target:
* overall comparison,
* to identify the most promising acceleration method and environment 
* to identify fields and aspects to conduct further research

\chapter{Język JavaScript}

JavaScript od momentu swojego powstania w~1995 roku stanowi jeden z~filarów rozwoju technologii webowych, zaczynając od dodania prostych mechanizmów interaktywności do statycznych stron internetowym, a~kończąc na byciu nieraz jedynym samodzielnym budulcem pełnowymiarowych aplikacji działających po stronie klienta i~serwera, aplikacji działających w~środowisku przeglądarki internetowej, ale też w~środowiskach natywnych, desktopowych i~mobilnych. Dlatego, aby zrozumieć w~pełni specyfikę problemu, który stanowi przystosowanie języka do wykonywania obliczeń numerycznych, w~tym rozdziale przybliżone zostały zagadnienia związane z~modelem wykonania, środowiskami oraz sposobami na podział kodu na moduły i~późniejsze ich wykorzystanie. Na końcu opisane zostały metody akceleracji, dla których przeprowadzono badania.



\section{Model wykonania}

Model wykonania języka JavaScript skoncentrowany jest w~głównej mierze na obsłudze zdarzeń. W~przeglądarce internetowej zdarzeniami takimi mogą być interakcje z~użytkownikiem, na przykład kiedy naciśnięty zostanie przycisk, albo interakcje z~siecią, kiedy otrzymamy odpowiedź na zapytanie z~wykorzystaniem obiektu \lstinline{XMLHttpRequest} lub skorzystamy z~Fetch API. Po stronie serwera zdarzeniami takimi mogą być odebranie zapytania, które serwer musi obsłużyć, obsługa strumieni, ale także wszelkie odpowiedzi na interakcje z~systemem operacyjnym. Podstawowymi interakcjami mogą być obsługa sygnałów, dostęp do plików, czy też obsługa sieci, która umożliwia połączenie na przykład z~bazą danych.

Zdarzenia te obsługuje pętla zdarzeń (ang. event loop). Na rysunku \ref{fig:event-loop} pokazano jej uproszczony model. Wyróżnia ona zadania, zwane także makro zadaniami, oraz mikro zadania. Dla każdego typu zadań utworzona zostaje osobna kolejka. Jeśli aktualnie wykonywane przetwarzanie sekwencyjne, którego ramki wywołań śledzone są na stosie, zakończy się, wtedy z~pętli zdarzeń pobierane i~wykonywane jest makro lub mikro zadanie. W~pierwszej kolejności wykonywane są wszystkie mikro zadania, a~gdy ich kolejka jest pusta, wykonywane jest kolejne makro zadanie.

Makro zadania dodawane są do kolejki, aby obsłużyć wspomniane już zdarzenia związane z~działaniami użytkownika lub inne zewnętrzne zdarzenia. Są one również dodawane do kolejki, kiedy mija czas zadany podczas wywołań funkcji \lstinline{setTimeout()} oraz \lstinline{setInterval()}. Warto zaznaczyć, że wywołania tych funkcji nie gwarantują wykonania dokładnie po zadanym czasie, ale traktują go jako próg czasowy, po jakim zadana funkcja zostanie dodana do kolejki makro zadań \cite{setTimeout}. Makro zadania dodane podczas jednej iteracji pętli nigdy nie zostaną wykonane w~tej samej iteracji.

Mikro zadania pochodzą tylko i~wyłącznie z~kodu użytkownika, bądź bibliotek i~wykorzystywane są do obsługi asynchronicznych zadań, zarządzania ich kolejnością i~obsługą błędów\cite{runtime}. Do ich tworzenia wykorzystuje się głównie obiekt \lstinline{Promise}}, który reprezentuje zadanie, które w~przyszłości zakończy się pomyślnie lub błędem. Dla takiego obiektu zdefiniować możemy funkcje, które wykonają się podczas scenariusza pomyślnego (\lstinline{then}), błędnego (\lstinline{catch}) oraz zawsze (\lstinline{finally}) \cite{promise}. Mikro zadania pochodzić mogą również od obserwatorów na przykład \lstinline{MutationObserver}, czy \lstinline{ResizeObserver}.

Zrozumienie sposobu wykonywania kodu w~asynchronicznym modelu języka JavaScript jest kluczowe do efektywnego wykorzystania możliwości, jakie idą za metodami akceleracji, których użycie możliwe jest tylko i~wyłącznie poprzez asynchroniczne wywołania. Opisane w~sekcji \ref{sec:acc-methods} metody bazujące na Worker'ach oraz WebGL wymagają interakcji poprzez wywołania asynchroniczne. Na listingu \ref{lst:async} pokazano przykład kodu asynchronicznego. Wywołania \mbox{\lstinline{console.log}} wykonają się, zawsze drukując liczby w~kolejności od 1 do 6.

\lstinputlisting[language=JavaScript, caption=Przykład kodu demonstrujący mechanizmy asynchroniczności w~języku JavaScript., label=lst:async] {./code/async.js}

Linijki 1 oraz 6 zostają wykonane synchronicznie. \lstinline{Promise} w~linijce 5 zostaje wykonany jako następny, ponieważ jako mikro zadanie, wykona się zaraz po operacjach synchronicznych. Następnie funkcje \lstinline{setTimeout} wykonają się w~kolejności ich wywołania, gdzie w~linijkach 2-4 na ich kolejność wpływ maja mechanizm \lstinline{Promise}. Timeout z~linijki 3, a~potem 4 zostają wywołane jaki pierwsze. Jako ostatni wykonuje się timeout z~linijki 2, ponieważ jego wywołanie, w~postaci mikro taska, przeniesione zostało na koniec wykonania synchronicznego.

\begin{figure}
    \centering
    

\begin{tikzpicture}

\node[label=below:Event Loop,minimum size=1.9cm] (loop) at (0,-8) {\scalefont{4}$\circlearrowleft$};

% \node[rectangle, draw, minimum width=3cm, minimum height=8cm, anchor=north] (Heap_frame) at (0,0) {};
% \node[minimum width=3cm, below=0.25 of Heap_frame.north] (Heap)  {Heap} ;
% \node[rectangle, draw, minimum width=1cm, minimum height=1cm, below right=0.54 and 0.33 of Heap.west] (A) {\lstinline{a}} ;
% \node[rectangle, draw, minimum width=1cm, minimum height=1cm, right=0.33 of A] (B)  {\lstinline{b}} ;
% \node[rectangle, draw, minimum width=1cm, minimum height=1cm, below=0.33 of B] (C)  {\lstinline{c}} ;
% \node[rectangle, draw, minimum width=1cm, minimum height=1cm, left=0.33 of C] (D)  {\lstinline{d}} ;
% \node[rectangle, draw, minimum width=1cm, minimum height=1cm, below=0.33 of D] (E)  {\lstinline{e}} ;
% \node[rectangle, draw, minimum width=1cm, minimum height=1cm, right=0.33 of E] (F)  {\lstinline{f}} ;
% \node[rectangle, draw, minimum width=1cm, minimum height=1cm, below=0.33 of E] (G)  {\lstinline{f}} ;

\node[rectangle, draw, minimum width=3.5cm, minimum height=6cm, anchor=north] (Stack_frame) at (0,0) {};
\node[minimum width=3cm, minimum height=1cm, below=0 of Stack_frame.north] (Stack)  {Stack} ;
\node[rectangle, draw, minimum width=3cm, minimum height=0.6cm, below=0 of Stack] (exec) {\lstinline{exec()}} ;
\node[rectangle, draw, minimum width=3cm, minimum height=0.6cm, below=0 of exec] (A) {\lstinline{function A()}} ;
\node[rectangle, draw, minimum width=3cm, minimum height=0.6cm, below=0 of A] (B)  {\lstinline{function B()}} ;
\node[rectangle, draw, minimum width=3cm, minimum height=0.6cm, below=0 of B] (C)  {\lstinline{function C()}} ;
\node[rectangle, draw, minimum width=3cm, minimum height=0.6cm, below=0 of C] (D)  {\lstinline{function D()}} ;
\node[rectangle, draw, minimum width=3cm, minimum height=0.6cm, below=0 of D] (E)  {\lstinline{function E()}} ;

% \path[->,thick] (1.5,-7) edge (2,-7);
% \path[<-,thick] (1.5,-7.5) edge (2,-7.5);


\node[rectangle, draw, minimum width=4.5cm, minimum height=5.2cm, anchor=north] (APIs_frame) at (11.5,0) {};
\node[minimum width=3cm, minimum height=1cm,, below=0 of APIs_frame.north] (APIs)  {APIs} ;
\node[rectangle, draw, minimum width=4cm, minimum height=0.75cm, below=0 of APIs] (main) {Fetch API} ;
\node[rectangle, draw, minimum width=4cm, minimum height=0.75cm, below=0 of main] (A) {Image Capture API} ;
\node[rectangle, draw, minimum width=4cm, minimum height=0.75cm, below=0 of A] (B)  {WebSocket API} ;
\node[rectangle, draw, minimum width=4cm, minimum height=0.75cm, below=0 of B] (C)  {Clipboard API} ;
\node[rectangle, draw, minimum width=4cm, minimum height=0.75cm, below=0 of C] (D)  {...} ;

\node[rectangle, draw, minimum width=11cm, minimum height=1cm, anchor=north east] (macro_frame) at (13.75,-6) {};
\node[ anchor=south east, above left=0 of macro_frame, anchor=south west] {Macro-task queue};

\node[rectangle, draw, minimum height=0.6cm, right=0.2 of macro_frame.west] (macroA)  
    {\lstinline{setTimeout()}} ;
\node[rectangle, draw, minimum height=0.6cm, right=0.2 of macroA] (macroB)  
    {\lstinline{setInterval()}} ;
\node[rectangle, draw, minimum height=0.6cm, right=0.2 of macroB] (macroC)  
    {\lstinline{fetch()}} ;
\node[rectangle, draw, minimum height=0.6cm, right=0.2 of macroC] (macroD)  
    {\lstinline{...}} ;

\node[rectangle, draw, minimum width=11cm, minimum height=1cm,  anchor=north east, below right=1.5cm and 0cm of macro_frame.west] (micro_frame)  {};
\node[ anchor=south east, above left=0 of micro_frame, anchor=south west] {Micro-task queue};

\node[rectangle, draw, minimum height=0.6cm, right=0.2 of micro_frame.west] (microA)  
    {\lstinline{Promise.then()}} ;
\node[rectangle, draw, minimum height=0.6cm, right=0.2 of microA] (microB)  
    {\lstinline{queueMicrotask()}} ;
\node[rectangle, draw, minimum height=0.6cm, right=0.2 of microB] (microC)  
    {\lstinline{...}} ;

\path [ultra thick,latex-] (loop) edge (macro_frame.west);
\path [ultra thick,latex-] (loop) edge (micro_frame.west);
\path [ultra thick,-latex] (loop.north) edge (Stack_frame);

\draw [ultra thick,-latex] (APIs_frame.south) -- (APIs_frame|-macro_frame.north);
\draw [ultra thick,-latex] (Stack_frame) -- (APIs_frame.west|-Stack_frame);


\end{tikzpicture}

    \caption{Uproszczony model pętli zdarzeń środowisk języka JavaScript w~wariancie z~wyróżnieniem API przeglądarek internetowych.}
    \label{fig:event-loop}
\end{figure}

\section{Środowiska JavaScript}
\label{sec:env-modules}

Rosnąca popularność języka JavaScript i~idących za jego stosowaniem możliwości przyspieszyła rozwój środowisk, w~których kod języka mógł być wykonywany. Pierwszym z~nich była przeglądarka internetowa. 

\subsection{Przeglądarka internetowa}

Głównym zadaniem przeglądarki internetowej jest pobieranie z~sieci i~wyświetlanie zawartości użytkownikowi oraz obsługa jego interakcji. Najważniejszym komponentem przeglądarki jest silnik renderujący, który zawiera między innymi silniki odpowiedzialne za parsowanie i~renderowanie struktury modelu DOM, adaptery służące do wywołań dostępnych bibliotek graficznych (OpenGL, Vulcan, DirectX) oraz silnik JavaScript. Z~perspektywy problemu stawianego w~tej pracy to właśnie silnik JavaScript jest najważniejszym komponentem silnika renderującego. Silnik języka zajmuje się parsowaniem, kompilacją do bytecode'u, interpretowaniem oraz późniejszą optymalizacją kodu. Posiada wiele możliwości optymalizacji spekulatywnej z~racji na specyfikę języka, który jest dynamicznie typowany \cite{meurer_2017}.

Popularnymi silnikami renderującymi są Blink z~silnikiem JavaScript V8 \cite{V8} oraz Gecko z~silnikiem JavaScript SpiderMonkey \cite{spidermonkey}. Silniki JavaScript skupione są na szybkim i~efektywnym wykonywaniu kodu i~nie zajmują się asynchronicznością i~pętlą zdarzeń. Odpowiedzialne za to są równolegle biblioteki. W~przeglądarce Google Chrome pętlę zdarzeń implementuje biblioteka LibEvent \cite{libevent}.

\subsection{NodeJS}

NodeJS jest powstałym w~2009 roku środowiskiem, które było odpowiedzią na architekturę pozostałych rozwiązań serwerowych, które angażują wiele procesów i~wątków do obsługi wielu zapytań w~tym samym czasie. Rodzi to problemy związane z~koniecznością przełączania kontekstu pomiędzy procesami oraz większe zapotrzebowanie na pamięć. Również każda operacja wejścia-wyjścia musi być synchroniczna, co prowadzi do zablokowania całego procesu w~oczekiwaniu na odpowiedź \cite{nodejs}.

Problemy te rozwiązało środowisko NodeJS, napisane w~C++ i~oparte na silniku V8. Razem biblioteką LibUv \cite{libuv} implementującą pętlę zdarzeń w~ramach jednego procesu wykonującego kod JavaScript użytkownika oraz wielu wątków, które realizują oczekiwanie na operacje asynchroniczne, pozwoliło rozwiązać problem operacji wejścia-wyjścia pozwalając w~ramach tych samych zasobów sprzętowych osiągnąć lepsza wydajność niż popularny serwer Apache \cite{node-apache}. Na rysunku \ref{fig:nodejs} pokazano architekturę środowiska NodeJS, która stoi pomiędzy aplikacją, a~systemem operacyjnym.

\begin{figure}
    \centering
    

\begin{tikzpicture}

\node[fill=red!50, draw, rectangle, anchor=north west, minimum width=12cm, minimum height=0.75cm]  at (0,0.85) {Application};

\node[fill=green!50, draw, rectangle, anchor=north west, minimum width=12cm, minimum height=0.75cm] at (0,0) {NodeJS API};
\node[fill=blue!20, draw, rectangle, anchor=north west, minimum width=7.95cm, minimum height=0.75cm]  at (0,-0.85) {NodeJS Bindings};
\node[fill=blue!20, draw, rectangle, anchor=north west, minimum width=3.95cm, minimum height=0.75cm]  at (8.05,-0.85) {C/C++ Addons};
\node[fill=orange!50, draw, rectangle, anchor=north west, minimum width=1.9cm, minimum height=0.75cm] (A)  at (0,-1.7) {V8};
\node[fill=violet!50, draw, rectangle, anchor=north west, minimum width=1.9cm, minimum height=0.75cm, right=0.1 of A] (B) {LibUv};
\node[fill=blue!20, draw, rectangle, anchor=north west, minimum width=1.9cm, minimum height=0.75cm, right=0.1 of B] (C) {c-ares};
\node[fill=blue!20, draw, rectangle, anchor=north west, minimum width=1.9cm, minimum height=0.75cm, right=0.1 of C] (D) {llhttp};
\node[fill=blue!20, draw, rectangle, anchor=north west, minimum width=1.9cm, minimum height=0.75cm, right=0.1 of D] (E) {OpenSSL};
\node[fill=blue!20, draw, rectangle, anchor=north west, minimum width=1.9cm, minimum height=0.75cm, right=0.1 of E] (F) {zlib};

\node[fill=black!20, draw, rectangle, anchor=north west, minimum width=12cm, minimum height=0.75cm]  at (0,-2.55) {Operating system};

\end{tikzpicture}
 
    \caption{Model architektury środowiska NodeJS.}
    \label{fig:nodejs}
\end{figure}

\subsection{Deno}

Deno powstał w~2018 roku, a~jego wersja 1.0.0 wydana została w~2020 roku. Jest to środowisko aspirujące do bycia następcą NodeJS rozwiązując jego problemy związane z~bezpieczeństwem, systemem budowania zależności bibliotek, czy importowania zależności \cite{deno}. Podobnie jak NodeJS do wykonywania kodu JavaScript wykorzystuje silnik V8, ale napisany jest w~języku Rust. Do obsługi asynchroniczności i~pętli zdarzeń wykorzystuje bibliotekę Tokio \cite{tokio}. W~przeciwieństwie do NodeJS i~przeglądarek internetowych obsługuje natywnie TypeScript - nadzbiór języka JavaScript umożliwiający wykorzystanie statycznego typowania oraz wszystkie zalety za tym idące.

\section{Modularność i~kompatybilność kodu}

Kolejnym ważnym aspektem rozważań jest modularność i~kompatybilność kodu pomiędzy środowiskami. Rozwój bibliotek języków jest naturalnym krokiem ewolucji ich ekosystemów i~aby taki ekosystem był ogólnodostępny, biblioteki dostępne są dla wszystkich w~postaci scentralizowanego rejestru paczek. 

\subsection{Modularność}

Dla środowiska NodeJS najpopularniejszym rejestrem jest \textit{npm registry} (node package manager). Za pomocą narzędzia o~tej samej nazwie można instalować i~zatrzaskiwać wersję paczek, które trafiają do folderu \lstinline{node_modules}, w~którym środowisko NodeJS domyślnie poszukuje kodu podczas importu paczek. Zainstalowane paczki są opisane wraz z~ich wersją w~pliku \mbox{\lstinline{package.json}} i~\lstinline{package-lock.json}, gdzie pierwszy z~nich zawiera wymaganą wersję zapisaną w~konwencji Semver \cite{semver}, a~drugi zatrzaśnięte zainstalowane wersje, co pozwala odtworzyć dokładną strukturę zależności.

Przeglądarki internetowe z~kolei pobierają dodatkowe biblioteki poprzez umieszczenie tagów \lstinline{<script/>} w~dowolnym miejscu na stronie, często z~użyciem sieci CDN (ang. content delivery network). \textit{npm} jest również popularnym rozwiązaniem służącym do dostarczania modułów niezbędnych do funkcjonowania stronom internetowym, jednak odbywa się to pośrednio. Za pomocą narzędzi zwanych bundler'ami, ze wszystkich niezbędnych zależności - właściwego kodu strony oraz zewnętrznych bibliotek instalowanych przy pomocy narzędzia \textit{npm}, budowany jest pojedynczy plik, który następnie jest ładowany przez przeglądarkę. Pomaga to zaoszczędzić liczbę połączeń przeglądarki do serwerów, a~co za tym idzie zaoszczędzić czas spędzony na inicjowaniu połączenia i~pobieraniu danych w~szczególności, że liczba możliwych otwartych połączeń przez przeglądarkę internetową jest limitowana (10 w~Google Chrome). Często obecnie budowane są dwa lub więcej plików - jeden z~bibliotekami zewnętrznymi oraz jeden lub więcej z~właściwym kodem strony w~celu wykorzystania mechanizmów pamięci podręcznej przeglądarki oraz dynamicznego i~opóźnionego ładowania niezbędnego kodu, co przyspiesza ładowanie strony. Popularnymi bundler'ami są Webpack, Snowpack, Parcel, Rollup oraz Vite.

Deno próbuje rozwiązać problem importowania modułów w~NodeJS, który wynika z~scentralizowanego rejestru i~paczek oraz z~faktu istnienia plików \lstinline{package*.json} i~konieczności wykonania procesu instalacji. Pobiera on zależności bezpośrednio z~sieci i~umieszcza w~pamięci podręcznej. Link do zależności jest jej jednoznacznym identyfikatorem, a~to znaczy, że powinien zawierać wersję paczki i~nigdy nie zmienić swojej zawartości. Pobranie i~kompilacja zależności w~czasie wykonania programu likwiduje potrzebę przechowywania listy zależności i~ich wersji oraz niweluje potrzebę ich instalacji. Deno wykorzystuje opisany w~dalszej części rozdziału format modułów ESM.

Niezależnie od tego gdzie przetrzymywane są moduły oraz w~jaki sposób są zarządzane przez środowisko, muszą one być finalnie przez nie skonsumowane. Mogą być one użyte bezpośrednio, ale też przetransformowane w~procesie budowania biblioteki, która będzie potem dalej konsumowana, czy paczki dla przeglądarki internetowej. W~procesie ewolucji ekosystemu języka JavaScript wykształciło się wiele formatów modułów. Niektóre z~nich są kompatybilne tylko z~przeglądarką internetową, niektóre tylko ze środowiskiem NodeJS bądź Deno.  

\subsubsection{AMD}

AMD (Asynchronous Module Definition) jest sposobem ładowania zależności w~przeglądarkach internetowych. Rozwija wzorzec modułów JavaScript \cite{jsmodulepattern} poprzez dodanie asynchronicznego pobierania i~ładowania zależności. Moduł jest funkcją, dzięki czemu zadeklarowane zmienne nie wyciekają poza jej zakres, a~jej wartość zwracana stanowi wartość, którą taki moduł eksportuje. Zadeklarowanie modułu i~jego zależności w~formacie AMD umożliwia funkcja \lstinline{define}.

\subsubsection{CommonJS}

Format CommonJS utworzony został na potrzeby środowiska NodeJS i~jest tam do dzisiaj wykorzystywany. Używa on globalnie dostępnej funkcji \lstinline{require}, która jako argument przyjmuje nazwę modułu lub relatywną ścieżkę do pliku \lstinline{*.js}, jednak z~pominięciem jego rozszerzenia, co stanowi problem podczas wyszukiwania modułów przez środowisko. Moduł może eksportować funkcje i~wartości poprzez dodanie ich do obiektu \lstinline{module.exports}. Pliki z~modułami w~tym formacie, aby lepiej je identyfikować, mogą mieć rozszerzenie \lstinline{*.cjs}.

\subsubsection{UMD}

UMD (Universal Module Definition) nie stanowi samodzielnego formatu, ale integruje formaty AMD, CommonJS oraz użycie zmiennych globalnych do definicji modułu i~jego zależności. Wyewoluował on z~potrzeby tworzenia bibliotek kompatybilnych z~wieloma środowiskami, dla których nie trzeba budować wielu wersji w~różnych formatach.

\subsubsection{Moduły ECMAScript}

Brak kompatybilności i~wiele formatów modułów, gdzie każde środowisko zaproponowało swój własny, wymusiło ich standaryzację w~specyfikacji języka. ECMAScript, którego implementacją jest JavaScript, w~wersji 2015 (zwanej również ES6) wprowadza definicję modułów zwanych ESModules, ESM \cite{ESModules}. Obecnie wspierane są one przez wszystkie analizowane tutaj środowiska i~są zalecaną metodą importowania zależności. Używają one słów kluczowych \lstinline{import} oraz \lstinline{export} tak, jak zostało to pokazane na listingu \ref{lst:esm}. Od wersji ES11 specyfikacji możliwe stało się dynamiczne importowanie modułów podczas wykonania, gdzie jako rezultat otrzymujemy obiekt \mbox{\lstinline{Promise}}. Pliki z~modułami w~tym formacie, aby lepiej je identyfikować, mogą mieć rozszerzenie \mbox{\lstinline{*.mjs}}. Wszystkie biblioteki wykorzystujące metody akceleracji badane w~tej pracy budowane są z~wykorzystaniem ESM.

\begin{lstlisting}[language=JavaScript, caption=Przykład wykorzystania ECMAScript Modules, label=lst:esm]
// main.mjs
import { add } from "./module"
console.log(add(2+2));

// module.mjs
export function add(foo, bar) { return foo + bar; }
\end{lstlisting}

\subsection{Kompatybilność}

Szerokie starania w~standaryzacji modułów umożliwiają tworzenie kodu kompatybilnego z~wieloma środowiskami. Jeśli jednak kod ten korzysta z~funkcjonalności samego środowiska, która istnieje w~pozostałych środowiskach, ale ich API nie są ze sobą zgodne, problem kompatybilności między środowiskami wciąż występuje. Wprowadza to niechciane mechanizmy do kodu wykrywające środowisko i~wymusza wykorzystanie wzorców projektowych takich jak \textit{adapter}, w~celu obsługi wszystkich wariantów API.

Przykładem takiego rozwiązania jest biblioteka \textit{axios}, która służy do wykonywania zapytań HTTP. W~środowisku przeglądarki internetowej do wykonywania zapytań wykorzystuje obiekt \lstinline{XMLHttpRequest}, a~w środowisku NodeJS wbudowany moduł \textit{http}. Rozwiązaniem problemu w~tym przypadku może być użycie Fetch API, które jako pierwsze zadebiutowało w~przeglądarkach internetowych oraz zaadaptowane zostało przez NodeJS, a~w Deno jest ono domyślnie przewidzianą formą wykonywania zapytań HTTP.

Innym przykładem braku kompatybilności pomiędzy podobnymi funkcjonalnościami jest wielowątkowość, która istnieje pod abstrakcją Worker'ów i~dokładnie zostanie omówiona w~rozdziale \ref{sec:acc-methods}. Przeglądarki internetowe oraz Deno, który ma na celu możliwie zbliżyć się do nich ze swoim API, implementują Web Worker API. NodeJS z~kolei do obsługi Worker'ów wykorzystuje wbudowany moduł \textit{worker\_threads}, który różni się od jego odpowiedników w~pozostałych środowiskach.    

\section{Metody akceleracji}
\label{sec:acc-methods}

Akceleracja obliczeń jest niekończącą się pogonią za nieskończenie krótkim czasem wykonania algorytmów. Zdaniem autora powinna być ona brana pod uwagę na każdym etapie ich rozwoju - od etapu prototypowania, do wdrożeń produkcyjnych, ponieważ na każdym z~nich może przynieść wymierne korzyści. Wspomnianą w~poprzednim rozdziale akcelerację praca traktuje jako wszelkie metody przyspieszające wykonywanie algorytmu.

\subsection{Optymalizacja wykonania sekwencyjnego}

Pierwszym aspektem, na który trzeba zwrócić uwagę jest sama interakcja z~silnikiem języka i~wykorzystanie jego możliwości oraz tego, w~jaki sposób potrafi on optymalizować wykonywany kod. Poprzez kompilację kodu Just-In-Time (JIT), czyli bezpośrednio przed jego wykonaniem oraz możliwość powtarzania tego procesu wykorzystując heurystyki dla zebranych danych, możliwe jest przyspieszenie wykonania kodu \cite{meurer_2017}. W~przypadku silnika V8 kod najpierw trafia do interpretera o~nazwie Ignition, gdzie kompilowany jest do bytecode'u, który zamieniany jest na instrukcje zgodne z~architekturą procesora. Równolegle bytecode wysyłany jest do kompilatora TurboFan, gdzie przechodzi optymalizację na poziomie funkcji, w~przeciwieństwie do innych rozwiązań JIT, które identyfikują wielokrotnie wykonywaną część bytecode'u \cite{meurer_2019}. Optymalizacja ta jest ograniczona do funkcji o~maksymalnym rozmiarze bytecode'u równym 60KB, więc ważne jest, aby umiejętnie rozbijać kod algorytmu na funkcje i~moduły.

Osoba projektująca dany algorytm może celowo unikać konstrukcji języka oraz niektórych wzorców, aby wspomóc mechanizmy optymalizacji i~zwiększyć wydajność w~skrajnych przypadkach nawet o~25\% (\cite{gong2015jitprof}, \cite{selakovic2016performance}). Dobrą praktyką jest stosowanie typowanych tablic (Typed Array) do przechowywania danych binarnych oraz danych o~znanym typie. Innym sposobem poprawny wydajności jest modyfikacja algorytmu tak, aby zredukować obciążenie związane z~obliczeniami zmiennoprzecinkowymi, na przykład poprzez stosowanie stablicowanych wartości funkcji trygonometrycznych (LUT).

\subsection{Natywne moduły}

Środowiska NodeJS i~Deno działające po stronie serwera pozwalają na uruchomienie z~ich poziomu bibliotek skompilowanych do kodu maszynowego konkretnej architektury. Deno pozwala uruchomić funkcję ze skompilowanego kodu języka Rust do postaci biblioteki na różnych platformach (\lstinline{*.so}, \lstinline{*.dll}, \lstinline{*.dylib}). NodeJS natomiast posiada własny system budowania natywnych modułów z~kodu C++ o~nazwie \textit{node-gyp}, którego zależnością jest język Python.

Zaletą korzystania z~natywnych modułów jest możliwość wykorzystania metod optymalizacji specyficznych dla platformy sprzętowej, jakie oferują języki niskiego poziomu kompilowane do kodu maszynowego. Metodę tę można rozwinąć o~metody akceleracji dostępne w~językach źródłowych, których przykładem może być wielowątkowość z~wykorzystaniem biblioteki \lstinline{pthreads}. Do wad takiego rozwiązania zaliczyć można konieczność pobierania lub budowania natywnych zależności w~momencie instalacji biblioteki oraz ogólną kompatybilność, która wykracza poza środowisko języka JavaScript. Wadą, która wpływa w~znaczącym stopniu na wydajność w~specyficznych przypadkach jest brak bezpośredniego dostępu do pamięci silnika JavaScript, co skutkuje koniecznością kopiowania danych w~warstwie powiązania natywnego modułu z~kodem języka JavaScript. Dla dużych danych proces ten okazać się może wąskim gardłem algorytmu.

\subsection{WebAssembly}

Kolejną metodą akceleracji wykonania sekwencyjnego jest WebAssembly (WASM), który jest językiem niskiego poziomu. Stanowi on cel kompilacji języków takich jak C++, czy Rust i~pozwala uruchamiać w~środowisku webowym złożone aplikacje, których wcześniejsze uruchomienie nie było możliwe ze względu na konieczność parsowania i~kompilacji dużej ilości kodu. Umożliwiło to na przykład łatwe przeniesienie aplikacji napisanych w~języku C++ z~wykorzystaniem biblioteki Qt i~uruchomienie ich w~przeglądarce internetowej \cite{qt-wasm}. Kolejnym przykładem jest jeden z~backend'ów biblioteki Tensorflow.js \cite{tensorflowjs}, która umożliwia trenowanie i~wdrażanie modeli uczenia maszynowego. Dzięki bibliotece PyScript, która portuje implementację języka Python napisaną w~C do języka WebAssembly, możliwe stało się wykonywanie kodu języka w~przeglądarce internetowej \cite{pyscript}.

\begin{lstlisting}[language=C++, caption=Funkcja licząca silnię w~języku C/C++, label=lst:factorial-cpp]
int factorial(int n) {
    if (n == 0)
        return 1;
    else
        return n * factorial(n-1);
}
\end{lstlisting}
    
\begin{lstlisting}[language=WASM, caption=Funkcja licząca silnię w~języku WASM, label=lst:factorial-wasm]
(func (param i64) (result i64)
    local.get 0
    i64.eqz
    if (result i64)
        i64.const 1
    else
        local.get 0
        local.get 0
        i64.const 1
        i64.sub
        call 0
        i64.mul
    end)
\end{lstlisting}

Wydajność kodu WebAssembly zbliżona jest do wykonania natywnego. Moduły operują na liniowym modelu danych, który nie jest współdzielony z~kodem języka JavaScript. Podobnie jak w~przypadku natywnych modułów występuje konieczność transformacji i~kopiowania danych do pamięci modułu, co może rodzić problemy z~wydajnością. Na listingu \ref{lst:factorial-cpp} przedstawiono funkcję liczącą silnię, która na listingu \ref{lst:factorial-wasm} została zapisana w~jej odpowiedniku w~WASM. WebAssembly przetwarzany jest w~postaci binarnej i~działa jak maszyna stosowa, a~na potrzeby analizy zapisywany jest w~formacie tekstowym w~postaci S-wyrażeń.

\subsubsection{asm.js}

Poprzednikiem WebAssembly był asm.js - podzbiór języka JavaScript w~procesie kompilacji z~języków źródłowych zoptymalizowany pod kątem wydajności wykonania i~specjalnie interpretowany przez przeglądarki wykorzystując kompilację Ahead-Of-Time. Kompilację tę wspiera do dziś jedynie przeglądarka Firefox, jednak optymalizacje silnika V8 Google Chrome 28 zapewniły dwukrotny wzrost wydajności podczas wykonania asm.js \cite{asm.js-chrome}. Nie jest on już rozwijany i~został wyparty przez WebAssembly.

\subsubsection{SIMD}

WebAssembly jest w~stanie wykorzystać architekturę SIMD (Single Instruction, Multiple Data), używając rejestrów i~instrukcji wektorowych \cite{wasm-simd}. Kompilatory takie jak LLVM potrafią, w~procesie auto-wektoryzacji wykorzystać instrukcje wektorowe, aby przyspieszyć działanie pętli oraz połączyć inne masowe operacje logiczne i~arytmetyczne. Wektoryzację taką można również przeprowadzić ręcznie wykorzystując funkcje, które operują bezpośrednio na typach wektorowych.

\subsection{Współbieżność}

W środowisku przeglądarki internetowej współbieżność osiągnięta może być tylko na poziomie wielu wątków, poprzez zastosowanie Worker'ów. Każdy Worker funkcjonuje jako osobny wątek w~ramach tego jednego procesu. Posiada on własną pętlę zdarzeń i~z głównym wątkiem komunikuje się jedynie asynchronicznymi wiadomościami. Poza obiektem \lstinline{SharedArrayBuffer}, wspieranego przez interfejs operacji atomowych \lstinline{Atomic}, aby zapobiec problemowi wyścigów, Worker'y nie współdzielą pamięci. Dane w~wiadomościach przekazywane są z~użyciem algorytmu klonowania strukturalnego, który wspiera natywne typy takie jak tablice, mapy, czy sety \cite{structured-clone}. Nie możliwe jest natomiast kopiowanie funkcji zdefiniowanych przez użytkownika oraz węzłów drzewa DOM, ponieważ tylko wątek główny może być odpowiedzialny za renderowanie widoku strony. Aby uniknąć procesu klonowania dużych zmiennych, które chcemy przenieść do Worker'a, i~które nie są potrzebne nam w~wątku głównym, możemy użyć obiektów \lstinline{Transferrable}, które jeśli wskazane, zostają przeniesione do kontekstu Worker'a, a~nie skopiowane, oszczędzając w~ten sposób pamięć i~czas procesora.

W środowiskach działających po stronie serwera współbieżne wykonanie możliwe jest przy zastosowaniu Worker'ów, ale również poprzez uruchomienie podprocesów, z~którymi można komunikować się poprzez standardowe strumienie wejścia oraz wyjścia, a~w przypadku NodeJS, również poprzez mechanizm wiadomości, który zajmuje się ich automatycznyą serializacją i~parsowaniem. 
 
\subsection{GPGPU}

GPGPU (ang. General-purpose computing on graphics processing units) zakłada użycie układów graficznych do wykonywania obliczeń ogólnego przeznaczenia, które do tej pory wykonywane były na CPU. W~środowiskach desktopowych, możemy wykorzystać takie układy używając rozwiązań powiązanych z~architekturą układu, czego przykładem jest NVIDIA CUDA. Możemy również skorzystać z~framework'ów implementujących warstwę abstrakcji będąc kompatybilnymi z~wieloma platformami i~architekturami jak na przykład OpenCL. 

Środowiska webowe zorientowane są na jak największą kompatybilność pomiędzy platformami, a~pierwotnie interakcja z~układami graficznymi możliwa była tylko w~środowisku przeglądarki internetowej poprzez wykorzystanie WebGL API do generowania grafiki w~elemencie \lstinline{<canvas/>}. Dla obliczeń ogólnego przeznaczenia stworzono standard WebCL, jednak nie zyskał on popularności i~nie był powszechnie implementowany. Sposobem, który okazał się skuteczny, aby użyć układ graficzny do innych rzeczy niż generowanie grafiki, paradoksalnie okazało się wykorzystanie procesów odpowiedzialnych za generowanie grafiki \cite{sapuan2018general}. Każdy piksel generowanego obrazu stanowić może pojedynczy wynik działania kernela czyli funkcji, której działanie jest masowo zrównoleglane. Dostarczenie danych wejściowych w~postaci tekstur oraz konstrukcja programu \texttt{Fragment Shader} wyliczającego kolor każdego wynikowego piksela pozwala wykonać obliczenia w~takiej samej abstrakcji jak dedykowane rozwiązania. 

\begin{figure}
    \centering
    \begin{tikzpicture}

\node[draw, rectangle, minimum height=0.9cm] (VBO) {Vertex Buffer Objects};
\node[draw, rectangle, minimum height=0.9cm, right=2 of VBO] (Attr) {Attributes};

\path [draw, latex-, ultra thick,dotted] (VBO) -- node[above] {Points to} (Attr);

\node[draw, rectangle, minimum height=0.9cm, below=1 of Attr] (VS) {Vertex Shader};
\path [draw, -latex, line width=0.1cm] (Attr) -- node[right] {Input} (VS);


\node[draw, rectangle, minimum height=0.9cm, below=1 of VS] (FS) {Fragment Shader};
\path [draw, -latex, line width=0.1cm] (VS) -- node[right] {} (FS);

\node[draw, rectangle, minimum height=0.9cm, below=1 of FS] (FB) {Frame Buffer};
\path [draw, -latex, line width=0.1cm] (FS) -- node[right] {} (FB);

\node[draw, rectangle, minimum height=0.9cm, left=2 of VS] (Uniforms) {Uniforms};
\path [draw, latex-, ultra thick] (VS) -- node[above] {Input} (Uniforms);
\draw [draw, latex-, ultra thick] (FS) -- node[above] {Input} (Uniforms|-FS);
\draw [draw, -, ultra thick] (Uniforms) -- node[above] {} (Uniforms|-FS);

\node[draw, rectangle, minimum height=0.9cm, right=2 of VS] (Varyings) {Varyings};
\path [draw, -latex, ultra thick] (VS) -- node[above] {Output} (Varyings);
\draw [draw, latex-, ultra thick] (FS) -- node[above] {Input} (Varyings|-FS);
\draw [draw, -, ultra thick] (Varyings) -- node[above] {} (Varyings|-FS);

\end{tikzpicture}

    \caption{Potok graficzny WebGL API, który może być wykorzystany do obliczeń ogólnego przeznaczenia.}
    \label{fig:webgl-pipeline}
\end{figure}

Na rysunku \ref{fig:webgl-pipeline} przedstawiono najważniejsze elementy potoku graficznego WebGL API. Na podstawie atrybutów, które wskazują na bufory z~danymi wierzchołków, program \textit{Vertex Shader} dla każdego z~nich oblicza ich współrzędne. Następnie po procesie transformacji wierzchołków na prymitywy (najczęściej trójkąty) i~ich rasteryzacji, program \textit{Fragment Shader} zajmuje się obliczeniem koloru każdego piksela wynikowego obrazu na podstawie interpolowanych wartości dostarczanych w~postaci \textit{Varyings}. Omijając etap związany z~pozycją wierzchołków i~wymuszając tylko kolorowanie właściwej liczby pikseli, możemy przenieść obliczenia do programu \textit{Fragment Shader}, gdzie dane wejściowe dostarczane są w~postaci tekstur za pomocą \textit{Uniforms}, czyli stałych dla całego potoku.

Podejście to, z~racji na specyficzność tego potoku, ma swoje ograniczenia. W~przeciwieństwie do pozostałych rozwiązań GPGPU jedynym wynikiem działania kernela jest wartość piksela. Niemożliwe zatem jest zapis do pamięci współdzielonej w~trakcie wykonywania algorytmu. Wąskie gardło wydajności stanowi odczyt wyników. Proces wysłania bufora ramki i~wyświetlenia go na elemencie \lstinline{<canvas/>} jest zoptymalizowany, jednak pobranie go z~GPU do postaci typowanej tablicy w~języku JavaScript jest już kosztowne czasowo.

W środowisku przeglądarki internetowej dostępny jest WebGL, a~co za tym idzie, istnieje możliwość wykorzystanie tej metody akceleracji obliczeń. Środowiska NodeJS oraz Deno, jako rozwiązania serwerowe, nie skupiły się na mechanizmach generowania grafiki. W~środowisku NodeJS istnieją jednak biblioteki implementujące kontekst WebGL, na przykład \textit{headless-gl} \cite{headless-gl}. Środowisko Deno, z~racji na swój młody wiek, na chwilę obecną nie posiada implementacji natywnej, jak i~w postaci bibliotek. 

\subsubsection{WebGPU}

Popularyzacja i~potrzeba tworzenia coraz bardziej wydajnych aplikacji webowych, sprowokowała powstanie specyfikacji WebGPU API, która stanowi uogólnienie przetwarzania masowo równoległego dostarczając bezpośrednio abstrakcję GPU \cite{webgpu_2022}. WebGPU może być wykorzystane do obliczeń ogólnego przeznaczenia, ale również do generowania grafiki. Obecnie jest jednak wciąż dostępne jako funkcjonalność eksperymentalna i~do działania w~środowisku przeglądarki internetowej oraz Deno potrzebuje specjalnej flagi. Środowisko NodeJS nie implementuje \mbox{WebGPU}.

\chapter{Transformacja Hough'a}
\label{sec:hough}

Transformacja Hough'a (czyt. Hafa) wykorzystywana jest w~procesie analizy obrazów i~służy do wykrywania na nim kształtów parametrycznych oraz nieparametrycznych w~zależności od jej wariantu \cite{mukhopadhyay2015survey}. Samo pojęcie transformacji odnosi się do odwzorowywania pojedynczych pikseli obrazu binarnego lub ich zbioru w procesie głosowania w~przestrzeni akumulatora. Obraz wejściowy wcześniej poddany być musi procesowi wykrywania krawędzi. Dane zebrane w~akumulatorze biorą następnie udział w~procesie, w~którym wyłonione zostają potencjalne kształty poprzez wykrywanie największych wartości w~akumulatorze. W~zależności od specyfiki problemu oraz wykrywanych kształtów wykrywanie maksimów może odbywać się na różne sposoby. Użyte może zostać proste progowanie, wykrywanie i~uśrednianie skupisk, czy też filtracja przestrzeni akumulatora. Ogólny schemat przetwarzania przedstawiony jest na rysunku \ref{fig:hough}.

\begin{figure}
    \centering
    \begin{tikzpicture}


\node[draw, rectangle, minimum height=0.9cm] (e) {Edge detection};
\node[draw, rectangle, minimum height=0.9cm, right=0.9cm of e] (t) {Hough transform};
\node[draw, rectangle, minimum height=0.9cm, right=0.9cm of t] (v) {Voting};
\node[draw, rectangle, minimum height=0.9cm, right=0.9cm of v] (p) {Peak detection};

\path [draw, latex-o, ultra thick] (e) -- node[right] {Input image} ++(0,2);
\path [draw, -latex, ultra thick] (e) -- node[right] {} (t);
\path [draw, -latex, ultra thick] (t) -- node[right] {} (v);
\path [draw, -latex, ultra thick] (v) -- node[right] {} (p);
\path [draw, -latex, ultra thick] (p) -- node[left] {Detected patterns} ++(0,2);

\end{tikzpicture}

    \caption{Ogólny schemat przetwarzania obrazu z~wykorzystaniem transformacji Hough'a.}
    \label{fig:hough}
\end{figure}

\section{Standard Hough Transform}

Pracą, która jako pierwsza opisała tę transformację jest zgłoszony w~1962r. patent Paula Hough'a \cite{hough1962method}. Opisał on wykrywanie linii poprzez zastosowanie odwzorowania PTLM (point-to-line mapping). Odwzorowanie to dla każdego piksela rysuje linię w~dwuwymiarowej przestrzeni akumulatora zgodnie z~kierunkowym równaniem prostej z~równania (\ref{eq:hough-1}) przekształcone do postaci (\ref{eq:hough-2}). Stosując odwzorowanie odwrotne dla punktów o~największych wartościach możemy otrzymać potencjalne linie na obrazie. Transformację stosującą pełne odwzorowanie wszystkich punktów obrazu na parametry kształtów nazywamy standardową transformacją Hough'a (Standard Hough Transform, SHT).
\begin{align}
    y(x) &= mx+c \label{eq:hough-1}\\
    c(m) &= -xm+y \label{eq:hough-2}
\end{align}
\begin{eqexpl}
    \item{$x, y$} współrzędne piksela na obrazie;
    \item{$m$} zbocze prostej;
    \item{$c$} punkt przecięcia prostej z~osią Y.
\end{eqexpl}

\begin{figure}
    \centering
    \begin{subfigure}{0.45\textwidth}
        \centering
        \begin{tikzpicture}[scale=0.8]
    \begin{scope}[yscale=-1] 
    \tkzInit[xmax=6,ymax=6,xmin=0,ymin=0]
    \tkzGrid[]
    \tikzset{xlabel style/.append style={above=5pt}}
    \tkzAxeXY[]
    \draw[gray, ultra thick] (-0.5,0.5) -- (5.5,6.5);
    \fill (0, 1)  circle[radius=2pt];
    \fill (1, 2)  circle[radius=2pt];
    \fill (2, 3)  circle[radius=2pt];
    \fill (3, 4)  circle[radius=2pt];
    \fill (4, 5)  circle[radius=2pt];
    \fill (5, 6)  circle[radius=2pt];
    \end{scope}
\end{tikzpicture}

    \caption{Binarny obraz wejściowy}\label{fig:houghSlopeA}
    \end{subfigure}\hfill
    \begin{subfigure}{0.45\textwidth}
        \centering
        \begin{tikzpicture}[scale=0.8]
    \begin{scope}[yscale=-1] 
    \tkzInit[xmax=6,ymax=6,xmin=0,ymin=0]
    \tkzGrid[]
    \tikzset{xlabel style/.append style={above=5pt}}
    \tkzAxeXY[]
    \draw[gray, ultra thick] (-0.5,1) -- (6.5,1);
    \draw[gray, ultra thick] (-0.5,2.5) -- (2.5,-0.5);
    \draw[gray, ultra thick] (-0.5,4) -- (1.75,-0.5);
    \draw[gray, ultra thick] (-0.5,5.5) -- (1.5,-0.5);
    \draw[gray, ultra thick] (-0.375,6.5) -- (1.375,-0.5);
    \draw[gray, ultra thick] (-0.1,6.5) -- (1.3,-0.5);
    \end{scope}
\end{tikzpicture}

    \caption{Wynik transformacji}\label{fig:houghSlopeB}
    \end{subfigure}

    \caption{Demonstracja transformacji Hough'a w~wariancie równania kierunkowego prostej.}
    \label{fig:houghSlope}
\end{figure}
\hspace{1cm}


Na rysunku \ref{fig:houghSlope} przedstawiono obraz wejściowy (rys. \ref{fig:houghSlopeA}) oraz wynik transformacji (rys. \ref{fig:houghSlopeB}). Na rysunkach reprezentujących obraz oś Y jest skierowana w~dół, co ułatwia interpretację w~zgodności ze sposobem indeksowania pikseli obrazów podczas ich przetwarzania, gdzie punkt $(0,0)$ znajduje się w~lewym górnym rogu. W~wyniku transformacji każdy jasny piksel obrazu $(x, y)$ został odwzorowany na linię zgodnie z~równaniem (\ref{eq:hough-2}). Linie te przecięły się w~jednym punkcie $(1,1)$. W~wariancie dyskretnym transformacji wartość akumulatora w~punkcie $(1,1)$ miałaby największą wartość. W~tym wypadku punkt $(1,1)$ akumulatora przekłada się na prostą o~równaniu $y = x+1$, co zgadza się z~prostą na rysunku \ref{fig:houghSlopeA}.
\begin{wrapfigure}{r}{7cm}
    \centering
    

\begin{tikzpicture}[scale=0.9]

    \begin{scope}[yscale=-1] 
    \tkzInit[xmax=5.5,ymax=5.5,xmin=0,ymin=0]
    \tkzGrid[]
    \tikzset{xlabel style/.append style={above=5pt}}
    \tkzAxeXY
    \tkzClip[space=1]

    \draw[gray, ultra thick] (-0.5 ,5.5) -- (5.5,-0.5);

    \fill (0, 5)  circle[radius=2pt];
    \fill (1, 4)  circle[radius=2pt];
    \fill (2, 3)  circle[radius=2pt];
    \fill (3, 2)  circle[radius=2pt];
    \fill (4, 1)  circle[radius=2pt];
    \fill (5, 0)  circle[radius=2pt];

    \tkzDefPoints{0/0/O,2.5/2.5/B,5/0/A}
    \tkzMarkAngle[size=0.5](A,O,B)
    \tkzFillAngle[size=0.5,fill=red!20, opacity=0.5](A,O,B)
    \tkzLabelAngle(A,O,B){$\theta$}

    \tkzMarkRightAngle[german, size=0.45](O,B,A)

    \draw[gray, thick, dashed] (0,0) -- (2.5,2.5) node [midway, right, color=black] {$\rho$}; ;
\end{scope}

\end{tikzpicture}

    \caption{Prosta opisana za pomocą odległości $\rho$ i~kąta $\theta$ od środka układu współrzędnych biegunowych.}
    \label{fig:houghLineAngle}
\end{wrapfigure}




Taka reprezentacja punktu w~przestrzeni akumulatora rodzi jednak problem w~przypadku wykrywania linii pionowych. Piksele obrazu zorientowane w~pionie w~przestrzeni akumulatora utworzą linie równoległe. Brak punktu przecięcia takich linii uniemożliwia wykrycie linii na obrazie. Rozwiązaniem tego problemu jest zaproponowana w~1972 roku zmiana reprezentacji prostej, gdzie zamiast zbocza i~punktu przecięcia z~osią Y użyto biegunowego układu współrzędnych oraz prostej normalnej do wykrywanej prostej \cite{duda1972use}. Przykładowa prosta została zaprezentowana na rysunku \ref{fig:houghLineAngle} i~reprezentowana jest przez wartości odległości $\rho=\frac{5\sqrt{2}}{2}$ i~kąta obrotu $\theta=\frac{\pi}{4}$ wokół środka układu współrzędnych. Prosta odwzorowywana jest na sinusoidę zgodnie z~równaniem \ref{eq:sinCos}.
\begin{align}
    \rho(\theta) &= x\cos{\theta} + y\sin{\theta} \label{eq:sinCos}
\end{align}
\begin{eqexpl}
    \item{$x, y$} współrzędne piksela na obrazie;
    \item{$\rho$} odległość prostej od środka układu współrzędnych;
    \item{$\theta$} obrót prostej od wokół układu współrzędnych.
\end{eqexpl}

Na rysunku \ref{fig:houghSinCos} przedstawiono zawartość akumulatora po transformacji obrazu z~rysunku \ref{fig:houghSlopeA}. Zgodnie z~oczekiwaniami sinusoidy te przecinają się w~punktach, które reprezentują wykryte linie. Dwa punkty $(\frac{3\pi}{4}, \frac{\sqrt{2}}{2})$ oraz $(\frac{7\pi}{4}, -\frac{\sqrt{2}}{2})$, z~racji na okresowość funkcji trygonometrycznych reprezentują tę samą linię. Przestrzeń akumulatora dla kąta obrotu można zatem ograniczyć do $\theta \in \left[ 0, \pi\right)$ \cite{immerkaer1998some}.

\begin{figure}[t]
    \centering
    \begin{tikzpicture}[xscale=2]
    \tkzInit[xmax=2*pi,ymax=3,xmin=0,ymin=-3]
    % \tkzHLine[color=red, dashed]{sqrt(2)/2}
    % \tkzHLine[color=red, dashed]{-sqrt(2)/2}
    \tkzGrid[xstep=pi/4]
    \tkzClip[space=1]
    \tkzDrawX[label=$\theta$,trig=4]
    \tkzLabelX[trig=4, below=12pt]
    \tkzDrawY[label=$\rho$]
    \tkzLabelY

    \draw plot[domain=0:2*pi,smooth] (\x,{0*cos(\x r) + 1*sin(\x r)});
    \draw plot[domain=0:2*pi,smooth] (\x,{1*cos(\x r) + 2*sin(\x r)});
    \draw plot[domain=0:2*pi,smooth] (\x,{2*cos(\x r) + 3*sin(\x r)});
    \draw plot[domain=0:2*pi,smooth] (\x,{3*cos(\x r) + 4*sin(\x r)});
    \draw plot[domain=0:2*pi,smooth] (\x,{4*cos(\x r) + 5*sin(\x r)});
    \draw plot[domain=0:2*pi,smooth] (\x,{5*cos(\x r) + 6*sin(\x r)});
\end{tikzpicture}

    \caption{Wynik transformacji Hough'a dla obrazu na rysunku \ref{fig:houghSlopeA} w~wariancie współrzędnych biegunowych.}
    \label{fig:houghSinCos}
\end{figure}


W czasie transformacji, na końcowy rezultat mają wpływ szumy, które powstały na skutek niedoskonałości procesu wykrywania krawędzi, są bardzo krótkimi krawędziami lub są krawędziami, ale nieuwzględnianymi w~rozwiązywanym problemie. Przykładem mogą być okręgi podczas wykrywania linii na obrazie. Proces głosowania w~przestrzeni akumulatora musi uwzględniać takie sytuacje i~proste progowanie może zostać zastąpione bardziej złożonymi metodami \cite{palmer1997optimizing, perantonis1998robust}.

Na potrzeby prowadzonych badań zaimplementowany został algorytm wykrywania prostych na~obrazie binarnym, który jako wykrywanie maksimów w~akumulatorze wykorzystuje proste progowanie.

\section{Circle Hough Transform}

Prosta do swojej reprezentacji potrzebuje dwóch parametrów - zbocza i~punktu przecięcia z~osią Y. Okrąg natomiast jest kształtem parametrycznym, które potrzebuje trzech parametrów - współrzędnych środka i~promienia. Idąc dalej, możemy rozszerzać liczbę parametrów, zyskując możliwość stosowania transformacji Hough'a do wykrywania coraz bardziej skomplikowanych kształtów.

Akumulator w~Circle Hough Transform (CHT) ma trzy wymiary, po jednym na każdy parametr tak samo jak w~SHT, która do wykrywania linii wykorzystywała dwuwymiarowy akumulator. Dla każdego z~możliwych promieni, dla każdego piksela obrazu w~przestrzeni akumulatora rysujemy okrąg, którego ten piksel jest środkiem. Efektywnie jeden piksel obrazu mapowany jest na stożek w~trójwymiarowej przestrzeni akumulatora. Na rysunku \ref{fig:houghCircleStandard} przedstawiono przykładowe okręgi  w~dwóch wymiarach dla stałego promienia. Wszystkie te narysowane okręgi przecinają się w~środku właściwego okręgu \cite{mukhopadhyay2015survey}. W~takiej trójwymiarowej przestrzeni akumulatora jego największe wartości wskazywać będą na konkretne środki oraz promienie potencjalnych okręgów na obrazie wejściowym.

\begin{figure}
    \centering
    \begin{subfigure}{0.45\textwidth}
        \centering
        \begin{tikzpicture}[scale=0.9]
    \begin{scope}[yscale=-1] 
    \tkzInit[xmax=6.2,ymax=6.2,xmin=0,ymin=0]
    \tkzClip[space=1]
    \tkzGrid[]
    \tikzset{xlabel style/.append style={above=5pt}}
    \tkzAxeXY[]


        \tkzDefPoints{3/3/O,3/5/C}
        \tkzDefCircle[through](O,C)
        \tkzGetLength{rOC}
        \tkzDrawCircle[dashed](O,C)
        \tkzDrawPoints(O)
    
        \tkzDefPoints{3/1/B}
        \tkzGetLength{rOB}
        \tkzDefCircle[through](B,O)
        \tkzDrawPoints(B)
        \tkzDrawCircle(B,O)

        \tkzDefPoints{5/3/B}
        \tkzGetLength{rOB}
        \tkzDefCircle[through](B,O)
        \tkzDrawPoints(B)
        \tkzDrawCircle(B,O)

        \tkzDefPoints{3/5/B}
        \tkzGetLength{rOB}
        \tkzDefCircle[through](B,O)
        \tkzDrawPoints(B)
        \tkzDrawCircle(B,O)

        \tkzDefPoints{1/3/B}
        \tkzGetLength{rOB}
        \tkzDefCircle[through](B,O)
        \tkzDrawPoints(B)
        \tkzDrawCircle(B,O)

    
    \end{scope}
\end{tikzpicture}

        \caption{CHT dla stałego promienia w~wariancie standardowym.}
        \label{fig:houghCircleStandard}
    \end{subfigure}\hfill
    \begin{subfigure}{0.45\textwidth}
        \centering
        \begin{tikzpicture}[scale=0.9]
    \begin{scope}[yscale=-1]
    \tkzInit[xmax=6.2,ymax=6.2,xmin=0,ymin=0]
    \tkzClip[space=1]
    \tkzGrid[]
    \tikzset{xlabel style/.append style={above=5pt}}
    \tkzAxeXY[]

        \tkzDefPoints{3/3/O,3/5/C}
        \tkzDefCircle[through](O,C)
        \tkzGetLength{rOC}
        \tkzDrawCircle[dashed](O,C)
        \tkzDrawPoints(O)
    
        \tkzDefPoint(3,3){A}\tkzDefPoint(0,0){O}
        \tkzDefShiftPoint[A](0:4){B}
        \tkzDefShiftPoint[A](90:4){C}
        \tkzDefShiftPoint[A](180:4){D}
        \tkzDefShiftPoint[A](270:4){E}
        \tkzDefShiftPoint[A](45:4){F}
        \tkzDefShiftPoint[A](135:4){G}
        \tkzDefShiftPoint[A](225:4){H}
        \tkzDefShiftPoint[A](-45:4){I}
        \tkzDrawSegments[ultra thick, opacity=0.5](A,B A,C A,D A,E A,F A,G A,H A,I)
  
    
    \end{scope}
\end{tikzpicture}

        \caption{CHT w~wariancie gradientowym dla wybranych kierunków.}
        \label{fig:houghCircleGradient}
        \vfill
    \end{subfigure}
    \caption{Wynik transformacji}\label{fig:houghCircle}
\end{figure}

\subsection{Wariant z~wykorzystaniem gradientu}

Wraz ze skomplikowaniem kształtu parametrycznego rośnie liczba jego parametrów, których liczba stanowi liczbę wymiarów akumulatora. Dla każdego jasnego piksela obrazu konieczne jest uaktualnienie akumulatora we wszystkich jego wymiarach, co prowadzi do zwiększenia wykładniczej złożoności obliczeniowej. Dlatego dąży się do redukcji wymiarowości problemu zazwyczaj łącząc przetwarzanie charakterystyczna dla SHT z~dodatkową informacją z~przestrzeni obrazu. Biblioteka OpenCV implementuje mniej złożony wariant transformacji \cite{ito2012detection}, który oparty jest na wykorzystaniu gradientu wykrytych krawędzi. W~wariancie tym najpierw w~dwuwymiarowym akumulatorze następuje głosowanie nad centrum potencjalnego okręgu.

W pierwszej kolejności binarny obraz poddawany jest operacji splotu z~filtrem Sobela osobno w~kierunku pionowym oraz poziomym. Pozwala to na uzyskanie pochodnych cząstkowych w~danym punkcie, które razem tworzą gradient, czyli prostopadły kierunek przebiegu krawędzi potencjalnie wskazujący środek wyszukiwanego okręgu. W~dwuwymiarowym akumulatorze od analizowanego punktu w~dwóch kierunkach rysowana jest linia o~długości maksymalnego poszukiwanego promienia. W~przypadku okręgu wszystkie te linie przecinają się w~jednym punkcie, co pokazane zostało na rysunku \ref{fig:houghCircleGradient}. Kolejnym krokiem jest wykrycie maksimów w~akumulatorze. Aby zmniejszyć szumy i~liczbę okręgów leżących blisko siebie, po procesie progowania środki leżące bliżej siebie niż ustalona odległość są łączona w~jeden. Następnie dla każdego punktu po wykryciu maksimów w~akumulatorze, w~drugim jednowymiarowym akumulatorze, dla każdego możliwego poszukiwanego promienia następuje głosowanie. Zliczana jest liczba jasnych pikseli obrazu, które znajdują się w~danej odległości od środka, co po wykryciu maksimum w~akumulatorze uzupełnia dane okręgu o~najbardziej prawdopodobny promień.

Na potrzeby prowadzonych badań zaimplementowany został algorytm wykrywania okręgów na~obrazie binarnym, który wykorzystuje metodą gradientów oraz jako wykrywanie maksimów w~akumulatorze podczas głosowania dla środków jak i~promieni wykorzystuje proste progowanie.

\subsection{Próbkowanie i~złożoność obliczeniowa}

Głównym elementem wpływającym na złożoność obliczeniową transformacji Hough'a jest liczba analizowanych parametrów kształtów. Dla SHT złożoność wynosi $O(n)$ dla procesu głosowania i~$O(S_\rho S_\theta)$ dla procesu wykrywania maksimum, gdzie $n$ jest liczbą jasnych pikseli obrazu, a~$S_\rho$ i~$S_\theta$ parametrami próbkowania parametrów w~przestrzeni akumulatora. Widać zatem, że złożoność w~tym wypadku zależy od jakości danych wejściowych, gdzie wszelkie szumy zwiększają czas wykonania algorytmu. Kolejnym elementem jest próbkowanie w~przestrzeni akumulatora. Zwiększenie próbkowania pozwala uzyskać większą precyzję detekcji, ale zwiększa liniowo (dla jednego wymiaru) rozmiar akumulatora, a~co za tym idzie ilość obliczeń wymaganych w~procesie głosowania. Ważnym czynnikiem jest specyfika problemu, który rozwiązywać ma transformacja Hough'a. Możemy zmniejszyć czas wykonania algorytmu ograniczając zakres poszukiwań głosując w~ustalonej podprzestrzeni akumulatora, na przykład analizując linie nachylone tylko pod określonym zakresem kątów i~leżące w danej odległości od punktu odniesienia.

Transformacja Hough'a użyta została w~algorytmach wspólnych dla wszystkich testów w~środowiskach, które zostały przystosowane do badanych metod akceleracji. Wybrana została do tego celu ze względu na swoją złożoność obliczeniową, podział na wiele etapów oraz, w~wariancie SHT, wykorzystania funkcji trygonometrycznych. Operuje ona również na dużych zbiorach danych, co dodatkowo pozwala rzucić światło na konieczność zarządzania pamięcią i~transferu danych dla wybranych metod akceleracji. 

\chapter{Metodologia pomiarów}

Testowanie wydajności języka JavaScript, który głównie wykorzystywany jest w~przeglądarkach internetowych, z~racji na ich szeroką kompatybilność oraz mnogość środowisk jest szczególnie problematyczny \cite{v8-real-perf}. Rozdział ten porusza problematykę testowania syntetycznego oraz rzeczywistego. Opisuje również bibliotekę stworzoną na potrzeby prowadzonych badań.

Pierwotnie testy wydajności wykonywane były za pomocą testów syntetycznych, mikrobenchmarków - krótkich fragmentów kodu, które miały określić wydajność pojedynczej lub małego podzbioru funkcjonalności języka, na przykład porównując wydajność zwykłych tablic \mbox{\lstinline{Array}} do tablic typowanych, której przykładem jest obiekt \lstinline{Int8Array}. Popularnym narzędziem do budowanie takich benchmarków była strona jsPerf \cite{jsperf}, która obecnie nie jest już utrzymywana.

Rosnąca liczba API i~elementów ekosystemu wykorzystująca coraz bardziej złożone mechanizmy doprowadziła do ewolucji mikrobenchmarków do statycznych zestawów testów. Przykładami takowych są wspomniany wcześniej Ostrich \cite{ostrich}, który wykonuje różnego rodzaju algorytmy numeryczne takie jak algorytm Bauma-Welcha, czy szybką transformację Fouriera. Innymi przykładami są benchmarki JetStream 2 oraz Octane sprawdzające, oprócz ogólnych algorytmów takich jak algorytmy sortowania, elementy specyficzne dla ekosystemu JavaScript takie jak czas kompilacji kompilatora TypeScript, działanie WebAssembly, czy wyrażeń regularnych \cite{octane, jetstream}.

Jednak w~przeglądarkach internetowych z~perspektywy ich głównego przeznaczenia liczy się wygoda użytkowania. Czas poświęcony na wykonywanie samego kodu JavaScript, razem z~pobocznymi procesami takimi jak Garbage Collector, kompilacja i~optymalizacja stanowią ok 40\% całego nakładu obliczeń przeglądarki, która musi oprócz tego parsować i~renderować DOM oraz reagować na zdarzenia. Popularnym narzędziem do pomiarów wydajności strony z~perspektywy czasu ładowania i~renderowania jest Google Lighthouse \cite{lighthouse}. Popularnymi metrykami używanymi w~tego typu pomiarach jest First Contentful paint, czyli czas do narysowania czegokolwiek na ekranie, czy Largest Contentful Paint, czyli czas po którym nastąpiła największe przerysowanie elementów strony. W~środowiskach serwerowych problem ładowania strony naturalnie nie występuje. Mierzy się natomiast czas zimnego startu, czyli czasu wykonania kodu razem z~czasem uruchomienia samego środowiska. Ma to szczególne znaczenie w~środowiskach \textit{serverless} takich jak AWS Lambda \cite{aws-lambda}.

Prowadząc badania w~ramach tej pracy nie musimy skupiać się na metrykach czasu ładowania strony lub uruchamiania środowiska serwerowego, ponieważ przy intensywnych lub powtarzalnych obliczeniach stanowią one pomijalną składową stałą.
Problem zimnego startu jednak występuje, choć w~różnym stopniu. Wpływ na to mają procesy optymalizacji wykonania sekwencyjnego oraz czas inicjalizacji metod akceleracji \cite{je-benchmarking}.

Do pomiaru samego czasu wykonania można podejść na kilka sposobów. Dla krótkich fragmentów kodu istnieje ryzyko, że czas ich wykonania nie zmieści się w~precyzji pomiaru czasu. Rozwiązaniem tego problemu jest pomiar czasu wykonania $t$ danej liczby iteracji $n$, a~finalnym wynikiem pomiary będzie iloraz $\frac{t}{n}$. Innym podejściem jest wykonywanie testów w~pętli, aż do osiągnięcia zadeklarowanego całkowitego czasu. Hybrydowym rozwiązaniem jest dynamiczne dostosowanie liczby cykli pojedynczego pomiaru czasu, aby zaspokoić wymagania całkowitego czasu testu.

Funkcje i~pętle stosowane do sterowania cyklami, analizując krótkie czasy wykonania, mogą stanowić istotną składową stałą wyników. Aby temu zapobiec stosuje się dynamicznie tworzoną funkcję testową na podstawie serializowanej funkcji testowanej, co możliwe jest poprzez wywołania \lstinline{Function.prototype.toString()}. Wywołanie to zwraca funkcję - argumenty i~jej ciało w~postaci ciągu znaków, a~następnie ciało duplikowane jest wymaganą liczbę razy. Rozwiązanie to, w~przypadku bibliotek, które w~procesie budowania przeszły proces minifikacji może prowadzić do błędów związanych z~nazwami symboli. Dzieje się tak ponieważ w~procesie minifikacji nazwy symboli, w~celu zmniejszenia rozmiaru kodu, zamieniane są na ich krótsze odpowiedniki, często kolejne kombinacje liter alfabetu. Nie jest to możliwe dynamicznie w~przypadku funkcji zserializowanej do ciągu znaków.

Problemem wartym rozważenie są również mechanizmy pomiaru czasu. Różnią się one w~zależności od środowiska i~w celu zapewnienia bezpieczeństwa mają one często ograniczoną rozdzielczość, aby zapobiec atakom czasowym jak Spectre i~Meltdown. Standardowym rozwiązaniem kompatybilnym ze wszystkimi analizowanymi środowiskami jest Performance API z~metodą \lstinline{performance.now()}, która zwraca liczbę zmiennoprzecinkową reprezentującą czas w~milisekundach od załadowania strony. W~przeglądarce Mozilla Firefox wartość ta zaokrąglona jest to 1ms, a~w Google Chrome do 0.1ms. Google Chrome po użyciu flagi \mbox{\lstinline{--enable-benchmarking}} przy uruchomieniu udostępnia obiekt \lstinline{chrome.Interval}, którego pomiary mają rozdzielczość $1 {\mu}s$. Deno po użyciu flagi \lstinline{--allow-hrtime} zwiększa rozdzielczość pomiarów przy użyciu Performance API, a~NodeJS udostępnia obiekt \lstinline{process.hrtime}, którego pomiary wykonywane są z~rozdzielczością $1ns$.

Biblioteka benchmark.js łączy przedstawione wcześniej rozwiązania, dynamicznie wykrywa środowisko, dynamicznie dostosowuje liczbę iteracji pojedynczego pomiaru czasu zgodnie ze zdefiniowanym czasem minimalnym i~maksymalnym całego benchmarku oraz dynamicznie buduje funkcję testującą, która odpakowuje ciało funkcji testowanej, aby uniknąć narzutu związanego ze utworzeniem dodatkowej ramki na stosie, oraz aby móc wykorzystać zdefiniowane zmienne lokalne. Biblioteka ta jednak od 4 lat nie jest utrzymywana. Jako format modułu wykorzystuje format UMD, gdzie założeniem badań jest użycie wyłącznie modułów ECMAScript. Nie ma również możliwości zrezygnowania z~dynamicznego budowania funkcji testującej, co prowadzi do błędów związanych z~budowaniem środowiska testowego. Z~tych powodów zadecydowano o~własnej implementacji biblioteki do przeprowadzania benchmarków opartą o~moduły ECMAScript oraz kompatybilną z~badanymi środowiskami.

\section{Biblioteka benchmark}
\label{sec:benchmark}

Biblioteka \textit{benchmark}, której diagram klas przedstawiono na rysunku \ref{fig:benchmark-puml}, pozwala wykonywać testy w~dwóch wariantach wywołania - \textit{zwykłym} i~\textit{extracted}. Tryb \textit{zwykły} wykonuje funkcję testową dostarczoną bezpośrednio w~konstrukcji klasy \lstinline{Benchmark}. Tryb \textit{extracted} odpakowuje funkcję dostarczoną do obiektu \lstinline{BenchmarkExtracted} i~umieszcza jej ciało w~budowanej funkcji testującej. Na listingu \ref{lst:benchmark-fn} pokazano szablon takiej funkcji. Jako argumenty funkcja ta przyjmuje jeden obiekt kontekstu. W~jej ciele definiowane są zmienne liczby iteracji oraz klasa timer'a. Następnie w~pętli \lstinline{while} umieszczone zostaje ciało funkcji testowanej. Całą pętlę otaczają definiowalne funkcje \lstinline{setup} oraz \lstinline{teardown}, oraz rozpoczęcie i~zakończenie pomiaru czasu. Funkcja ta budowana jest dla każdej iteracji testu za każdym razem zastępując znaki \lstinline{@} innym numerem, co efektywnie zmienia nazwy zmiennych. Działanie to jest charakterystyczne dla trybu extracted i~pozwala na uniemożliwienie silnikowi optymalizacji kodu, co umożliwia za każdym razem uwzględnienie czasu zimnego startu.

\begin{figure}
  \centering
  \includegraphics[width=\linewidth]{diagrams/out/benchmark.png}
  \caption{Diagram klas biblioteki Benchmark.}
  \label{fig:benchmark-puml}
\end{figure}
\clearpage


\begin{lstlisting}[language=JavaScript, label=lst:benchmark-fn, caption=Kompozycja funkcji w~trybie \textit{extracted}.]
const template = `
  return (
    ${async ? "async " : ""}function(@context) {
      let @n = @context.config.microRuns;
      let @t = new @context.timer()
      ${setupString}
      @t.start();
      while(@n--){
          ${fnString}
      }
      let @tt = @t.stop();
      ${teardownString}
      return @tt;
    }
  )`.replace(/@/g, config.name + (this.nameCounter++).toString());
\end{lstlisting}

Oprócz podziału na dwie klasy, które reprezentują obydwa tryby, każda z~nich może wykonać testy będąc w~wariancie \textit{Iterations}, ograniczonym liczbą iteracji. W~wariancie \textit{TimeIterations} test ograniczony jest czasem minimalnym i~maksymalnym, a~liczba iteracji dobierana jest dynamicznie jedynie na podstawie minimalnej ich liczby, aby spełnić warunek minimalnego czasu testu. Na diagramie klas (rys. \ref{fig:benchmark-puml}) przedstawiony interfejs \lstinline{IBenchmark} definiuje metody klas biblioteki dla dwóch trybów, dodatkowo w~wariantach synchronicznym i~asynchronicznym.

W trybie \textit{zwykłym} istnieje możliwość wyeliminowania problemu zimnego startu. Pierwsze wykonania, które cechować się mogą dłuższym niż reszta czasem są pomijane. Kryterium decydującym o~rozpoczęciu właściwego zbierania próbek jest próg współczynnika wariancji dla czasów próbek z~okna o~konfigurowalnej długości, który zdefiniowany jest wzorem (\ref{eq:cov}) \cite{georges2007statistically}.

\begin{align}
 c_v &= \frac{\sigma}{\mu} \approx \widehat{c_v} = \frac{s}{\overline{x}} \label{eq:cov}
\end{align}
\begin{eqexpl}[1cm]
  \item{$c_v, \widehat{c_v}$} współczynnik wariancji i~jego estymator;
  \item{$\sigma, s$} odchylenie standardowe z~populacji i~próby;
  \item{$\mu, \overline{x}$} średnia arytmetyczna z~populacji i~próby.
\end{eqexpl}
\vspace{0.5cm}
Na listingu \ref{lst:benchmark-example} przedstawiono przykładowe wywołanie benchmarku. Możliwe do ustawienia parametry wraz z~ich opisem pokazane zostały w~tabeli \ref{tab:bench-params}.

\begin{lstlisting}[language=JavaScript, label=lst:benchmark-example, caption=Przykładowy benchmark mierzący czas wykonania funkcji \lstinline{Function.prototype()} 2000 razy.]
const ben = new Benchmark(function () {
  for (let index = 0; index < 2000; index++) Function.prototype();
});
const results = ben.runTimeIterations({
  minTime: 200,
  minSamples: 30,
  microRuns: 20,
  steadyState: true,
  cov: 0.01,
  covWindow: 10,
});
\end{lstlisting}


Ze względu na to, że problem zimnego startu stanowi pomijalną część algorytmów intensywnych obliczeniowo, które dodatkowo, jak to ma miejsce w~przypadku przetwarzania obrazów, często są wykonywane wielokrotnie w~ramach tej samej instancji aplikacji, przeprowadzone badania w~pierwszej części nie uwzględniają tego czynnika oraz dodatkowo go eliminują wykorzystując wyżej opisany mechanizm. 


\begin{wraptable}{r}{7.5cm}
    \vspace{-0.75cm}
    \caption{Versions of environments.}
    \label{tab:versions}
    \setlength{\tabcolsep}{0.5em}
    \begin{tabular}{lllll}%
        \hline
        Env     & Version      & Engine version           \\
        \hline
        Chrome  & 97.0.4692.71 & V8 9.7.106.18            \\
        Firefox & 96.0         & SpiderMonkey  96.0       \\
        Node    & 16.13.2      & V8    9.4.146.24-node.14 \\
        Deno    & 1.18.0       & V8      9.8.177.6        \\
        \hline
    \end{tabular}
    \vspace{-0.5cm}
\end{wraptable}

Badania wykonano tylko w~trybie \textit{zwykłym} w~wariancie \textit{TimeIterations}. Zaimplementowano i~zbadano czasy wykonania algorytmów transformacji Hough'a w~wariancie SHT z~zastosowaniem tablic LUT dla funkcji trygonometrycznych oraz bez. Takie same badana wykonano również dla wariantu CHT. Dla każdej z~metod akceleracji zbadano również czas zimnego startu, który różni się pomiędzy metodami akceleracji. W~tabeli \ref{tab:bench-params} opisano parametry benchmarków biblioteki \textit{benchmark}, z~jakimi wykonano testy, a~w tabeli \ref{tab:versions} opisano dokładne wersje środowisk wraz z~wersjami ich silników JavaScript, na których przeprowadzono testy.

\begin{table}
    \caption{Opis parametrów uruchomienia benchmarku w bibliotece \textit{benchmark}.}
    \centering
    \renewcommand\arraystretch{1.1}
    \begin{tabularx}{\linewidth}[t]{l>{\raggedleft\arraybackslash}p{1.9cm} X}
        \bfseries{Parametr} & \bfseries{Wartość dla testów} & \bfseries{Opis} \\ \hline
        \lstinline{GeneralConfig.name} & --- & Nazwa identyfikująca benchmark. \\ 
        \lstinline{GeneralConfig.microRuns} & 1 & Liczba iteracji $n$ pojedynczej próbki, której rzeczywisty czas $t_r$ kalkulowany jest wg. wzoru \mbox{$t_r = \frac{n}{t_c}$} na podstawie czasu całkowitego $t_c$. Wartość jest dostosowywana dynamicznie w wariancie \textit{TimeIterations} i~dotyczy wtedy pierwszej próbki.\\ \hline
        %
        \lstinline{StartConfig.steadyState} & \lstinline{true} & Czy pomijać pierwsze wykonania na podstawie progowania CoV. \\
        \lstinline{StartConfig.cov} & 0.01 & Próg stabilności czasu wykonania. \\
        \lstinline{StartConfig.covWindow} & 5 & Szerokość okna CoV. \\ \hline
        \lstinline{IterationsConfig.samples} & 0.01 & Liczba iteracji w wariancie \textit{Iterations}. \\ \hline
        \lstinline{TimeConfig.minSample} & 50 & Minimalna zebrana liczba próbek w wariancie \textit{TimeIterations}. \\
        \lstinline{TimeConfig.minTime} & 1000 & Minimalny czas benchmarku w wariancie \textit{TimeIterations} w milisekundach. \\
        \lstinline{TimeConfig.maxTime} & 30000 & Maksymalny czas benchmarku w wariancie \textit{TimeIterations} w milisekundach. \\  \hline


    \end{tabularx}
    \label{tab:bench-params}
\end{table}


\section{Badane aspekty}

Jednym z~analizowanych aspektów jest czas wykonania algorytmu dla stałego rozmiaru problemu. W~przypadku transformacji Hough'a rozmiarem problemu jest liczba jasnych pikseli obrazu, ale także częstotliwość próbkowania akumulatora, która odpowiada za dokładność detekcji. Czas wykonania dla tego samego rozmiaru problemu został porównany pomiędzy środowiskami i~metodami akceleracji.

Kolejnym aspektem jest zmiana czasu wykonania algorytmu w~zależności od rozmiaru problemu, również w~odniesieniu do metod akceleracji i~środowiska. Zmiana rozmiaru problemu została dobrana tak, że czas wykonania powinien rosnąć liniowo. Pozwoli to zaobserwować wszelkie optymalizacje i~odchylenia od liniowego wzrostu czasu wykonania. Rozmiarem problemu dla algorytmu w~wariancie SHT jest próbkowanie kąta obrotu wokół środka układu współrzędnych $S_\theta$, którego wartość mówi ile pikseli wymiaru akumulatora przypada na jeden stopień. Dla wariantu CHT rozmiarem problemu jest zakres długości wykrywanych promieni, co skutkuje koniecznością rysowania dłuższych linii wzdłuż kierunku gradientu.

Ostatnim z~badanych aspektów jest zjawisko zimnego startu i~jego wpływ na pierwsze wykonania algorytmu. Pierwszy wykonania, które w~poprzednich przypadkach są pomijane, aż do ustabilizowania się ich czasów, są będą przeanalizowane. 

Testy wykonane zostały na maszynie wyposażoną w~procesor Intel\textsuperscript{\tiny\textregistered} Core\textsuperscript{\tiny\texttrademark} i7-12700KF i~układ graficzny Nvidia 970 GTX w~systemie operacyjnym Ubuntu 20.04.1. Procesor na potrzeby testów miał wyłączone skalowanie częstotliwości oraz ze względu na swoją hybrydową architekturę tylko 4 rdzenie typu P-Core były włączone dla procesów testowych, co zostało osiągnięte przez wykorzystanie narzędzia \lstinline{taskset}. Testy wykonania sekwencyjnego zostały wykonane przy pomocy biblioteki \textit{google/benchmark} \cite{google-benchmark}.

\chapter{Implementacja algorytmów}
\label{sec:implementation}

W rozdziale tym opisana została implementa oraz wyniki pomiarów algorytmów dla przewidzianych środowisk oraz metod akceleracji (tabela \ref{tab:implemented}). Na początku opisana została organizacja plików projektu oraz sposób budowania bibliotek. Następnie omówione zostały wyniki detekcji z~wykorzystaniem transformacji Hough'a, które powinny być spójne dla wszystkich implementowanych metod akceleracji. Wszelkie problemy i~różnice wyników pomiędzy implementacjami poszczególnych metod opisane zostaną w~rozdziale \ref{sec:methods-results}. 

%oraz wyniki testów wydajności relatywnie do wykonania wariantu sekwencyjnego. 

%Na końcu omówiono przykład rzeczywistego użycia i~porównano wszystkie wyniki pomiarów.

\section{Organizacja plików}

Jednym z~założeń tej pracy jest analiza metod budowania bibliotek, które wykorzystują badane metody akceleracji obliczeń. Projekt, w~którym znajdują się zaimplementowane metody, został zbudowany w~oparciu o~strategię \textit{monorepo}, która zakłada, w~ramach jednego repozytorium, współistnienie wielu projektów ze wzajemnie zdefiniowanymi relacjami \cite{monorepo}. W~implementacji tej metody zostało wykorzystane narzędzie \textit{Lerna} \cite{lerna}. Zarządza ono projektami w~abstrakcji pakietów, pozwala instalować w~nich zewnętrzne zależności, wykonywać komendy w~zdefiniowanym zakresie oraz tworzyć relację między pakietami. Każdy pakiet posiada swój katalog \lstinline{node_modules}, w~którym zdefiniowane są zależności. \textit{Lerna}, tworząc relację między pakietami, tworzyw nim dowiązanie symboliczne do katalogu powiązanego pakietu, co pozwala innym narzędziom traktować go, jakby był zależnością instalowaną z~zewnątrz. \textit{Lerna}, jako prekursor omawianej strategii organizacji projektu, jest już wypierana przez konkurencyjne narzędzie, które rozwiązują problemy związana z~zarządzeniem zależnościami i~dostępem, czy dodają możliwości rozproszonego wykonania i~obsługę pamięci podręcznej dla wyników komend. Przykładami takowych są Nx, Rush, Turborepo, czy Bazel.

\begin{figure}[ht]
    \centering
    \includegraphics[width=\linewidth]{diagrams/out/files.png}
    \caption{Układ najważniejszych plików w~repozytorium projektu z~paczkami implementującymi metody akceleracji zaznaczonymi na czerwono.}
    \label{fig:files}
\end{figure}

Na rysunku \ref{fig:files} przedstawiono drzewo, reprezentujące układ najważniejszych z~perspektywy organizacji repozytorium plików. Na czerwono zaznaczone zostały paczki, które implementują metody akceleracji. Oprócz nich w~paczce \textit{test-simple} zaimplementowano prostą stronę internetową, która importuje zbudowane biblioteki i~wyświetla wyniki przetwarzania dla testowego obrazu w~celu sprawdzenia poprawności zaimplementowanych algorytmów. Paczka \textit{frontend} implementuje stronę internetową, która symuluje realistyczny scenariusz przetwarzania obrazu w~przeglądarce internetowej. Paczka \textit{meta} zawiera definicje interfejsów w~języku TypeScript, z~których korzystają wszystkie zaimplementowane metody akceleracji. Daje to pewność spójności implementacji i~możliwości porównania koniecznych działań dla metod i~środowisk, aby taką spójność osiągnąć. Zależności pomiędzy paczkami zaprezentowana została na rysunku \ref{fig:deps}

\begin{figure}[ht]
    \centering
    \includegraphics[width=\linewidth]{diagrams/out/deps.png}
    \caption{Zależności pomiędzy paczkami w~projekcie.}
    \label{fig:deps}
\end{figure}

\section{Budowanie bibliotek}

Głównym wymaganiem stawianym przed biblioteką jest bycie kompatybilną z~możliwie wieloma środowiskami, bycie eksportowaną w~formacie modułu ECMAScript oraz zapewniającą wygodę użytkowania, na którą zalicza się eksportowanie typów języka TypeScript. W~głównej mierze format modułu zapewnia jego kompatybilność, ponieważ wszystkie badane środowiska obsługują moduły ECMAScript. W~tej sekcji, na przykładzie biblioteki \textit{benchmark} omówiony zostanie proces jej budowania. Pozostałe biblioteki zawierające metody akceleracji nie różnią się u~podstaw w~procesie budowania, a~wszelkie różnice specyficzne dla metod akceleracji zostaną opisane wraz z~opisem ich implementacji.

Biblioteki budowane są w~wykorzystaniem narzędzia Webpack w~wersji piątej. Transformuje ono pliki wejściowe, analizując i~również transformując ich zależności. W~zależności od konfiguracji i~użytych \textit{loader'ów}, które stanowią ogniwa łańcucha transformacji, mogą obsługiwać wiele formatów plików i~produkować wyjścia w~wybranych konfiguracjach. Mogą generować pojedynczy plik z~kodem aplikacji lub podzielony na części.

\begin{lstlisting}[language=JavaScript, caption=Konfiguracja narzędzia Webpack służąca do budowania biblioteki \textit{benchmark},label=lst:webpack-config]
const config = {
  entry: "./src/main.ts",
  devtool: "cheap-module-source-map",
  output: {
    path: resolve(__dirname, "dist"),
    library: { type: "module" },
    environment: { module: true },
  },
  plugins: [
    new ForkTsCheckerWebpackPlugin({
      typescript: { build: true, mode: "write-dts" },
    }),
  ],
  module: {
    rules: [
      {
        test: /\.(ts|tsx)$/i,
        loader: "ts-loader",
        exclude: ["/node_modules/"],
        options: {
          transpileOnly: true,
          configFile: resolve(__dirname, "./tsconfig.build.json"),
        },
      },
    ],
  },
  // ...
  experiments: { outputModule: true },
  externalsType: "module",
  optimization: { minimize: false },
};
\end{lstlisting}

Na listingu \ref{lst:webpack-config} przedstawione zostały najważniejsze części obiektu, który stanowi konfigurację procesu budowania biblioteki \textit{benchmark}. Pole \lstinline{entry} wskazuje na pojedynczy plik wejściowy, który importuje i~eksportuje to, co ma stanowić wyjście budowanego modułu. Dla pola \lstinline{devtool}, które określa sposób mapowania kody wygenerowanego na kod źródłowy dla narzędzi developerskich, ustawiono wartość \lstinline{"cheap-module-source-map"}. Wykorzystanie domyślnej wartości \lstinline{"eval"} nie jest możliwe, ponieważ definicje map kodu biblioteki budowana w~trybie \textit{development} oraz importowane przez inną bibliotekę interferują z~procesem jej budowania generując błędy.
W polu \lstinline{output} zdefiniowano miejsce zapisu plików wynikowych, typ generowanej biblioteki oraz właściwości środowiska, gdzie obsługa modułów ECMAScript jest możliwa. Webpack 5 obsługę modułów dostarcza jako opcję eksperymentalną, która musi być jawnie zadeklarowana w~polu \lstinline{experiments.outputModule}. Pole \lstinline{externalsType: "module"} definiuje format importu zewnętrznych zależności w~generowanym kodzie. Plugin \lstinline{ForkTsCheckerWebpackPlugin} odpowiedzialny jest za, równolegle z~procesem budowania kodu JavaScript, generowanie plików \lstinline{*.d.ts} zawierające definicję typów języka TypeScript. Sprawdza on również statycznie poprawność kodu i~informuje o~ewentualnych błędach. Jako, że pliki wynikowe biblioteki \textit{benchmark} stanowią formę pośrednią, która będzie stanowić wejście innych procesów budowania, proces minifikacji kodu nie jest wymagany, co definiuje pole \lstinline{optimization.minimise}.

Najważniejszym elementem tej konfiguracji jest pole \lstinline{module.rules}, które definiuje proces transformacji plików. Skonfigurowano tutaj \textit{ts-loader}, który odpowiedzialny jest za transpilację plików w~języku TypeScript do plików języka JavaScript, które obsługiwane są przez przeglądarkę internetową i~NodeJS. Pomimo tego, że Deno natywnie obsługuje język TypeScript nic nie stoi na przeszkodzie, aby podczas przeprowadzania testów posłużyć się modułami w~języku JavaScript, będące wynikiem budowania.

\section{Implementacja algorytmów}

Opis algorytmów należy zacząć od definicji interfejsu funkcji, który opisuje format danych wejściowych i~wyjściowych. W~definicjach zawarte są interfejsy dla dwóch wariantów algorytmów transformacji Hough'a - SHT i~CHT. 

\begin{figure}[h]
  \centering
  \includegraphics[width=\linewidth]{diagrams/out/meta.png}
  \caption{Definicja interfejsu algorytmów.}
  \label{fig:meta}
\end{figure}

Na rysunku \ref{fig:meta} znajduje się diagram klas reprezentujący definicję typów wykorzystywanych w~implementacji algorytmów dla wszystkich metod akceleracji. Interfejsy posegregowano zgodnie z~wariantami transformacji, używając typów generycznych, oraz zgodnie z~wymaganiami metod akceleracji wydzielając opcje dla metod ze zrównolegleniem. Na listingu \ref{lst:meta} pokazano definicję typów funkcji generycznych z~podziałem na te synchroniczne (\lstinline{HT}) i~asynchroniczne (\lstinline{HTAsync}), na podstawie których zdefiniowane zostały aliasy funkcji dla wariantów transformacji, metod akceleracji oraz współbieżności. Funkcja zawsze przyjmuje jako argumenty binarny obraz w~zmiennej typu \lstinline{Uint8Array} oraz obiekt opcji, który rozszerza interfejs \lstinline{HTOptions} w~zależności od typu algorytmu i~metody akceleracji. Opcja \lstinline{returnHSpace} określa czy razem z~wynikami zwracać akumulator i~na potrzeby testów jest wyłączona. Zwracanie akumulatora razem z~wynikami przynieść może spadek wydajności jeśli będzie stosowane w~interfejsach wymuszających kopiowanie danych. Od wariantu transformacji zależy również zwracany typ rozszerzający interfejs \lstinline{HTResult}.


\begin{lstlisting}[language=JavaScript, float=ht, caption=Definicja typów funkcji wariantu SHT i~CHT,label=lst:meta]
export type HT<O extends HTOptions, HTResult> = (
  binaryImage: Uint8Array,
  options: O
) => HTResults<HTResult>;

export type HTAsync<O extends HTOptions, HTResult> = (
  binaryImage: Uint8Array,
  options: O
) => Promise<HTResults<HTResult>>;


export type SHTResults = HTResults<SHTResult>;
export type SHT = HT<SHTOptions, SHTResult>;
export type SHTAsync = HTAsync<SHTOptions, SHTResult>;
export type SHTParallelAsync = HTAsync<SHTParallelOptions, SHTResult>;

export type CHTResults = HTResults<CHTResult>;
export type CHT = HT<CHTOptions, CHTResult>;
export type CHTAsync = HTAsync<CHTOptions, CHTResult>;
export type CHTParallelAsync = HTAsync<CHTParallelOptions, CHTResult>;
\end{lstlisting}

\subsection{Wyniki działania algorytmów}

W tej sekcji przedstawiono obrazy będące wejściem, wyjściem i~wizualizacją dwuwymiarowego akumulatora, która została wygenerowana poprzez odwzorowanie zakresu wartości komórek na zakres wartości $\lbrack 0, 255\rbrack$.

\subsubsection{Standard Hough Transform}

Na rysunku \ref{fig:sht} przedstawiono obraz wejściowy, który stanowił podstawę weryfikacji poprawności, (rys. \ref{fig:sht:input}) oraz wynik transformacji z~naniesionymi wykrytymi liniami (rys. \ref{fig:sht:output}). W~odróżnieniu od typowych obrazów wejściowych, ten został poddany operacji progowania, a~nie wykrywania krawędzi. Pozwoliło to zwiększyć liczbę jasnych pikseli obrazu, a~tym samym rozmiar problemu. Próbkowanie akumulatora dla odległości i~kąta obrotu wynosi kolejno $S_\rho = 1$ i~$S_\theta = 1$, co na podstawie szerokości $w = 419$ i~$h = 423$ obrazu wejściowego definiuje rozmiar akumulatora jako $w_{acc} = 360$ i~$h_{acc} = \lfloor\sqrt{w^2+h^2}\rfloor = 595$.

\begin{figure}
  \centering

  \begin{minipage}[c]{.475\linewidth}
    \subcaptionbox{Binarny obraz wejściowy\label{fig:sht:input}}
      {
        \includegraphics[width=\textwidth]{img/sht/input.jpeg}
      }
      \subcaptionbox{Wynik detekcji\label{fig:sht:output}}
      {
        \includegraphics[width=\textwidth,trim=0 0 0 -1cm ]{img/sht/output.png}
      }%
      \end{minipage}%
  \hfill
  \begin{minipage}[c]{.475\linewidth}
    \subcaptionbox{Akumulator po głosowaniu\label{fig:sht:acc}}
      {
        \includegraphics[width=\linewidth]{img/sht/equantial_acc.png}
      }%
  \end{minipage}%
  \caption{Wynik działania zaimplementowanego algorytmu SHT w~wariancie sekwencyjnym.}
  \label{fig:sht}
\end{figure}

\subsubsection{Circle Hough Transform}

Obraz testowy dla wariantu CHT przedstawiono na obrazie \ref{fig:cht:input}, a~wynik detekcji na \ref{fig:cht:output}. Poprzez rysowanie prostopadłych do kierunku krawędzi linii o~zakresie długości odpowiadającym długości wyszukiwanych promieni, w~procesie głosowania wygenerowany został akumulator przedstawiony na rysunku \ref{fig:cht:acc}. Jego rozmiar odpowiada rozmiarowi obrazu wejściowego, a~najjaśniejsze punkty są kandydatami na środki okręgów. Oprócz akumulatora dwuwymiarowego do detekcji promienia CHT wykorzystuje nieprzedstawiony tutaj akumulator jednowymiarowy, który bierze udział w~głosowaniu nad najlepszym promieniem. Dzieje się to osobno dla każdego kandydata na środek okręgu.

\begin{figure}
    \centering
    \begin{subfigure}{.475\linewidth}
          \includegraphics[width=\textwidth]{img/cht/input.png}
          \caption{Binarny obraz wejściowy}\label{fig:cht:input}
    \end{subfigure}%
    \hfill
    \begin{subfigure}{.475\linewidth}
      \includegraphics[width=\textwidth]{img/cht/output.png}
      \caption{Wynik detekcji}\label{fig:cht:output}
    \end{subfigure}% 
    \bigskip
    
    \begin{subfigure}{.475\linewidth}
      \includegraphics[width=\textwidth]{img/cht/sequential_acc.png}
      \caption{Akumulator po głosowaniu na środek okręgu}\label{fig:cht:acc}
    \end{subfigure}%
    \caption{Wynik działania zaimplementowanego algorytmu CHT w~wariancie sekwencyjnym.}
    \label{fig:cht}
  \end{figure}
  
\chapter{Implementacja metod akceleracji i~rezultaty pomiarów}
\label{sec:methods-results}

W rozdziale tym opisane zostały sposoby implementacji badanych metod akceleracji obliczeń oraz zaprezentowany zostały rezultaty przeprowadzonych testów wydajności. Wskazane zostały również na wszelkie aspekty, na które trzeba zwrócić uwagę w~procesie budowania bibliotek z~daną metodę akceleracji oraz ich możliwości i~ograniczenia. Wszystkie wyniki pomiarów porównane zostały do wykonania sekwencyjnego w~C++, a~dla pozostałych metod akceleracji na wykresach kolorem szarym zaznaczony został zakres czasów wykonania sekwencyjnego wszystkich środowisk języka JavaScript.

Jak już wcześniej wspomniano w~rozdziale \ref{sec:benchmark}, dla każdej z~metod akceleracji zaimplementowano trzy algorytmy. Niżej przedstawiono nazwy i~oznaczenia używane na potrzeby implementacji i~prezentacji wyników.

\begin{itemize}
    \item Standard Hough Transform (SHT \textit{non-LUT}, \lstinline{SHT_Simple}) -- algorytm detekcji linii wykorzystujący intensywne obliczenia z~użyciem funkcji trygonometrycznych.
    \item Standard Hough Transform (SHT \textit{LUT}, \lstinline{SHT_Simple_Lookup}) -- algorytm detekcji linii wykorzystujący tablicę LUT do zapisanie potrzebnych wartości funkcji trygonometrycznych.
    \item Circle Hough Transform (CHT, \lstinline{CHT_Simple}) -- algorytm detekcji okręgów wykorzystujący metodę gradientu w~celu redukcji złożoności obliczeniowej (rozmiaru akumulatora). Zmienny parametr maksymalnego promienia obliczany jest ze wzoru $r_{max} = 20+10n$, gdzie $n$ to współczynnik inkrementowany w~kolejnych iteracjach pomiarów.
\end{itemize}


\section{Wykonanie sekwencyjne}

Implementacja procesu budowania biblioteki z~algorytmem transformacji w~wariancie wykonania sekwencyjnego nie wymagała zmian w~procesie budowania w~porównaniu do biblioteki \textit{benchmark}, będącej punktem odniesienia. Na rysunku \ref{plot:sequential} pokazano rezultaty pomiarów czasu wykonania w~postaci wykresów dla każdego z~implementowanych algorytmów.

\subsection{Wyniki pomiarów}

We wszystkich wariantach transformacji implementacja w~C++ osiągnęła najlepsze czasy wykonania, co jest oczekiwanym rezultatem. Dla algorytmu SHT Najlepiej zoptymalizowanym okazało się środowisko przeglądarki Google Chrome będące \tms{3.71} wolniejsze od implementacji w~C++ dla wariantu \textit{non-LUT} i~$S_\theta = 1$. Dla wszystkich algorytmów środowiska serwerowe NodeJS oraz Deno osiągnęło porównywalne wyniki z~minimalną przewagą środowiska NodeJS, która była powtarzalna pomiędzy wieloma uruchomieniami testów, jednak jest pomijalnie mała. Najgorzej zoptymalizowana okazuje się przeglądarka Mozilla Firefox, będąc \tms{1.71} wolniejszą od Google Chrome dla $S_\theta = 1$. Interesującym dla niej zjawiskiem jest optymalizacja zachodząca dla SHT \textit{LUT} i~$S_\theta \geq 5$. 

Analizując różnice pomiędzy wykonaniami wariantów \textit{non-LUT} i~\textit{LUT} widać, że wszystkie środowiska zyskują na optymalizacji związanej z~używaniem zmiennych do przechowywania wartości funkcji trygonometrycznych, ponieważ optymalizacja ta stanowi jedyną różnicę w~implementacji. Jednak przeglądarka Mozilla Firefox zdecydowanie gorzej radzi sobie z~wykorzystaniem tej optymalizacji, co prowadzi do zwiększenia przewagi NodeJS i~Deno z~\tms{1.05} do \tms{1.99} większego czasu wykonania dla $S_\theta = 1$.

Dla algorytmu CHT środowiska NodeJS i~Deno, jak i~przeglądarka Google Chrome dają porównywalne rezultaty. Również i~tutaj przeglądarka Mozilla Firefox okazała się być wolniejsza od pozostałych środowisk wykonując ten sam algorytm na tych samych danych.

Na rysunku \ref{fig:profiler-seq} widać wynik profilowania wykonania sekwencyjnego algorytmu CHT dla $n=1$. Biblioteka \textit{benchmark} w~wewnętrznej metodzie \lstinline{waitForSteadyState} czeka, aż czas wykonania kodu się ustabilizuje. Pozwala to silnikowi JavaScript ustabilizować czas wykonania co w~pokazanych wynikach profilowania dzieje się w~bloku \textit{Optimize Code}, które występują tylko na początku pomiarów.

\begin{figure}[h]
    \centering
    \includegraphics[width=\linewidth]{img/seq-profiler.png}
    \caption{Wynik profilowania wykonania sekwencyjnego algorytmu CHT w~przeglądarce Google Chrome.}
    \label{fig:profiler-seq}
\end{figure}

Zbadano również różnicę pomiędzy wartościami akumulatorów, która pomiędzy wariantami \textit{non-LUT} i~\textit{LUT} powinna być równa zero. Wykryto jednak różnicę dla jednego piksela, która widoczna jest na rysunku \ref{fig:diff:seq_lut}. Pochodzić ona może z~różnicy w~precyzji operacji zmiennoprzecinkowych. Tablica LUT funkcji trygonometrycznych zbudowana została z~wykorzystaniem pojedynczej precyzji i~obiektu tablicy typu \lstinline{Float32Array}. JavaScript wewnętrznie do reprezentacji liczb zmiennoprzecinkowych używa podwójnej precyzji.



\begin{figure}[ht]
    \groupBenchmark{
        \plotBenchmark{cpp_theta_SHT_Simple.csv}{cppColor}{{}}
        \addlegendentry{C++}

        \plotBenchmark{js-sequential_theta_SHT_Simple_node.csv}{nodeColor}{{}}
        \addlegendentry{Node}

        \plotBenchmark{js-sequential_theta_SHT_Simple_deno.csv}{denoColor}{{}}
        \addlegendentry{Deno}

        \plotBenchmark{js-sequential_theta_SHT_Simple_Firefox.csv}{firefoxColor}{{}}
        \addlegendentry{Firefox}

        \plotBenchmark{js-sequential_theta_SHT_Simple_Chrome.csv}{chromeColor}{{}}
        \addlegendentry{Chrome}

    } {
        \plotBenchmark{cpp_theta_SHT_Simple_Lookup.csv}{cppColor}{{}}
        \addlegendentry{C++}

        \plotBenchmark{js-sequential_theta_SHT_Simple_Lookup_node.csv}{nodeColor}{{}}
        \addlegendentry{Node}

        \plotBenchmark{js-sequential_theta_SHT_Simple_Lookup_deno.csv}{denoColor}{{}}
        \addlegendentry{Deno}

        \plotBenchmark{js-sequential_theta_SHT_Simple_Lookup_Firefox.csv}{firefoxColor}{{}}
        \addlegendentry{Firefox}

        \plotBenchmark{js-sequential_theta_SHT_Simple_Lookup_Chrome.csv}{chromeColor}{{}}
        \addlegendentry{Chrome}
    } {
        \plotBenchmark{cpp_theta_CHT_Simple.csv}{cppColor}{{}}
        \addlegendentry{C++}

        \plotBenchmark{js-sequential_theta_CHT_Simple_node.csv}{nodeColor}{{}}
        \addlegendentry{Node}

        \plotBenchmark{js-sequential_theta_CHT_Simple_deno.csv}{denoColor}{{}}
        \addlegendentry{Deno}

        \plotBenchmark{js-sequential_theta_CHT_Simple_Firefox.csv}{firefoxColor}{{}}
        \addlegendentry{Firefox}

        \plotBenchmark{js-sequential_theta_CHT_Simple_Chrome.csv}{chromeColor}{{}}
        \addlegendentry{Chrome}
    }
    [3400][850][14]
    \caption{Wyniki pomiarów czasu wydajności dla wykonania sekwencyjnego SHT i CHT.}
    \label{plot:sequential}
\end{figure}


\begin{figure}[h]
    \begin{subfigure}{0.3\textwidth}
        \includegraphics[width=\linewidth] {../../packages/js-benchmarks/img/diff_seq_seq_lookup.png}
        \caption{SHT Sekwencyjny \textit{LUT}}\label{fig:diff:seq_lut}
    \end{subfigure}\hfill
    \begin{subfigure}{0.3\textwidth}
        \includegraphics[width=\linewidth] {../../packages/js-benchmarks/img/diff_seq_wasm.png}
        \caption{SHT WASM}\label{fig:diff:wasm}
    \end{subfigure}\hfill
    \begin{subfigure}{0.3\textwidth}
        \includegraphics[width=\linewidth] {../../packages/js-benchmarks/img/diff_seq_gpu.png}
        \caption{SHT WebGL}\label{fig:diff:gpu}
    \end{subfigure}
    \caption{Znormalizowana do przedziału $\lbrack 0, 255\rbrack$ absolutna różnica wartości akumulatorów z~wynikiem głosowania pomiędzy wykonaniem sekwencyjnym SHT \textit{non-LUT}.}\label{fig:diff}
\end{figure}



\section{NodeJS Native C++ Addon}

Implementacja natywnego modułu w~środowisko NodeJS wymaga utworzenia, oprócz właściwego kodu w~języku C++, warstwy abstrakcji, która zapewnia obsługę interfejsu po stronie języka JavaScript, pozwala na określenie typów zmiennych, kopiowanie danych, czy wywoływanie funkcji. Warstwa ta zaimplementowana została za pomocą Node-API, która dostarcza niezbędne interfejsy i~funkcje w~języku C++, aby efektywnie powiązać kod C++ i~JavaScript~\cite{napi}. Implementacja natywnych modułów korzysta dokładnie z~tego samego kodu, dla którego przeprowadzony pomiary w~natywnym C++. Kod ten został skompilowany do postaci bibliotek współdzielonych, które w~procesie linkowania jest dołączany do natywnego modułu NodeJS. 

Na listingu \ref{fig:nodejs} przedstawiony został plik z~paczki \textit{node-cpp-sequential}, który dołącza nagłówki z~paczki \textit{cpp-sequential}. Funkcja \lstinline{Napi::Object Init(Napi::Env env, Napi::Object exports)} zwraca obiekt, który definiuje wartości eksportowane przez moduł. W~niej właśnie definiowane są funkcje takie jak \lstinline{SHTSimple}, której implementację stanowi \lstinline{SHTSimpleBind}. Funkcja ta odpowiedzialna jest za stworzenie obiektów z~danymi wejściowymi zgodnie z~interfejsem dołączanych bibliotek (\lstinline{getTestImage} oraz \lstinline{getSHTOptions}), wywołania właściwej funkcji oraz zbudowanie obiektu z~odpowiedzią (\lstinline{getSHTResultBind}).

\begin{lstlisting}[language=C++, float=h, caption=Plik powiązania kodu C++ z~JavaScript, label=lst:cpp-js]
#define NAPI_DISABLE_CPP_EXCEPTIONS
#include "CHTSimple.h"
#include "SHTSimple.h"
#include "SHTSimpleLookup.h"
#include "napi.h"
#include <cstdint>

using namespace Napi;
    
Napi::Object SHTSimpleBind(const Napi::CallbackInfo &info) {
    Napi::Env env = info.Env();
    auto testImageBind = info[0].As<Napi::Uint8Array>();
    auto testImage = getTestImage(testImageBind);
    auto optionsBind = info[1].As<Napi::Object>();
    SHTOptions options = getSHTOptions(optionsBind);
    SHTResults results = SHTSimple(testImage, options);
  
    return getSHTResultBind(env, options, results);
  }
// ...
Napi::Object Init(Napi::Env env, Napi::Object exports) {
    exports.Set(Napi::String::New(env, "SHTSimple"),
                Napi::Function::New(env, SHTSimpleBind));
    // ...
    return exports;
}
  
NODE_API_MODULE(addon, Init)
\end{lstlisting}



\begin{figure}
    \groupBenchmark{
        \plotBenchmark{cpp-addon_theta_SHT_Simple_node.csv}{nodeColor}{{}}
        \addlegendentry{Node}

        \seqReference
    } {
        \plotBenchmark{cpp-addon_theta_SHT_Simple_Lookup_node.csv}{nodeColor}{{}}
        \addlegendentry{Node}

        \seqReferenceLookup
    }[1400][350]
    \label{plot:cpu-addon}
    \caption{Node C++ addon SHT execution benchmark results. Gray area shows sequential JavaScript execution performance range.}
\end{figure}


\subsection{Wyniki pomiarów}

Na rysunku \ref{plot:cpu-addon} widać, że wykorzystanie tej metody akceleracji w~każdym przypadku przyspieszyło działania algorytmu względem odpowiadającego mu wykonania sekwencyjnego w~środowisku NodeJS. Przyspieszenie nastąpiło również względem wszystkich innych środowisk poza przypadkiem wariantu SHT \textit{LUT}, gdzie wydajność osiągnęła poziom wydajności odpowiednika wykonania sekwencyjnego w~przeglądarce Google Chrome.

Wariant SHT \textit{LUT} był \tms{4.40} szybszy od SHT \textit{non-LUT}, co wraz z~wynikami algorytmu CHT pozwala wyciągnąć wniosek, że jeśli w~algorytmie nie znamy z~góry niezbędnych wartości funkcji trygonometrycznych lub obliczenia są oparte głównie na liczbach całkowitych, to wykorzystanie tej metody akceleracji przynosi poprawę wydajności algorytmów. Oczywiście nie można zapominać o~wadach takiego rozwiązania, jaką stanowi konieczność implementacji warstwy wiążącej obydwa języki. Istnieje również konieczność zapewnienia kompatybilności biblioteki z~różnymi środowiskami uruchomieniowymi, dla których biblioteka musi zostać zbudowana zawczasu, lub w~momencie instalacji.

\section{WebAssembly i~asm.js}

Budowanie biblioteki wykorzystujące WebAssembly, tak jak w~przypadku natywnych modułów NodeJS, wymaga wykorzystania specyficznych narzędzi. Do budowy biblioteki we wszystkich wariantach wykorzystany został Emscripten \cite{emscripten} - zestaw narzędzi, dzięki którym dowolny przenośny kod w~językach kompatybilnych z~backendem kompilatora LLVM może został skompilowany do WebAssembly. Możliwym stało się uruchamianie w~środowiskach webowych kodu języków takich jak Python czy Lua, poprzez kompilację interpreterów tych języków.

\subsection{Implementacja procesu budowania}

Na listingu \ref{lst:wasm-build} przedstawiono wywołanie komendy \lstinline{emcc}, które inicjuje proces kompilacji. Z~perspektywy modularności oraz wygody użytkowania niezbędne jest omówienie opcji dostarczonych do komendy. Flaga \lstinline{--bind} uruchamia narzędzie Embind i~dodaje do kodu wynikowego modułu powiązania pomiędzy funkcjami eksportowanymi przez moduł, a~tymi zdefiniowanymi w~kodzie C++. Na listingu \ref{lst:wasm-bind} pokazano część kodu definiującego warstwę powiązania obydwu języków. Aby zapewnić zgodność z~docelowym interfejsem funkcja \lstinline{SHTSimpleBind} eksportowana jest jako \lstinline{SHTSimple}. W~porównaniu do natywnych modułów NodeJS w~tym wypadku Embind sam zajmuje się przetwarzaniem obiektów wejściowych na podstawie dostarczonych odwzorowań (linijka 16) i~powiązane funkcje zamiast argumentu z~całym kontekstem, otrzymać mogą argumenty w~docelowych typach. Jednak również tutaj wszystkie dane muszą zostać skopiowane do liniowego modelu pamięci WebAssembly. Za przetwarzanie obrazu wejściowego do postaci wektora liczb w~formacie \lstinline{uint8_t} odpowiedzialna jest funkcja \lstinline{emscripten::convertJSArrayToNumberVector}, która wymusza traktowanie wartości jako liczby, wpływając pozytywnie na wydajność. Opcja \lstinline{ALLOW_MEMORY_GROWTH=1} umożliwia powiększenie obszaru pamięci, który domyślnie ma rozmiar 16.0MB, co może nie wystarczyć, aby pomieścić przestrzeń akumulatora dla większych wartości próbkowania. Aby wynik kompilacji funkcjonował jako biblioteka eksportująca moduł ES6, należy użyć flag \lstinline{--no-entry} oraz opcji \lstinline{EXPORT_ES6=1}, \lstinline{MODULARIZE} oraz \lstinline{SINGLE_FILE=1}, aby binarny kod WebAssembly był zagnieżdżony w~pliku \lstinline{*.mjs} w~postaci kodu \textit{base64}. Nie można również zapomnieć o~możliwościach optymalizacji samego kompilatora, czyli o~flagach \lstinline{-O3}, czy \lstinline{-ffast-math}, która wprowadza optymalizacje działań matematycznych rezygnując z~poprawności wyników na ostatnich bitach.

\begin{lstlisting}[language=bash, float=ht, label=lst:wasm-build, caption=Komenda wykorzystana podczas kompilacji kodu C++ do modułu WebAssembly., showstringspaces=false]
COMMON_ARGS="-Inode_modules/cpp-sequential/include 
    --bind \
    -s MODULARIZE \
    -s ALLOW_MEMORY_GROWTH=1 \
    -s FILESYSTEM=0 \
    -s SINGLE_FILE=1 \
    -s ENVIRONMENT=web \
    -s EXPORT_ES6=1 \
    --no-entry \
    -std=c++17\
    -ffast-math\
    -O3"

### Non-SIMD
NON_SIMD_ARGS="$COMMON_ARGS \
    src/wasm_sequential.cc \
    node_modules/cpp-sequential/src/SHTSimpleLookup.cpp \
    node_modules/cpp-sequential/src/SHTSimple.cpp \
    node_modules/cpp-sequential/src/CHTSimple.cpp"

emcc $(echo $NON_SIMD_ARGS -o build/wasmSequential.mjs)
\end{lstlisting}

\begin{lstlisting}[language=C++, float=ht, label=lst:wasm-bind, caption=Wybrane fragmenty kodu warstwy powiązania WASM pomiędzy C++ i~JavaScript.]
// ...
#include <emscripten.h>
#include <emscripten/bind.h>

extern "C" {
// ...
EMSCRIPTEN_KEEPALIVE
SHTResults SHTSimpleBind(emscripten::val binaryImageBind, SHTOptions options) {
  auto testImage =
      emscripten::convertJSArrayToNumberVector<uint8_t>(binaryImageBind);
  return SHTSimple(testImage, options);
}
// ...
}

EMSCRIPTEN_BINDINGS(wasm_sequential) {
  emscripten::register_vector<uint32_t>("VectorUint32");
  emscripten::register_vector<uint8_t>("VectorUint8");
  emscripten::value_object<SHTSamplingOptions>("SHTSamplingOptions")
      .field("rho", &SHTSamplingOptions::rho)
      .field("theta", &SHTSamplingOptions::theta);
  // ...
  emscripten::function("SHTSimple", &SHTSimpleBind);
  // ...
}
\end{lstlisting}

\subsection{Warianty testów}

Dla każdego wariantu algorytmu transformacji przewidziano pomiary wydajności dla każdej z~wymienionych niżej metod.

\begin{itemize}
    \item asm.js -- wydajny podzbiór języka JavaScript zoptymalizowany pod kątem możliwości optymalizacji silnika i~inferencji typów,
    \item WASM -- standardowy wynik kompilacji kodu C++
    \item WASM SIMD (implicite) -- wynik kompilacji kodu C++ z~włączonym procesem automatycznej wektoryzacji wykonania, w~celu wykorzystania instrukcji wektorowych SIMD,
    \item WASM SIMD (explicite) -- wynik kompilacji kodu C++ z~ręcznie przeprowadzonym procesem wektoryzacji wykonania.
\end{itemize}

Zbudowanie biblioteki w~asm.js wymaga dodania opcji \lstinline{WASM=0}. Wariant SIMD explicite i~implicite budowany jest z~dodatkową flagą \lstinline{-msimd128}. Wariant SIMD explicite jest ręcznie przystosowany do obsługi operacji na wektorowych rejestrach 128b. Przykładem takiego zastosowania jest pokazana na listingu \ref{lst:simd} funkcja obliczająca minimalne i~maksymalne współrzędne w~przestrzeni akumulatora za jednym razem dla osi OX i~OY. W~procesie implementacji wersji SIMD explicite wykorzystano operacje na czteroelementowych wektorach liczb całkowitych bądź zmiennoprzecinkowych dla każdej intensywnie obliczeniowo pętli redukując liczbę ich iteracji czterokrotnie.

\begin{lstlisting}[language=C++, float=ht, label=lst:simd, caption=Funkcja \lstinline{getBounds} algorytmu CHT z~wykorzystaniem instrukcji SIMD.]
#include <wasm_simd128.h>
/ ...
void getBounds(int32_t x, int32_t max, v128_t vRBounds, int32_t *out) {
  v128_t vZero = wasm_i32x4_const_splat(0);
  v128_t vMax = wasm_i32x4_splat(max);
  v128_t vX = wasm_i32x4_splat(x);
  vX = wasm_i32x4_add(vRBounds, vX);
  vX = wasm_i32x4_max(vX, vZero);
  vX = wasm_i32x4_min(vX, vMax);
  wasm_v128_store(out, vX);
}
\end{lstlisting}



\begin{figure}[p]
    \groupBenchmark{
        \plotBenchmark{js-asm_theta_SHT_Simple_node.csv}{nodeColor}{{}}
        \addlegendentry{Node}

        \plotBenchmark{js-asm_theta_SHT_Simple_Firefox.csv}{firefoxColor}{{}}
        \addlegendentry{Firefox}

        \plotBenchmark{js-asm_theta_SHT_Simple_Chrome.csv}{chromeColor}{{}}
        \addlegendentry{Chrome}

        \seqReference
    } {
        \plotBenchmark{js-asm_theta_SHT_Simple_Lookup_node.csv}{nodeColor}{{}}
        \addlegendentry{Node}

        \plotBenchmark{js-asm_theta_SHT_Simple_Lookup_Firefox.csv}{firefoxColor}{{}}
        \addlegendentry{Firefox}

        \plotBenchmark{js-asm_theta_SHT_Simple_Lookup_Chrome.csv}{chromeColor}{{}}
        \addlegendentry{Chrome}

        \seqReferenceLookup
    }[8500][2200]
    \caption{asm.js SHT execution benchmark results.}
    \label{plot:asm}
\end{figure}




\begin{figure}
    \groupBenchmark{
        \plotBenchmark{js-wasm_theta_SHT_Simple_node.csv}{nodeColor}{{}}
        \addlegendentry{Node}

        \plotBenchmark{js-wasm_theta_SHT_Simple_Firefox.csv}{firefoxColor}{{}}
        \addlegendentry{Firefox 95}

        \plotBenchmark{js-wasm_theta_SHT_Simple_Chrome.csv}{chromeColor}{{}}
        \addlegendentry{Chrome 97}

        \seqReference
    } {
        \plotBenchmark{js-wasm_theta_SHT_Simple_Lookup_node.csv}{nodeColor}{{}}
        \addlegendentry{Node}

        \plotBenchmark{js-wasm_theta_SHT_Simple_Lookup_Firefox.csv}{firefoxColor}{{}}
        \addlegendentry{Firefox 95}

        \plotBenchmark{js-wasm_theta_SHT_Simple_Lookup_Chrome.csv}{chromeColor}{{}}
        \addlegendentry{Chrome 97}

        \seqReferenceLookup
    }[2500][950]
    \label{plot:wasm}
    \caption{WASM SHT execution benchmark results. Gray area shows sequential JavaScript execution performance range.}
\end{figure}




\begin{figure}
    \groupBenchmark{
        \plotBenchmark{js-wasm_simd_explicit_theta_SHT_Simple_node.csv}{nodeColor}{{}}
        \addlegendentry{Node}

        \plotBenchmark{js-wasm_simd_explicit_theta_SHT_Simple_Firefox.csv}{firefoxColor}{{}}
        \addlegendentry{Firefox 95}

        \plotBenchmark{js-wasm_simd_explicit_theta_SHT_Simple_Chrome.csv}{chromeColor}{{}}
        \addlegendentry{Chrome 97}

        \seqReference
    } {
        \plotBenchmark{js-wasm_simd_explicit_theta_SHT_Simple_Lookup_node.csv}{nodeColor}{{}}
        \addlegendentry{Node}

        \plotBenchmark{js-wasm_simd_explicit_theta_SHT_Simple_Lookup_Firefox.csv}{firefoxColor}{{}}
        \addlegendentry{Firefox 95}

        \plotBenchmark{js-wasm_simd_explicit_theta_SHT_Simple_Lookup_Chrome.csv}{chromeColor}{{}}
        \addlegendentry{Chrome 97}

        \seqReferenceLookup
    }[2200][650]
    \label{plot:wasm_simd_explicit}
    \caption{WASM SIMD (explicit) SHT execution benchmark results. Gray area shows sequential JavaScript execution performance range.}
\end{figure}



\subsection{Wyniki pomiarów}

Analizując wykres przedstawiający wyniki pomiarów czasów wykonania algorytmów z~wykorzystaniem kompilacji do asm.js przedstawiony na rysunku \ref{plot:asm} zobaczyć można, że ta metoda akceleracji w~każdym przypadku daje gorsze rezultaty od wykonania sekwencyjnego. Może to mieć dwie potencjalne przyczyny. Pierwszą z~nich jest nieprzeprowadzanie kompilacji kodu przez środowiska, które z~asm.js nie są w~pełni kompatybilne, co w~przypadku tych konkretnych algorytmów powoduje utratę wydajności. W~zbudowanym module nie znajdują się funkcje trygonometryczne \lstinline{Math.sin} i~\lstinline{Math.cos}. Oznacza to, że w~kodzie znajduje się ich ręczna implementacja, która nie korzysta z~biblioteki standardowej. Drugim możliwym powodem, w~szczególności w~przypadku przeglądarki Mozilla Firefox, która z~asm.js jest w~pełni kompatybilna, może być problem z~wykrywaniem kodu asm.js, który jest umieszczony w~jednej paczce z~resztą kodu w~procesie budowania. Świadczyć może o~tym fakt, że podczas profilowania wykonania nie występowały bloki \textit{Compile}, co ma miejsce podczas wykonania wyizolowanych przykładów.

Kompilacja do WebAssembly zwiększyła wydajność wariantu algorytmu SHT \textit{non-LUT} we wszystkich testowanych środowiskach (rys. \ref{plot:wasm}). Jednak największy wpływ miała na środowiska przeglądarek internetowych, które zdają się lepiej reagować na te metodę akceleracji niż środowisko NodeJS. W~wariancie SHT \textit{LUT} za to nie zaobserwowano żadnej poprawy względem wykonania sekwencyjnego. dodatkowo nie zaobserwowano optymalizacji czasu wykonania dla $S_\theta \geq 5$. Warianty te różnią się jedynie zmniejszoną liczbą wywołać funkcji trygonometrycznych. Założyć można zatem, że zaobserwowana optymalizacja przeglądarki Mozilla Firefox dotyczyła właśnie ich i~nie odbyła w~momencie kompilacji do WebAssembly. 

Przypadek algorytmu CHT pokazuje jak różne środowiska mogą reagować na ten sam kod. W~przeglądarce Google Chrome z~silnikiem V8 zaobserwowano poprawę wydajności, gdzie wykonanie z~wykorzystaniem WebAssembly było \tms{1.70} szybsze od wykonania sekwencyjnego dla $n = 1$. Mozilla Firefox jednak zachowuje się inaczej. Dla $n = 1$ wykonanie z~wykorzystaniem WebAssembly było \tms{1.63} szybsze, niż wykonanie sekwencyjne, jednak dla $n = 10$ czas wykonania był już taki sam jak czas wykonania sekwencyjnego. Na tej podstawie można wyciągnąć wniosek, że w~przypadku przeglądarki Mozilla Firefox duża liczba iteracji generuje narzut, który niweluje korzyści, jakie płyną z~ogólnej poprawny wydajności dzięki WebAssembly. 

Kompilacja do WebAssembly z~włączonym procesem automatycznej wektoryzacji w~narzędziu Emscripten nie przyniosła żadnej różnicy w~porównaniu do wykonania zwykłego kodu WebAssembly. Natomiast w~przypadku ręcznej optymalizacji, czego przykład pokazany został na listingu \ref{lst:simd}, poprawa wydajności zależała od środowiska. Na rysunku \ref{plot:wasm_simd_explicit} zaobserwować można, że wszystkie środowiska pozytywnie zareagowały na zastosowaną metodę dla wariantu algorytmu SHT \textit{non-LUT}. Najszybsza znów okazała się przeglądarka Google Chrome, osiągając wynik \tms{1.50} lepszy od wykonania sekwencyjnego. W~wariancie SHT \textit{LUT} jednak wykonanie sekwencyjne było \tms{1.15} szybsze. Wykonanie wariantu CHT w~przeglądarka Mozilla Firefox, w~porównaniu do wykonania sekwencyjnego, było wolniejsza, a~wykonanie w~środowiskach Google Chrome i~NodeJS zyskało na wydajności.

Dla analizowanego kąta obrotu $\theta = \frac{\pi}{2}$ wykryto duże różnice w~akumulatorze pomiędzy wykonaniem SHT WASM a~sekwencyjnym SHT. Różnica widoczna jest na rysunku \ref{fig:diff:wasm}. Oprócz wyraźnej linii różnica występuje też dla wielu pojedynczych pól akumulatora i~występować może z~różnicy w~reprezentacji w~algorytmie liczb zmiennoprzecinkowych, co generuje różnice w~wyliczeniach pól akumulatora.

\section{Współbieżność}

Zrównoleglenie wykonania bezdyskusyjnie przynosi wzrost wydajności. Jednak, aby mogło być zastosowane algorytm musi dać się zrównoleglić. Również docelowe środowiska muszą tę współbieżność wspierać. Wspólnym mianownikiem wszystkich badanych środowisk jest wielowątkowość w~postaci Worker'ów. Oprócz nich środowisko NodeJS posiada natywne mechanizmy komunikacji między procesami, jednak nie są one przedmiotem badań w~tej pracy.

Komunikacja pomiędzy Worker'ami odbywa się z~wykorzystaniem interfejsów WebWorker API. Dostarczają one mechanizm wiadomości, które są asynchronicznie przetwarzane przez Worker'y, które jako niezależne wątki mają osobną pętlę zdarzeń. Wiadomość jest w~abstrakcji przetwarzania zdarzeniem. Metoda \lstinline{Worker.postMessage(...)} oraz definiowana funkcja \lstinline{self.onmessage} pozwala na wysyłanie i~obsługę zdarzeń wiadomości pomiędzy wątkami. Mechanizm ten jest bardzo prosty, niweluje problemy wyścigów oraz współdzielenia zasobów, ponieważ wszystkie dane są albo kopiowane, albo transferowane z~jednego wątku do drugiego. Jego wadą jest sam interface, który wymaga tworzenia dodatkowych mechanizmów serializacji i~parsowania złożonych wiadomości. Z~pomocą przychodzi, zastosowana przy implementacji Worker'ów na potrzeby badan, biblioteka Comlink \cite{comlink}. Pozwala ona na zdalne wywoływanie funkcji i~synchronizację zmiennych, która interfejsem nie odbiega od sposobu interakcji z~metodami i~atrybutami obiektów. Jest ona kompatybilna ze wszystkimi badanymi środowiskami. 

Implementacja Worker'ów wymagała przystosowania algorytmów. Podzielono wykonanie głównych ich pętli pomiędzy wątki. W~testach badano czasy wykonania dla liczby wątków $c=4$. Aby zapewnić kompatybilność ze wszystkimi środowiskami konieczne było zbudowanie dwóch wersji biblioteki. Biblioteka w~środowisku Deno była uruchamiana z~pominięciem procesu budowania, ponieważ Deno pozwala bezpośrednio na uruchomienie kodu w~języku TypeScript. 

Dla środowiska przeglądarki internetowej konieczne jest użycie loader'a \lstinline{worker-loader}, który, z~opcją \lstinline{inline: 'no-fallback'}, odpowiedzialny jest za dołączenie kodu Worker'a do kodu biblioteki, aby taka biblioteka mogła być ponownie zbudowana razem z~docelową aplikacją. W~standardowym scenariuszu, kiedy Worker budowany jest razem z~aplikacją, Worker jest pobierany jako osobny plik przez przeglądarkę. Kod musi być dołączony w~tym samym pliku, ponieważ niemożliwym jest, bez dodatkowych skryptów, wskazanie i~skopiowanie pliku Worker'a w~procesie budowania.

Dla środowiska NodeJS konieczne jest jawne wskazanie środowiska, dla którego przeznaczona jest budowana biblioteka poprzez ustawienie opcji \lstinline{target: 'node'}. NodeJS do obsługi Worker'ów używa wbudowanego modułu, WebWorker API jest lekko zmodyfikowane, a~konstruktor obiektu \lstinline{Worker} nie jest dostępny jako symbol globalny. Narzędzie Webpack musi uwzględnić wszystkie te elementy. Niekompatybilność ta uniemożliwia w~wypadku tej metody akceleracji zbudowanie pliku kompatybilnego ze wszystkimi środowiskami. Na rysunku \ref{fig:workers-struct} przedstawiono diagram zależności modułów w~kodzie, w~tym adapterów, które zapewnia kompatybilność kodu algorytmu ze wszystkimi środowiskami. Kod części właściwego algorytmu, który tworzy i~wywołuje funkcje Worker'ów w~środowiska Deno jest oderwana od jego odpowiednika wspólnego dla przeglądarek i~NodeJS. Jest to spowodowane niekompatybilnością API biblioteki Comlink w~wersji na środowisko Deno.

\begin{figure}[h]
    \centering
    \includegraphics[width=\linewidth]{diagrams/out/workers.png}
    \caption{Sieć zależności pomiędzy modułami implementującymi algorytm SHT w~wersji \textit{non-LUT}, która zapewnia kompatybilność ze wszystkimi środowiskami.}
    \label{fig:workers-struct}
\end{figure}

\subsection{Wyniki pomiarów}

Pomiary dla wykonania wielowątkowego przeprowadzone były przy przeznaczonych do tego 4 fizycznych rdzeni procesora, aby uniknąć wpływu hyper-threading'u na czas wykonania. Wymagało to przypisania 8 wirtualnych rdzeni używając maski \lstinline{0x000000ff} jako konfiguracji narzędzia \lstinline{taskset}.



\begin{figure}[h]
    \groupBenchmark{
        \plotBenchmark{js-workers_theta_SHT_Simple_node.csv}{nodeColor}{{}}
        \addlegendentry{Node}

        \plotBenchmark{js-workers_theta_SHT_Simple_deno.csv}{denoColor}{{}}
        \addlegendentry{Deno}

        \plotBenchmark{js-workers_theta_SHT_Simple_Firefox.csv}{firefoxColor}{{}}
        \addlegendentry{Firefox}

        \plotBenchmark{js-workers_theta_SHT_Simple_Chrome.csv}{chromeColor}{{}}
        \addlegendentry{Chrome}

        \seqReference
    } {
        \plotBenchmark{js-workers_theta_SHT_Simple_Lookup_node.csv}{nodeColor}{{}}
        \addlegendentry{Node}

        \plotBenchmark{js-workers_theta_SHT_Simple_Lookup_deno.csv}{denoColor}{{}}
        \addlegendentry{Deno}

        \plotBenchmark{js-workers_theta_SHT_Simple_Lookup_Firefox.csv}{firefoxColor}{{}}
        \addlegendentry{Firefox}

        \plotBenchmark{js-workers_theta_SHT_Simple_Lookup_Chrome.csv}{chromeColor}{{}}
        \addlegendentry{Chrome}

        \seqReferenceLookup
    }{
        \plotBenchmark{js-workers_theta_CHT_Simple_node.csv}{nodeColor}{{}}
        \addlegendentry{Node}

        \plotBenchmark{js-workers_theta_CHT_Simple_deno.csv}{denoColor}{{}}
        \addlegendentry{Deno}

        \plotBenchmark{js-workers_theta_CHT_Simple_Firefox.csv}{firefoxColor}{{}}
        \addlegendentry{Firefox}

        \plotBenchmark{js-workers_theta_CHT_Simple_Chrome.csv}{chromeColor}{{}}
        \addlegendentry{Chrome}

        \seqReferenceCircle
    }[1700][500][10]
    \caption{Wyniki pomiarów czasu wydajności dla wykonania SHT i CHT z wykorzystaniem czterech Worker'ów.}
    \label{plot:workers}
\end{figure}


\begin{wraptable}{r}{7.9cm}
    \caption{Przyspieszenie i jego efektywność dla metody akceleracji z wykorzystaniem Worker'ów \mbox{($S_\theta = 1, n=1, p = 4$)}.}
    \label{tab:speedup}
    \begin{tabular}{rlrr}%
        \hline
        Alg.                 & Środ.   & Przysp. & Efek. \\
        \hline
        SHT \textit{non-LUT} & Chrome  & 2.78    & 0.69  \\
        SHT \textit{non-LUT} & Firefox & 2.89    & 0.72  \\
        SHT \textit{non-LUT} & Node    & 3.19    & \textcolor{green!70!black}{0.80}  \\
        SHT \textit{non-LUT} & Deno    & 2.56    & \textcolor{red!70!black}{0.64}  \\
        \hline
        SHT \textit{LUT}     & Chrome  & 1.84    & 0.46  \\
        SHT \textit{LUT}     & Firefox & 1.83    & 0.46  \\
        SHT \textit{LUT}     & Node    & 1.66    & \textcolor{red!70!black}{0.41}  \\
        SHT \textit{LUT}     & Deno    & 1.94    & \textcolor{green!70!black}{0.48}  \\
        \hline
        CHT                  & Chrome  & 1.19    & \textcolor{red!70!black}{0.30} \\
        CHT                  & Firefox & 1.29    & \textcolor{green!70!black}{0.32}  \\
        CHT                  & Node    & 1.20    & \textcolor{red!70!black}{0.30}  \\
        CHT                  & Deno    & 1.22    & 0.31  \\
        \hline
    \end{tabular}
\end{wraptable}

Na rysunku \ref{plot:workers} pokazane zostały wykresy z~wynikami pomiarów czasów wykonania z~wykorzystaniem czterech Worker'ów. Jak w~większości przypadków również i~tutaj, dla obydwu wariantów SHT, prym we wzroście wydajności wiedzie przeglądarka Google Chrome. Przy krótszych czasach wykonania dla wariantu CHT uwydatnia się większa niż dla poprzednich metod wartość odchylenia standardowego. Szczególnie widoczna jest w~przypadku przeglądarki Mozilla Firefox. Spowodowane jest to asynchronicznością komunikacji z~Worker'em, gdzie wiadomości nie są obsługiwane od razu, a~trafiają do pętli zdarzeń. W~zależności od implementacji, jak i~liczby zdarzeń do obsłużenia przez pętlę, czasy te mogą się różnić. Kod Worker'ów jest również osobno optymalizowany, co zwiększa skalę zjawiska zimnego startu w~przypadku tej metody akceleracji.  Na rysunku \ref{fig:profiler-workers} widać wynik profilowania pierwszych wykonań kody na stworzonych instancjach Worker'ów. W~pierwszych wykonaniach widać bloki \textit{Optimise Code}, a~dalsze wykonania wykonują się już po optymalizacji. Podczas długiej przerwy pomiędzy wykonaniami proces optymalizacji zachodził dla wątku głównego.

\begin{figure}[h]
    \centering
    \includegraphics[width=\linewidth]{img/workers-profiler.png}
    \caption{Wynik profilowania wykonania algorytmu CHT w~przeglądarce Google Chrome z~użyciem Worker'ów.}
    \label{fig:profiler-workers}
\end{figure}

W tabeli \ref{tab:speedup} przedstawiono miarę przyspieszenia obliczeń oraz jej efektywność. Największe przyspieszenie, a~zarazem jego efektywność ma miejsce w~przypadku wariantu SHT \textit{non-LUT} i~środowiska NodeJS. Różnica wyników pomiędzy wariantami SHT \textit{non-LUT} i~SHT \textit{LUT} jest znacząca i~kolejny raz wskazuje na funkcje trygonometryczne i~ogólnie zmiennoprzecinkowe jako czynnik, na który trzeba zwrócić uwagę podczas optymalizacji algorytmów. W~wariancie CHT przyspieszenie jest niewielkie, ponieważ zrównolegleniu poddany został sam proces głosowania na środek okręgu. Jeśli w~danych wejściowych większa liczba pikseli byłaby zapalona, etap głosowania byłby bardziej wymagający, co zwiększyłoby efektywność przyspieszenia.

Worker'y współdzielą pamięć akumulatora w~postaci obiektu \lstinline{SharedArrayBuffer}. Z~uwagi na to, że każdy Worker działał na własnym fragmencie pamięci podczas procesu głosowania, nie było konieczne stosowanie operacji atomowych z~wykorzystaniem metod interfejsu \lstinline{Atomic}. Użycie takiego interfejsu spowalniało wykonanie algorytmu w~testach podczas implementacji. 


\section{GPGPU}

Implementacja akceleracji z~użyciem układu graficznego wymaga dostosowania algorytmu do masowego zrównoleglenia wydzielając funkcje jądra (kernel), która wykonywać się będzie równolegle na setkach rdzeni jednocześnie zawsze rozwijając wszystkie gałęzie wyrażeń warunkowych. Stworzona implementacja korzysta z~mechanizmów generowania grafiki WebGL API, co nakłada dodatkowe ograniczenia w~przeciwieństwie do standardowego podejścia. Kernel nie ma możliwości modyfikacji danych w~pamięci współdzielonej, a~jego jedynym wynikiem jest wektor liczb. W~zależności od algorytmu liczby te mogą być różnie interpretowane - jako zmienne logiczne, liczby stało i~zmiennoprzecinkowe, czy też w~standardowy sposób jako kolor piksela.

\subsection{Standard Hough Transform}

Przystosowanie procesu głosowania dla wariantu SHT wymagało podejście skoncentrowanego na generowaniu jednej wartości elementu akumulatora. Proces ten jest masowo zrównoleglony przez układ graficzny równolegle generując wszystkie wartości w~akumulatorze. 

Jeśli jeden piksel na obrazie odwzorowywany jest na sinusoidę w~przestrzeni akumulatora, a~jeden punkt w~przestrzeni akumulatora to jedna linia na obrazie, to odwrotnym podejściem do głosowania jest zliczenie, dla jednego elementu akumulatora, ile pikseli na obrazie przechodzi przez potencjalną linię, na którą może być on odwzorowany. Żeby uwzględnić wszystkie piksele linii, których zbocze $m > 1$, iteracja po przestrzeni obrazu odbywa się również w~wariancie, w~którym układ współrzędnych jest obrócony o~$90^\circ$. Dlatego równania (\ref{eq:gpuD}) i~(\ref{eq:gpuE}) różnią się obliczające sumę zapalonych pikseli, w~zależności kąta nachylenia zamieniają współrzędne $x$ i~$y$ miejscami.
Końcowy wzór na wartość pola akumulatora przedstawiony został na równaniu (\ref{eq:gpuF}).
\begin{align}
    \rho(\theta) &= x\cos{\theta} + y\sin{\theta} \label{eq:gpuA} \\
    y_a(\rho,\theta,x) &= \frac{\rho-x\sin{\theta}}{\cos{\theta}} \label{eq:gpuB}\\
    x_a(\rho,\theta,y) &= \frac{\rho-y\cos{\theta}}{\sin{\theta}} \label{eq:gpuC}\\
    A_x(\rho, \theta) &= 
    \begin{cases} 
    \sum^{w-1}_{x=0}I(x, y_a(\rho,\theta,x)),&\rho \in \left[\frac{\pi}{4}, \frac{3\pi}{4}\right) \lor \rho \in \left[\frac{5\pi}{4}, \frac{7\pi}{4}\right) \\
    0,& \text{dla pozostałych }\rho
    \end{cases} \label{eq:gpuD}\\
    A_y(\rho, \theta) &= 
    \begin{cases} 
    \sum^{h-1}_{y=0} I(x_a(\rho,\theta,y), y),& \rho \in \left[0, \frac{\pi}{4}\right) \lor \rho \in \left[\frac{3\pi}{4}, \frac{5\pi}{4}\right) \lor \rho \in \left[\frac{7\pi}{4}, 2\pi\right) \\
    0,& \text{dla pozostałych }\rho
    \end{cases} \label{eq:gpuE}\\
    A(\rho, \theta) &= A_x(\rho, \theta) + A_y(\rho, \theta) \label{eq:gpuF}
\end{align}
\begin{eqexpl}
    \item{$x, y$} współrzędne piksela na obrazie;
    \item{$I(x, y)$} wartość piksela na obrazu.
    \item{$A_x, A_y$} częściowa zawartość akumulatora w~zależności od kąta padania prostej;
    \item{$A$} zawartość akumulatora.
    \item{$w, h$} szerokość i~wysokość obrazu;
    \item{$x_a, y_a$} funkcje obliczające drugą współrzędną linii na podstawie odwzorowania z~akumulatora;
    \item{$\rho$} odległość prostej od środka układu współrzędnych;
    \item{$\theta$} obrót prostej od wokół układu współrzędnych;
\end{eqexpl}
\vspace{0.5cm}


Zbudowanie biblioteki nie wymagało dodatkowych zmian konfiguracji bazowej narzędzia Webpack. Do implementacji obliczeń z~wykorzystaniem WebGL wykorzystano bibliotekę \textit{gpu.js}. Listing \ref{lst:gpu-sht} przedstawia funkcję tworzącą kernel dla wariantu SHT\textit{non-LUT}. Jedynym argumentem funkcji kernela jest tablica z~wejściowym obrazem. Pozostałe parametry dostarczane są jako stałe, co przyspiesza do nich dostęp. Sprawia to jednak, że dla każdego zestawu opcji istnieje konieczność stworzenia nowego kernela. W~implementowanym algorytmie kernele zapisywane są w~mapie, której kluczem są wartości opcji. Kernel zwraca jedną liczbę stanowiącą wynik równania (\ref{eq:gpuF}).

Na listingu również zobaczyć możemy utworzenie funkcji za pomocą konstruktora, która zwraca właściwa funkcję i~jest od razu wykonywana - \lstinline{new Function(`return function (testImage/* : number[] */){/*...*/}`)()}. Jest to niezbędne, aby mieć pełną kontrolę nad wejściem do mechanizmów transpilacji biblioteki \textit{gpu.js}, ponieważ kod w~procesie budowania biblioteki jest minifikowany, co zmienia nazwy zmiennych i~stosuje skrócone notacje, z~którymi biblioteka \textit{gpu.js} nie umie sobie poradzić. Na listingu \ref{lst:gpu-mini} widać przykład zminifikowanego kodu błędnie transpilowanego do języka GLSL, gdzie problem stanowi połączenie post-inkrementacji z~wyrażeniem logicznym.

\begin{lstlisting}[language=JavaScript, float=ht, label=lst:gpu-mini, caption=Przykład minifikacji kodu błędnie transpilowanego do języka GLSL przez bibliotekę \textit{gpu.js}]
// before
if(offset < this.constants.width * this.constants.height &&
                testImage[offset] == 1) {
    acc+=1;
}
// after
t<this.constants.width*this.constants.height&&1==e[t]&&a++;
\end{lstlisting}

% TODO: new Function  ze stringa czemu

\begin{lstlisting}[language=JavaScript, float=ht, label=lst:gpu-sht, caption=Funkcja tworząca kernel dla wariantu SHT \textit{non-LUT}]

import { GPU, IKernelRunShortcutBase } from "gpu.js";

export const gpu = new GPU();
export function createSHTSimpleKernel(
    hsWidth: number, hsHeight: number, width: number, height: number,
    samplingRho: number, samplingTheta: number
) {
  return gpu
    .createKernel(
      new Function(`return function (testImage/* : number[] */) {
        // ...
        let acc = 0;
        // ...
        return acc;
      }`)(),
      {
        output: [hsWidth * hsHeight],
        constants: {hsWidth, width, height, samplingRho, samplingTheta},
      }
    )
    .setLoopMaxIterations(Math.max(width, height))
    .setOptimizeFloatMemory(true) as IKernelRunShortcutBase<Float32Array>;
}
\end{lstlisting}

\subsection{Circle Hough Transform}

W wariancie CHT zaimplementowano łącznie 3 kernele. Dwa z~nich odpowiedzialne są za obliczenie splotu obrazu z~operatorem Sobela, aby wyznaczyć gradient krawędzi. Zwracają one wynik w~postaci tekstury, której dane znajdują się bezpośrednio na GPU. Tekstury te są następnie wejściem do kernela wyznaczającego wartość pola akumulatora w~procesie głosowania na środki okręgów. Dzięki takiemu złożeniu kerneli dane splotów nie przechodzą przez CPU, aby trafić z~powrotem do układ graficznego. Wartość piksela wyznaczana jest na podstawie liczby linii, o~długości wyznaczanej na podstawie maksymalnego i~minimalnego wyszukiwanego promienia, przechodzących przez dany punkt w~przestrzeni akumulatora.

\subsection{Wyniki pomiarów}

Na rysunku \ref{plot:gpu} przedstawiono wyniki pomiarów czasu wykonania w wykorzystaniem biblioteki \textit{gpu.js}. Widać, że metoda ta cechuje się bardzo niestabilnymi czasami wykonania. Brak dużego odchylenia standardowego sugeruje, że czas zależeć może od ogólnego obciążenia systemu oraz od optymalizacji wykorzystanych zasobów procesora graficznego, co sugeruje podobny czas wykonania dla $S_\theta \in \{5,6,7\}$ w wariancie SHT \textit{LUT}. Dla SHT w każdym środowisku i wariancie algorytmu udało się przyspieszyć czas wykonania bądź osiągnąć porównywalny względem wykonania sekwencyjnego w języku C++. Przyspieszenie wynosi \tms{3.81} dla wariantu SHT \textit{non-LUT} i oscyluje w okolicy \tms{1} dla SHT \textit{LUT} oraz $S_\theta = 1$. Ze względu na odwrotne podejście do generowania wartości w akumulatorze oraz na zmniejszoną precyzje obliczeń dla liczb zmiennoprzecinkowych w procesorach graficznych, różnica pomiędzy wygenerowanym akumulatorem dla tej metody, a wariantem sekwencyjnym jest znacząca i została pokazana na rysunku \ref{fig:diff:gpu}. Nie wpływa ona jednak znacząco na jakoś detekcji. 

Wąskim gardłem w metodach akceleracji z wykorzystaniem układów graficznych jest komunikacja CPU z GPU. Wysyłanie i odbieranie danych zajmuje nieproporcjonalnie dużo czasu, co widać na rysunku \ref{fig:profiler-gpu}. 



\begin{figure}
    \groupBenchmark{
        %\plotBenchmark{js-gpu_theta_SHT_Simple_node.csv}{nodeColor}{{}}
        %\addlegendentry{Node}

        %\plotBenchmark{js-gpu_theta_SHT_Simple_deno.csv}{denoColor}{{}}
        %\addlegendentry{Deno}

        \plotBenchmark{js-gpu_theta_SHT_Simple_Firefox.csv}{firefoxColor}{{}}
        \addlegendentry{Firefox}

        \plotBenchmark{js-gpu_theta_SHT_Simple_Chrome.csv}{chromeColor}{{}}
        \addlegendentry{Chrome}

        \seqReference
    } {
        %\plotBenchmark{js-gpu_theta_SHT_Simple_Lookup_node.csv}{nodeColor}{{}}
        %\addlegendentry{Node}

        %\plotBenchmark{js-gpu_theta_SHT_Simple_Lookup_deno.csv}{denoColor}{{}}
        %\addlegendentry{Deno}

        \plotBenchmark{js-gpu_theta_SHT_Simple_Lookup_Firefox.csv}{firefoxColor}{{}}
        \addlegendentry{Firefox}

        \plotBenchmark{js-gpu_theta_SHT_Simple_Lookup_Chrome.csv}{chromeColor}{{}}
        \addlegendentry{Chrome}

        \seqReferenceLookup
    }[550][200]
    \label{plot:gpu}
    \caption{WebGL SHT execution benchmark results. Gray area shows sequential JavaScript execution performance range.}
\end{figure}


\begin{figure}[h]
    \centering
    \includegraphics[width=\linewidth]{img/gpu-profiler.png}
    \caption{Wynik profilowania wykonania algorytmu CHT w~przeglądarce Google Chrome z wykorzystaniem biblioteki \textit{gpu.js}.}
    \label{fig:profiler-gpu}
\end{figure}


W przypadku wersji CHT nie udało się uzyskać przyspieszenia, a dodatkowo implementacja w taki sam sposób zwiększyła złożoność obliczeniową, która stała się ponad-liniowa. Dla każdego punktu akumulatora wszystkie punkty wokół niego na obrazie, nie dalej niż maksymalnie szykany promień, muszą zostać sprawdzone pod kątem generowania linii, która przecina generowany punkt akumulatora. Na tej podstawie można oszacować złożoność jako $O(n^2)$. Dla dużych promieni rzędu 220 pikseli, czas wykonania potrafi wynieść 25s. W ramach pracy nie analizowano dalej tego problemu, a wszelkie próby poprawy złożoności algorytmu nie przyniosły efektu. Doświadczenia z udaną akceleracją wariantu SHT algorytmu pozwala sądzić, że zaimplementowanie wersji z jednym trójwymiarowym akumulatorem przyniosłoby oczekiwane rezultaty. Ograniczeniem w takiej implementacji byłby rozmiar akumulatora. W obydwu przeglądarkach maksymalny rozmiar wyjścia to $16384\times16384$. Rozwiązaniem tego problemu może być podział wykonania na części, gdzie każda z nich generuje maksymalny możliwy fragment akumulatora.

\section{Problem zimnego startu}


\section{Przykład przetwarzania obrazu w przeglądarce internetowej}

\chapter{Podsumowanie}


\newcommand{\comma}{, }
\begin{table}
    \label{tab:envs}
    \caption{Comparison of implemented methods in analyzed environments. The general comparison was done using Chrome as a reference point and geometric mean for times comparable with Chrome.}

    \begin{tabularx}{\linewidth}{X r r r r}%
        \hline
                                 & \multicolumn{4}{c}{\bfseries Execution time[ms]}                                                             \\
        \bfseries Method         & \bfseries Chrome                                 & \bfseries Firefox     & \bfseries Node  & \bfseries Deno

        \csvreader[
            head to column names,
            before first line=                                                                                                                  \\\hline,
            late after line=                                                                                                                    \\,
            late after last line=                                                                                                               \\\hline
        ]{../../benchmark/environments/envs.csv}{}% use head of csv as column names
        {
        \name\                   & \chrome\ (\chromeF)                              & \firefox\ (\firefoxF) & \node\ (\nodeF) & \deno\ (\denoF)
        }% specify your coloumns here 
        \bfseries Geometric mean &
        \csvreader[
            head to column names,
            late after last line=                                                                                                               \\\hline

        ]{../../benchmark/environments/geoMeans.csv}{}% use head of csv as column names
        {
        (\chrome)                & (\firefox)                                       & (\node)               & (\deno)
        }% specify your coloumns here 
    \end{tabularx}

\end{table}


\clearpage




%\bibliographystyle{plalpha}
\bibliographystyle{abbrv}
% \bibliographystyle{plain}

%Warning: References should be collected in a~separate file. You can use the programm JabRef for editing. 
%         But please remember, that not all JabRef types of entries are supported by BibTeX.
%         File name below is given without extenion 
%         (here BibTeX will look for "bibiography.bib" in the main directory)
\setlength{\bibitemsep}{2pt} % introduced to make smaller seps between bibliographic items
\bibliography{bibliography}

\appendix
\chapter{Opis załączonej płyty CD/DVD}
Dołączona płyta zawiera wszystkie pliki wykorzystane do stworzenia projektu, dokumentacji i~niniejszej pracy. Na płycie znajduje się cała zawartość projektu zaimplementowanego w~strategii \textit{monorepo} i~w katalogu głównym znajdują się pliki odpowiedzialne za jej obsługę.  Pozostałe pliki podzielono na następujące katalogi:

\begin{itemize}
    \item \texttt{.vscode} - ustawienia środowiska VsCode,
    \item \texttt{benchmark} - wyniki testów wydajności w~formacie \texttt{*.csv},
    \item \texttt{docs} - notatki oraz treść i~materiały niniejszej pracy,
    \begin{itemize}[topsep=0pt]
        \item[\textbullet] \texttt{notes} - notatki
        \item[\textbullet] \texttt{out} - pliki będące wynikiem kompilacji plików źródłowych pracy
        \begin{itemize}[topsep=0pt]
            \item[\textbullet] \texttt{W04N\_241292\_2022\_praca magisterska.pdf} - plik PDF z~zawartością pracy
        \end{itemize}
        \item[\textbullet] \texttt{src} - pliki źródłowe pracy
    \end{itemize} 
    \item \texttt{packages} - zbiór pakietów \textit{monorepo}
    \begin{itemize}[topsep=0pt]
        \item[\textbullet] \texttt{benchmark} - Biblioteka \textit{benchmark}
        \item[\textbullet] \texttt{cpp-sequential} - Implementacja metody sekwencyjnej w~C++
        \item[\textbullet] \texttt{frontend} - Rzeczywisty przykład użycia
        \item[\textbullet] \texttt{js-benchmarks} - Przeprowadzanie pomiarów wydajności
        \item[\textbullet] \texttt{js-gpu} - Implementacja metody WebGL
        \item[\textbullet] \texttt{js-sequential} - Implementacja metody sekwencyjnej w~JavaScript
        \item[\textbullet] \texttt{js-workers} - Implementacja metody Workers
        \item[\textbullet] \texttt{meta} - Typy języka TypeScript współdzielone pomiędzy implementacje
        \item[\textbullet] \texttt{node-cpp-sequential} - Implementacja metody Native C++ Addon w~NodeJS
        \item[\textbullet] \texttt{test-simple} - Strona testowa dla implementowanych metod
        \item[\textbullet] \texttt{wasm-sequential} - Implementacja metody WASM oraz jej wariantów
    \end{itemize} 
    \item \texttt{test} - zbiór testowych obrazów,
\end{itemize}

%\include{appendixB}

%%Uncomment the lines below, if you want to use index
%%\chapterstyle{noNumbered}
%%\phantomsection 
%%\addcontentsline{toc}{chapter}{Index}
%%\printindex


\end{document}
