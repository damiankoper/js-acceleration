\chapter{Opis załączonej płyty CD/DVD}
Dołączona płyta zawiera wszystkie pliki wykorzystane do stworzenia projektu, dokumentacji i~niniejszej pracy. Na płycie znajduje się cała zawartość projektu zaimplementowanego w~strategii \textit{monorepo} i~w katalogu głównym znajdują się pliki odpowiedzialne za jej obsługę.  Pozostałe pliki podzielono na następujące katalogi:

\begin{itemize}
    \item \texttt{.vscode} - ustawienia środowiska VsCode,
    \item \texttt{benchmark} - wyniki testów wydajności w~formacie \texttt{*.csv},
    \item \texttt{docs} - notatki oraz treść i~materiały niniejszej pracy,
    \begin{itemize}[topsep=0pt]
        \item[\textbullet] \texttt{notes} - notatki
        \item[\textbullet] \texttt{out} - pliki będące wynikiem kompilacji plików źródłowych pracy
        \begin{itemize}[topsep=0pt]
            \item[\textbullet] \texttt{W04N\_241292\_2022\_praca magisterska.pdf} - plik PDF z~zawartością pracy
        \end{itemize}
        \item[\textbullet] \texttt{src} - pliki źródłowe pracy
    \end{itemize} 
    \item \texttt{packages} - zbiór pakietów \textit{monorepo}
    \begin{itemize}[topsep=0pt]
        \item[\textbullet] \texttt{benchmark} - Biblioteka \textit{benchmark}
        \item[\textbullet] \texttt{cpp-sequential} - Implementacja metody sekwencyjnej w~C++
        \item[\textbullet] \texttt{frontend} - Rzeczywisty przykład użycia
        \item[\textbullet] \texttt{js-benchmarks} - Przeprowadzanie pomiarów wydajności
        \item[\textbullet] \texttt{js-gpu} - Implementacja metody WebGL
        \item[\textbullet] \texttt{js-sequential} - Implementacja metody sekwencyjnej w~JavaScript
        \item[\textbullet] \texttt{js-workers} - Implementacja metody Workers
        \item[\textbullet] \texttt{meta} - Typy języka TypeScript współdzielone pomiędzy implementacje
        \item[\textbullet] \texttt{node-cpp-sequential} - Implementacja metody Native C++ Addon w~NodeJS
        \item[\textbullet] \texttt{test-simple} - Strona testowa dla implementowanych metod
        \item[\textbullet] \texttt{wasm-sequential} - Implementacja metody WASM oraz jej wariantów
    \end{itemize} 
    \item \texttt{test} - zbiór testowych obrazów,
\end{itemize}
