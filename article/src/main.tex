%%%%%%%%%%%%%%%%%%%%%%%%%%%%%%%%%%%%%%%%%%%%%%%%%%%%%%%%%%%%%%%%%%%%%
%%                                                                 %%
%% Please do not use \input{...} to include other tex files.       %%
%% Submit your LaTeX manuscript as one .tex document.              %%
%%                                                                 %%
%% All additional figures and files should be attached             %%
%% separately and not embedded in the \TeX\ document itself.       %%
%%                                                                 %%
%%%%%%%%%%%%%%%%%%%%%%%%%%%%%%%%%%%%%%%%%%%%%%%%%%%%%%%%%%%%%%%%%%%%%

%%\documentclass[referee,sn-basic]{sn-jnl}% referee option is meant for double line spacing

%%=======================================================%%
%% to print line numbers in the margin use lineno option %%
%%=======================================================%%

%%\documentclass[lineno,sn-basic]{sn-jnl}% Basic Springer Nature Reference Style/Chemistry Reference Style

%%======================================================%%
%% to compile with pdflatex/xelatex use pdflatex option %%
%%======================================================%%

%%\documentclass[pdflatex,sn-basic]{sn-jnl}% Basic Springer Nature Reference Style/Chemistry Reference Style

%%\documentclass[sn-basic]{sn-jnl}% Basic Springer Nature Reference Style/Chemistry Reference Style


\documentclass[sn-mathphys]{style/sn-jnl}% Math and Physical Sciences Reference Style


%%\documentclass[sn-aps]{sn-jnl}% American Physical Society (APS) Reference Style
%%\documentclass[sn-vancouver]{sn-jnl}% Vancouver Reference Style
%%\documentclass[sn-apa]{sn-jnl}% APA Reference Style
%%\documentclass[sn-chicago]{sn-jnl}% Chicago-based Humanities Reference Style
%%\documentclass[sn-standardnature]{sn-jnl}% Standard Nature Portfolio Reference Style
%%\documentclass[default]{sn-jnl}% Default
%%\documentclass[default,iicol]{sn-jnl}% Default with double column layout

%%%% Standard Packages
%%<additional latex packages if required can be included here>
%%%%

%%%%%=============================================================================%%%%
%%%%  Remarks: This template is provided to aid authors with the preparation
%%%%  of original research articles intended for submission to journals published 
%%%%  by Springer Nature. The guidance has been prepared in partnership with 
%%%%  production teams to conform to Springer Nature technical requirements. 
%%%%  Editorial and presentation requirements differ among journal portfolios and 
%%%%  research disciplines. You may find sections in this template are irrelevant 
%%%%  to your work and are empowered to omit any such section if allowed by the 
%%%%  journal you intend to submit to. The submission guidelines and policies 
%%%%  of the journal take precedence. A detailed User Manual is available in the 
%%%%  template package for technical guidance.
%%%%%=============================================================================%%%%


\jyear{2022}%

%% as per the requirement new theorem styles can be included as shown below
\theoremstyle{thmstyleone}%
\newtheorem{theorem}{Theorem}%  meant for continuous numbers
%%\newtheorem{theorem}{Theorem}[section]% meant for section-wise numbers
%% optional argument [theorem] produces theorem numbering sequence instead of independent numbers for Proposition
\newtheorem{proposition}[theorem]{Proposition}% 
%%\newtheorem{proposition}{Proposition}% to get separate numbers for theorem and proposition etc.

\theoremstyle{thmstyletwo}%
\newtheorem{example}{Example}%
\newtheorem{remark}{Remark}%

\theoremstyle{thmstylethree}%
\newtheorem{definition}{Definition}%

\raggedbottom
%%\unnumbered% uncomment this for unnumbered level heads


\begin{document}

\title[Analysis and comparison of acceleration methods\dots]{Analysis and comparison of acceleration methods in JavaScript environments based on simplified standard Hough transform algorithm}

%%=============================================================%%
%% Prefix	-> \pfx{Dr}
%% GivenName	-> \fnm{Joergen W.}
%% Particle	-> \spfx{van der} -> surname prefix
%% FamilyName	-> \sur{Ploeg}
%% Suffix	-> \sfx{IV}
%% NatureName	-> \tanm{Poet Laureate} -> Title after name
%% Degrees	-> \dgr{MSc, PhD}
%% \author*[1,2]{\pfx{Dr} \fnm{Joergen W.} \spfx{van der} \sur{Ploeg} \sfx{IV} \tanm{Poet Laureate} 
%%                 \dgr{MSc, PhD}}\email{iauthor@gmail.com}
%%=============================================================%%

\author*[1]{\fnm{Damian} \sur{Koper}}\email{kopernickk@gmail.com}
\author[1]{\fnm{Marek} \sur{Woda}}\email{marek.woda@pwr.edu.pl}
% \equalcont{These authors contributed equally to this work.}

\affil*[1]{\orgdiv{Faculty of Information and Communication Technology}, \orgname{Wroclaw University of Science and Technology}}

%%==================================%%
%% sample for unstructured abstract %%
%%==================================%%

\abstract{
    JavaScript has become one of the most widely used programming language in the world. However it has not been widely adopted to perform numerical computing by a community and maintainers for a long time. In this paper, we present analysis of acceleration methods of choice in JavaScript execution environments including browser, Node and Deno.  We focus evenly on adopting the same codebase to take advantage of every method, benchmarking our solutions and analysis of a toolchain building libraries as compatible as possible with multiple environments. To compare performance, we use a simplified standard Hough transform algorithm with threshold as voting phase. As a reference points of our benchmarks, we use sequential version of the algorithm written in both JavaScript and C++.
}

%%================================%%
%% Sample for structured abstract %%
%%================================%%

% \abstract{\textbf{Purpose:} The abstract serves both as a general introduction to the topic and as a brief, non-technical summary of the main results and their implications. The abstract must not include subheadings (unless expressly permitted in the journal's Instructions to Authors), equations or citations. As a guide the abstract should not exceed 200 words. Most journals do not set a hard limit however authors are advised to check the author instructions for the journal they are submitting to.
% 
% \textbf{Methods:} The abstract serves both as a general introduction to the topic and as a brief, non-technical summary of the main results and their implications. The abstract must not include subheadings (unless expressly permitted in the journal's Instructions to Authors), equations or citations. As a guide the abstract should not exceed 200 words. Most journals do not set a hard limit however authors are advised to check the author instructions for the journal they are submitting to.
% 
% \textbf{Results:} The abstract serves both as a general introduction to the topic and as a brief, non-technical summary of the main results and their implications. The abstract must not include subheadings (unless expressly permitted in the journal's Instructions to Authors), equations or citations. As a guide the abstract should not exceed 200 words. Most journals do not set a hard limit however authors are advised to check the author instructions for the journal they are submitting to.
% 
% \textbf{Conclusion:} The abstract serves both as a general introduction to the topic and as a brief, non-technical summary of the main results and their implications. The abstract must not include subheadings (unless expressly permitted in the journal's Instructions to Authors), equations or citations. As a guide the abstract should not exceed 200 words. Most journals do not set a hard limit however authors are advised to check the author instructions for the journal they are submitting to.}

\keywords{javascript, acceleration, sht, standard hough transform, node, browser, deno, webgl, webpack, wasm, simd, workers}

%%\pacs[JEL Classification]{D8, H51}

%%\pacs[MSC Classification]{35A01, 65L10, 65L12, 65L20, 65L70}

\maketitle

\section{Introduction}\label{sec1}

asdasd



\begin{figure}
    \groupBenchmark{
        \plotBenchmark{cpp_theta_SHT_Simple.csv}{cppColor}{{}}
        \addlegendentry{C++}

        \plotBenchmark{js-sequential_theta_SHT_Simple_node.csv}{nodeColor}{{}}
        \addlegendentry{Node}

        \plotBenchmark{js-sequential_theta_SHT_Simple_deno.csv}{denoColor}{{}}
        \addlegendentry{Deno}

        \plotBenchmark{js-sequential_theta_SHT_Simple_Firefox.csv}{firefoxColor}{{}}
        \addlegendentry{Firefox}

        \plotBenchmark{js-sequential_theta_SHT_Simple_Chrome.csv}{chromeColor}{{}}
        \addlegendentry{Chrome}

    } {
        \plotBenchmark{cpp_theta_SHT_Simple_Lookup.csv}{cppColor}{{}}
        \addlegendentry{C++}

        \plotBenchmark{js-sequential_theta_SHT_Simple_Lookup_node.csv}{nodeColor}{{}}
        \addlegendentry{Node}

        \plotBenchmark{js-sequential_theta_SHT_Simple_Lookup_deno.csv}{denoColor}{{}}
        \addlegendentry{Deno}

        \plotBenchmark{js-sequential_theta_SHT_Simple_Lookup_Firefox.csv}{firefoxColor}{{}}
        \addlegendentry{Firefox}

        \plotBenchmark{js-sequential_theta_SHT_Simple_Lookup_Chrome.csv}{chromeColor}{{}}
        \addlegendentry{Chrome}
    } {
        \plotBenchmark{cpp_theta_CHT_Simple.csv}{cppColor}{{}}
        \addlegendentry{C++}

        \plotBenchmark{js-sequential_theta_CHT_Simple_node.csv}{nodeColor}{{}}
        \addlegendentry{Node}

        \plotBenchmark{js-sequential_theta_CHT_Simple_deno.csv}{denoColor}{{}}
        \addlegendentry{Deno}

        \plotBenchmark{js-sequential_theta_CHT_Simple_Firefox.csv}{firefoxColor}{{}}
        \addlegendentry{Firefox}

        \plotBenchmark{js-sequential_theta_CHT_Simple_Chrome.csv}{chromeColor}{{}}
        \addlegendentry{Chrome}
    }
    [3400][850][14]
    \caption{Wyniki pomiarów czasu wydajności dla wykonania sekwencyjnego SHT i CHT.}
    \label{plot:sequential}
\end{figure}


\begin{appendices}


\end{appendices}

%%===========================================================================================%%
%% If you are submitting to one of the Nature Portfolio journals, using the eJP submission   %%
%% system, please include the references within the manuscript file itself. You may do this  %%
%% by copying the reference list from your .bbl file, paste it into the main manuscript .tex %%
%% file, and delete the associated \verb+\bibliography+ commands.                            %%
%%===========================================================================================%%

\bibliography{bibliography}% common bib file
%% if required, the content of .bbl file can be included here once bbl is generated
%%\input sn-article.bbl

%% Default %%
%%\input sn-sample-bib.tex%

\end{document}
