\documentclass[runningheads]{llncs}

\usepackage{pgfplots}
\usepackage{tikz}
\usetikzlibrary{pgfplots.groupplots} % needs to be loaded exactly like this
\usepgfplotslibrary{fillbetween}
\usepgfplotslibrary{colorbrewer}
\usetikzlibrary{patterns}

\usepackage{csvsimple}
\usepackage{siunitx}
\usepackage{tabularx}
\usepackage{subcaption}
\usepackage{graphicx}
\usepackage{xargs}
\usepackage{url}
\usepackage{amssymb}
\usepackage{pifont}
\usepackage{threeparttable,tablefootnote}
\usepackage{wrapfig}
\newcommand{\cmark}{\ding{51}}%
\newcommand{\xmark}{\ding{55}}%

\pgfplotsset{
    table/search path={./../../benchmark},
    every pin edge/.style={solid},
    compat=newest
}
\definecolor{cppColor}{rgb}{0,0,0}
\definecolor{nodeColor}{rgb}{0,0.6,0}
\definecolor{denoColor}{rgb}{0,0,1}
\definecolor{chromeColor}{rgb}{1,0,0}
\definecolor{firefoxColor}{rgb}{1,0.5,0}

\newcommand{\plotBenchmark}[3]{%
    \addplot+ [
        #2,
        #3,
        mark=*,
        mark options={solid,fill=#2, scale=0.5},
        error bars/.cd,
        error bar style={mark size=3pt, solid, #2},
        y dir=both,
        y explicit,
    ]
    table [
            x=sizeTheta,
            y=mean,
            y error=stdev,
            col sep=comma
        ] {#1};
}

\newenvironment{chartBenchmark}
{\begin{tikzpicture}
        \begin{axis} [
                width=\linewidth,
                height=0.45\linewidth,
                legend style={font=\tiny},
                grid,
                grid style=dashed,
                xlabel={$S_\theta$ [pixels per degree]},
                ylabel={Time [ms]},
                mark options={solid},
                legend pos=north west
            ] }
            %
            {
        \end{axis}
    \end{tikzpicture}
}

\newcommandx\groupBenchmark[4][3=1000, 4=800]{
    \begin{tikzpicture}
        \begin{groupplot}[
                group style={group size=2 by 1},
                width=0.52552\linewidth,
                grid,
                grid style=dashed,
                legend style={
                        legend columns=5,
                        font=\tiny,
                    },
                tick label style={font=\tiny},
                every axis title shift=0pt,
                max space between ticks=12,
                ymin=0
            ]

            \nextgroupplot[
                title={SHT Simple},
                ylabel={Time [ms]},
                legend to name={CommonLegend},
                xlabel={$S_\theta$ [pixels per degree]},
                ymax=#3
            ]
            #1

            \nextgroupplot[
                title={SHT Simple Lookup},
                xlabel={$S_\theta$ [pixels per degree]},
                legend to name={CommonLegend},
                ymax=#4
            ]
            #2
        \end{groupplot}
        \path (group c1r1.south east) -- node[below, yshift=-7ex] {\ref{CommonLegend}} (group c2r1.south west);
    \end{tikzpicture}
}


\newcommand\seqReference{
    \plotBenchmark{cpp_theta_SHT_Simple.csv}{cppColor}{dashed}
    \addlegendentry{C++ Sequential};
    \plotBenchmark
    {js-sequential_theta_SHT_Simple_Firefox.csv}
    {firefoxColor}
    {name path=Firefox_Seq,opacity=0}
    \plotBenchmark
    {js-sequential_theta_SHT_Simple_Chrome.csv}
    {chromeColor}
    {name path=Chrome_Seq,opacity=0}
    \addplot [black,opacity=0.1] fill between [of=Chrome_Seq and Firefox_Seq];
}

\newcommand\seqReferenceLookup{
    \plotBenchmark{cpp_theta_SHT_Simple_Lookup.csv}{cppColor}{dashed}
    \addlegendentry{C++ Sequential};
    \plotBenchmark
    {js-sequential_theta_SHT_Simple_Lookup_Firefox.csv}
    {firefoxColor}
    {name path=Firefox_Seq_Lookup,opacity=0}
    \plotBenchmark
    {js-sequential_theta_SHT_Simple_Lookup_Chrome.csv}
    {chromeColor}
    {name path=Chrome_Seq_Lookup,opacity=0}
    \addplot [black,opacity=0.1] fill between [of=Chrome_Seq_Lookup and Firefox_Seq_Lookup];
}



\begin{document}
%
\title{Performance and analysis of acceleration methods in JavaScript environments based on simplified standard Hough transform algorithm}
%
\titlerunning{Performance and analysis of acceleration methods\dots}

\author{Damian Koper \and Marek Woda}
%
% \authorrunning{D. Koper et al.}
% First names are abbreviated in the running head.
% If there are more than two authors, 'et al.' is used.
%
\institute{Faculty of Information and Communication Technology,\\
Wroclaw University of Science and Technology\\
\email{kopernickk@gmail.com, marek.woda@pwr.edu.pl}}
%
\maketitle              % typeset the header of the contribution
%
\begin{abstract}
  In this paper, we present analysis of acceleration methods of choice in JavaScript execution environments including browser, Node and Deno.  We focus evenly on adopting the same codebase to take advantage of every method, benchmarking our solutions and analysis of a toolchain building libraries as compatible as possible with multiple environments. To compare performance, we use a simplified standard Hough transform algorithm with threshold as voting phase. As a reference points of our benchmarks, we use sequential version of the algorithm written in both JavaScript and C++.

  \keywords{javascript \and acceleration \and sht \and standard hough transform \and node \and browser \and deno \and webgl \and webpack \and wasm \and simd \and workers \and coldstart}
\end{abstract}

\section{Introduction}\label{sec:introduction}

TODO: JS in general, JS in numerical computing.

TODO: Environments briefly.

TODO: Acceleration methods briefly.

TODO: Building methods and complex architectures.

TODO: Target:
* overall comparison,
* to identify the most promising acceleration method and environment 
* to identify fields and aspects to conduct further research

\section{Benchmarking}\label{sec:benchmarking}

We tested the performance of mentioned acceleration methods in different environments. Implementation status and reason if not implemented is shown in table~\ref{tab:implemented}. Tested JS environments and their versions are described in table~\ref{tab:versions}.


As an algorithm to benchmark, we chose a simplified standard variant of Hough transform(SHT) \cite{mukhopadhyay2015survey}. Choosing a single algorithm over the whole benchmark suite gives us granular control over implementation, building process, and adaptation for each acceleration method. Hough transform, in the standard variant, is used to detect lines in binary images. It maps points to values in an accumulator space, called parameter space. Unlike the original \cite{hough1962method}, modern version maps points to curves $(x, y)$ using polar coordinates $(\theta, \rho)$ according to (\ref{eq:polar_hough}) \cite{duda1972use}.
Peaks in the parameter space can be mapped back to line candidates.
\begin{equation}
    \label{eq:polar_hough}
    f(x,y) = \rho(\theta) = x\cos{\theta}+y\sin{\theta}
\end{equation}


The resolution of the accumulator determines the precision of line detection, the size of the computational problem, and the required memory. The computational complexity of the sequential algorithm equals $O(wh)$ where $w$ and $h$ are dimensions of an input image. It could be also expressed as $O(S_{\theta} S_\rho)$ with constant input dimensions where $S_{\theta}$ and $S_\rho$ denotes angular and pixel sampling respectively. We benchmark each method for various problem sizes keeping everything constant but $S_\theta$. We implemented a version simplified from commonly seen ones. Our implementation defines the anchor point of polar coordinates in the upper left corner of the image instead of its center. It also bases the voting process on a simple threshold instead of analyzing the image space~\cite{palmer1997optimizing}.

\begin{wraptable}{r}{7.5cm}
    \vspace{-0.5cm}
    \begin{threeparttable}
        \caption{Analyzed acceleration methods and environments.}
        \label{tab:implemented}
        \setlength{\tabcolsep}{0.3em}
        \begin{tabularx}{7.5cm}{lllll }%
            \hline
                         & Chrome                            & Firefox                           & Node                              & Deno                              \\
            \hline
            Sequential   & \textcolor{blue}{\cmark}          & \textcolor{blue}{\cmark}          & \textcolor{blue}{\cmark}          & \textcolor{blue}{\cmark}          \\
            Native addon & \textcolor{red}{\xmark} \tnote{1} & \textcolor{red}{\xmark} \tnote{1} & \textcolor{blue}{\cmark}          & \textcolor{red}{\xmark} \tnote{2} \\
            asm.js       & \textcolor{blue}{\cmark}          & \textcolor{blue}{\cmark}          & \textcolor{blue}{\cmark}          & \textcolor{red}{\xmark} \tnote{2} \\
            WASM         & \textcolor{blue}{\cmark}          & \textcolor{blue}{\cmark}          & \textcolor{blue}{\cmark}          & \textcolor{red}{\xmark} \tnote{2} \\
            WASM+SIMD    & \textcolor{blue}{\cmark}          & \textcolor{blue}{\cmark}          & \textcolor{blue}{\cmark}          & \textcolor{red}{\xmark} \tnote{2} \\
            Workers      & \textcolor{blue}{\cmark}          & \textcolor{blue}{\cmark}          & \textcolor{blue}{\cmark}          & \textcolor{blue}{\cmark}          \\
            WebGL        & \textcolor{blue}{\cmark}          & \textcolor{blue}{\cmark}          & \textcolor{red}{\xmark} \tnote{2} & \textcolor{red}{\xmark} \tnote{1} \\
            WebGPU       & \textcolor{red}{\xmark} \tnote{3} & \textcolor{red}{\xmark} \tnote{3} & \textcolor{red}{\xmark} \tnote{1} & \textcolor{red}{\xmark} \tnote{3} \\
            \hline
        \end{tabularx}

        \begin{tablenotes}\footnotesize
            \item [1] Not available in environment
            \item [2] Requires external package or non-C++ codebase
            \item [3] Unstable or under a flag
        \end{tablenotes}

        \vspace{-0.5cm}
    \end{threeparttable}
\end{wraptable}

We believe that the algorithm being a representation of CPU-intensive task is sufficient for performing valuable benchmarks across different environments since it requires many iterations to fill the accumulator and enforces intensified memory usage for input data and the accumulator. Relying heavily on $\sin$ and $\cos$ functions, also allows us to test their performance. Because of that, we implemented two variants of each algorithm - \textit{non-LUT} and \textit{LUT}. The first one uses standard $\sin$ and $\cos$ functions and the second one caches their results in a lookup table. As shown in Figure \ref{fig:sht_example}, we also use the image after threshold operation instead of commonly used edge-detection. It requires more pixels to be processed thus increases problem size.



\begin{wraptable}{r}{7.5cm}
    \vspace{-0.75cm}
    \caption{Versions of environments.}
    \label{tab:versions}
    \setlength{\tabcolsep}{0.5em}
    \begin{tabular}{lllll}%
        \hline
        Env     & Version      & Engine version           \\
        \hline
        Chrome  & 97.0.4692.71 & V8 9.7.106.18            \\
        Firefox & 96.0         & SpiderMonkey  96.0       \\
        Node    & 16.13.2      & V8    9.4.146.24-node.14 \\
        Deno    & 1.18.0       & V8      9.8.177.6        \\
        \hline
    \end{tabular}
    \vspace{-0.5cm}
\end{wraptable}

Each benchmark lasts $5s$ minimum and $30s$ maximum or $50$ runs. We want to rely on the most likely execution scenarios which are influenced by JS engine optimizations, resulting in a shorter execution time. To resolve this cold start problem we use the coefficient of variance metric ($c_v$). We start the actual benchmark after the window of $5$ executions where $c_v \le 1\permil$.

\begin{figure}[]

    \begin{subfigure}{0.3\textwidth}
        \includegraphics[width=\linewidth] {../../test/threshold/1.jpg}
        \caption{Input image}\label{fig:sht_example:a}
    \end{subfigure}\hfill
    \begin{subfigure}{0.3\textwidth}
        \includegraphics[width=\linewidth] {../../packages/js-benchmarks/img/seq.png}
        \caption{Accumulator}\label{fig:sht_example:b}
    \end{subfigure}\hfill
    \begin{subfigure}{0.3\textwidth}
        \includegraphics[width=\linewidth] {img/sht_result.png}
        \caption{Detected lines}\label{fig:sht_example:c}
    \end{subfigure}
    \caption{Input image, accumulator and visualized result of sequential SHT algorithm in non-LUT variant ($S_\theta = 1, S_\rho=1$).}\label{fig:sht_example}
\end{figure}

Benchmarks were performed on a platform equipped with Intel\textsuperscript{\tiny\textregistered} Core\textsuperscript{\tiny\texttrademark} i7-12700KF CPU and Nvidia 970 GTX GPU using Ubuntu 20.04.1. The CPU had frequency scaling turned off and due to its hybrid architecture only 4 P-Cores were enabled for a benchmark using \texttt{taskset} utility.

%\section{Methods and results}\label{sec:methods_results}

\subsection{Sequential}

TODO: Algorithm description, variants, method description, other methods are only necessary modification of this method

TODO: Accumulator difference (one pixel differs - float64 vs float32) - single figure with 4 differences referred to later

TODO: Results, C++ as reference

TODO: Notice Firefox optimization after 5th run and chrome overall better optimization



\begin{figure}[ht]
    \groupBenchmark{
        \plotBenchmark{cpp_theta_SHT_Simple.csv}{cppColor}{{}}
        \addlegendentry{C++}

        \plotBenchmark{js-sequential_theta_SHT_Simple_node.csv}{nodeColor}{{}}
        \addlegendentry{Node}

        \plotBenchmark{js-sequential_theta_SHT_Simple_deno.csv}{denoColor}{{}}
        \addlegendentry{Deno}

        \plotBenchmark{js-sequential_theta_SHT_Simple_Firefox.csv}{firefoxColor}{{}}
        \addlegendentry{Firefox}

        \plotBenchmark{js-sequential_theta_SHT_Simple_Chrome.csv}{chromeColor}{{}}
        \addlegendentry{Chrome}

    } {
        \plotBenchmark{cpp_theta_SHT_Simple_Lookup.csv}{cppColor}{{}}
        \addlegendentry{C++}

        \plotBenchmark{js-sequential_theta_SHT_Simple_Lookup_node.csv}{nodeColor}{{}}
        \addlegendentry{Node}

        \plotBenchmark{js-sequential_theta_SHT_Simple_Lookup_deno.csv}{denoColor}{{}}
        \addlegendentry{Deno}

        \plotBenchmark{js-sequential_theta_SHT_Simple_Lookup_Firefox.csv}{firefoxColor}{{}}
        \addlegendentry{Firefox}

        \plotBenchmark{js-sequential_theta_SHT_Simple_Lookup_Chrome.csv}{chromeColor}{{}}
        \addlegendentry{Chrome}
    } {
        \plotBenchmark{cpp_theta_CHT_Simple.csv}{cppColor}{{}}
        \addlegendentry{C++}

        \plotBenchmark{js-sequential_theta_CHT_Simple_node.csv}{nodeColor}{{}}
        \addlegendentry{Node}

        \plotBenchmark{js-sequential_theta_CHT_Simple_deno.csv}{denoColor}{{}}
        \addlegendentry{Deno}

        \plotBenchmark{js-sequential_theta_CHT_Simple_Firefox.csv}{firefoxColor}{{}}
        \addlegendentry{Firefox}

        \plotBenchmark{js-sequential_theta_CHT_Simple_Chrome.csv}{chromeColor}{{}}
        \addlegendentry{Chrome}
    }
    [3400][850][14]
    \caption{Wyniki pomiarów czasu wydajności dla wykonania sekwencyjnego SHT i CHT.}
    \label{plot:sequential}
\end{figure}


TODO: Performance from devtools, show times in Math.sin and Math.cos

\begin{figure}
    \begin{subfigure}{0.29\textwidth}
        \includegraphics[width=\linewidth] {../../packages/js-benchmarks/img/diff_seq_seq_lookup.png}
        \caption{Sequential \textit{LUT}}\label{fig:diff:seq_lut}
    \end{subfigure}\hfill
    \begin{subfigure}{0.29\textwidth}
        \includegraphics[width=\linewidth] {../../packages/js-benchmarks/img/diff_seq_wasm.png}
        \caption{WASM}\label{fig:diff:wasm}
    \end{subfigure}\hfill
    \begin{subfigure}{0.29\textwidth}
        \includegraphics[width=\linewidth] {../../packages/js-benchmarks/img/diff_seq_gpu.png}
        \caption{WebGL}\label{fig:diff:gpu}
    \end{subfigure}
    \caption{Normalized absolute accumulator difference from sequential \textit{non-LUT} variant. Note the two-pixel difference in the upper right corner for Sequential \textit{LUT}.}\label{fig:diff}
\end{figure}


\subsection{Node C++ addon}

TODO: Method description, building process

TODO: Results, C++ and Sequential as reference



\begin{figure}
    \groupBenchmark{
        \plotBenchmark{cpp-addon_theta_SHT_Simple_node.csv}{nodeColor}{{}}
        \addlegendentry{Node}

        \seqReference
    } {
        \plotBenchmark{cpp-addon_theta_SHT_Simple_Lookup_node.csv}{nodeColor}{{}}
        \addlegendentry{Node}

        \seqReferenceLookup
    }[1400][350]
    \label{plot:cpu-addon}
    \caption{Node C++ addon SHT execution benchmark results. Gray area shows sequential JavaScript execution performance range.}
\end{figure}


\subsection{WebAssembly and Asm.js}

TODO: Method description, building process

TODO: Results, C++ and Sequential as reference



\begin{figure}[p]
    \groupBenchmark{
        \plotBenchmark{js-asm_theta_SHT_Simple_node.csv}{nodeColor}{{}}
        \addlegendentry{Node}

        \plotBenchmark{js-asm_theta_SHT_Simple_Firefox.csv}{firefoxColor}{{}}
        \addlegendentry{Firefox}

        \plotBenchmark{js-asm_theta_SHT_Simple_Chrome.csv}{chromeColor}{{}}
        \addlegendentry{Chrome}

        \seqReference
    } {
        \plotBenchmark{js-asm_theta_SHT_Simple_Lookup_node.csv}{nodeColor}{{}}
        \addlegendentry{Node}

        \plotBenchmark{js-asm_theta_SHT_Simple_Lookup_Firefox.csv}{firefoxColor}{{}}
        \addlegendentry{Firefox}

        \plotBenchmark{js-asm_theta_SHT_Simple_Lookup_Chrome.csv}{chromeColor}{{}}
        \addlegendentry{Chrome}

        \seqReferenceLookup
    }[8500][2200]
    \caption{asm.js SHT execution benchmark results.}
    \label{plot:asm}
\end{figure}




\begin{figure}
    \groupBenchmark{
        \plotBenchmark{js-wasm_theta_SHT_Simple_node.csv}{nodeColor}{{}}
        \addlegendentry{Node}

        \plotBenchmark{js-wasm_theta_SHT_Simple_Firefox.csv}{firefoxColor}{{}}
        \addlegendentry{Firefox 95}

        \plotBenchmark{js-wasm_theta_SHT_Simple_Chrome.csv}{chromeColor}{{}}
        \addlegendentry{Chrome 97}

        \seqReference
    } {
        \plotBenchmark{js-wasm_theta_SHT_Simple_Lookup_node.csv}{nodeColor}{{}}
        \addlegendentry{Node}

        \plotBenchmark{js-wasm_theta_SHT_Simple_Lookup_Firefox.csv}{firefoxColor}{{}}
        \addlegendentry{Firefox 95}

        \plotBenchmark{js-wasm_theta_SHT_Simple_Lookup_Chrome.csv}{chromeColor}{{}}
        \addlegendentry{Chrome 97}

        \seqReferenceLookup
    }[2500][950]
    \label{plot:wasm}
    \caption{WASM SHT execution benchmark results. Gray area shows sequential JavaScript execution performance range.}
\end{figure}


\subsection{WebAssembly SIMD}

TODO: Method description, building process



\begin{figure}
    \groupBenchmark{
        \plotBenchmark{js-wasm_simd_explicit_theta_SHT_Simple_node.csv}{nodeColor}{{}}
        \addlegendentry{Node}

        \plotBenchmark{js-wasm_simd_explicit_theta_SHT_Simple_Firefox.csv}{firefoxColor}{{}}
        \addlegendentry{Firefox 95}

        \plotBenchmark{js-wasm_simd_explicit_theta_SHT_Simple_Chrome.csv}{chromeColor}{{}}
        \addlegendentry{Chrome 97}

        \seqReference
    } {
        \plotBenchmark{js-wasm_simd_explicit_theta_SHT_Simple_Lookup_node.csv}{nodeColor}{{}}
        \addlegendentry{Node}

        \plotBenchmark{js-wasm_simd_explicit_theta_SHT_Simple_Lookup_Firefox.csv}{firefoxColor}{{}}
        \addlegendentry{Firefox 95}

        \plotBenchmark{js-wasm_simd_explicit_theta_SHT_Simple_Lookup_Chrome.csv}{chromeColor}{{}}
        \addlegendentry{Chrome 97}

        \seqReferenceLookup
    }[2200][650]
    \label{plot:wasm_simd_explicit}
    \caption{WASM SIMD (explicit) SHT execution benchmark results. Gray area shows sequential JavaScript execution performance range.}
\end{figure}


TODO: refer to accumulator differences

TODO: Instead of performance of implicit SIMD (llvm vectorization), mention that it is not better than standard WASM, but SIMD instructions are used to load/store data.

%

\begin{figure}
    \groupBenchmark{
        \plotBenchmark{js-wasm_simd_implicit_theta_SHT_Simple_node.csv}{nodeColor}{{}}
        \addlegendentry{Node}

        \plotBenchmark{js-wasm_simd_implicit_theta_SHT_Simple_Firefox.csv}{firefoxColor}{{}}
        \addlegendentry{Firefox}

        \plotBenchmark{js-wasm_simd_implicit_theta_SHT_Simple_Chrome.csv}{chromeColor}{{}}
        \addlegendentry{Chrome}

        \seqReference
    } {
        \plotBenchmark{js-wasm_simd_implicit_theta_SHT_Simple_Lookup_node.csv}{nodeColor}{{}}
        \addlegendentry{Node}

        \plotBenchmark{js-wasm_simd_implicit_theta_SHT_Simple_Lookup_Firefox.csv}{firefoxColor}{{}}
        \addlegendentry{Firefox}

        \plotBenchmark{js-wasm_simd_implicit_theta_SHT_Simple_Lookup_Chrome.csv}{chromeColor}{{}}
        \addlegendentry{Chrome}

        \seqReferenceLookup
    }[2500][1000]
    \label{plot:wasm_simd_implicit}
    \caption{WASM SIMD (implicit) SHT execution benchmark results. Gray area shows sequential JavaScript execution performance range.}
\end{figure}


TODO: Performance from devtools if interesting

\subsection{Workers}

TODO: Method description, building process

TODO: Results, C++ and Sequential as reference



\begin{figure}[h]
    \groupBenchmark{
        \plotBenchmark{js-workers_theta_SHT_Simple_node.csv}{nodeColor}{{}}
        \addlegendentry{Node}

        \plotBenchmark{js-workers_theta_SHT_Simple_deno.csv}{denoColor}{{}}
        \addlegendentry{Deno}

        \plotBenchmark{js-workers_theta_SHT_Simple_Firefox.csv}{firefoxColor}{{}}
        \addlegendentry{Firefox}

        \plotBenchmark{js-workers_theta_SHT_Simple_Chrome.csv}{chromeColor}{{}}
        \addlegendentry{Chrome}

        \seqReference
    } {
        \plotBenchmark{js-workers_theta_SHT_Simple_Lookup_node.csv}{nodeColor}{{}}
        \addlegendentry{Node}

        \plotBenchmark{js-workers_theta_SHT_Simple_Lookup_deno.csv}{denoColor}{{}}
        \addlegendentry{Deno}

        \plotBenchmark{js-workers_theta_SHT_Simple_Lookup_Firefox.csv}{firefoxColor}{{}}
        \addlegendentry{Firefox}

        \plotBenchmark{js-workers_theta_SHT_Simple_Lookup_Chrome.csv}{chromeColor}{{}}
        \addlegendentry{Chrome}

        \seqReferenceLookup
    }{
        \plotBenchmark{js-workers_theta_CHT_Simple_node.csv}{nodeColor}{{}}
        \addlegendentry{Node}

        \plotBenchmark{js-workers_theta_CHT_Simple_deno.csv}{denoColor}{{}}
        \addlegendentry{Deno}

        \plotBenchmark{js-workers_theta_CHT_Simple_Firefox.csv}{firefoxColor}{{}}
        \addlegendentry{Firefox}

        \plotBenchmark{js-workers_theta_CHT_Simple_Chrome.csv}{chromeColor}{{}}
        \addlegendentry{Chrome}

        \seqReferenceCircle
    }[1700][500][10]
    \caption{Wyniki pomiarów czasu wydajności dla wykonania SHT i CHT z wykorzystaniem czterech Worker'ów.}
    \label{plot:workers}
\end{figure}


TODO: Performance from devtools

TODO: speedup math + speedup efficiency

\subsection{WebGL}

TODO: Method description, building process (minification caveats)

TODO: refer to accumulator differences

TODO: Results, C++ and Sequential as reference



\begin{figure}
    \groupBenchmark{
        %\plotBenchmark{js-gpu_theta_SHT_Simple_node.csv}{nodeColor}{{}}
        %\addlegendentry{Node}

        %\plotBenchmark{js-gpu_theta_SHT_Simple_deno.csv}{denoColor}{{}}
        %\addlegendentry{Deno}

        \plotBenchmark{js-gpu_theta_SHT_Simple_Firefox.csv}{firefoxColor}{{}}
        \addlegendentry{Firefox}

        \plotBenchmark{js-gpu_theta_SHT_Simple_Chrome.csv}{chromeColor}{{}}
        \addlegendentry{Chrome}

        \seqReference
    } {
        %\plotBenchmark{js-gpu_theta_SHT_Simple_Lookup_node.csv}{nodeColor}{{}}
        %\addlegendentry{Node}

        %\plotBenchmark{js-gpu_theta_SHT_Simple_Lookup_deno.csv}{denoColor}{{}}
        %\addlegendentry{Deno}

        \plotBenchmark{js-gpu_theta_SHT_Simple_Lookup_Firefox.csv}{firefoxColor}{{}}
        \addlegendentry{Firefox}

        \plotBenchmark{js-gpu_theta_SHT_Simple_Lookup_Chrome.csv}{chromeColor}{{}}
        \addlegendentry{Chrome}

        \seqReferenceLookup
    }[550][200]
    \label{plot:gpu}
    \caption{WebGL SHT execution benchmark results. Gray area shows sequential JavaScript execution performance range.}
\end{figure}


TODO: Performance from devtools, readpixel time factor


\clearpage
 
\bibliographystyle{splncs04}
\bibliography{bibliography}

\end{document}
