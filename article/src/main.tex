\documentclass[runningheads]{llncs}

\usepackage{pgfplots}
\usepackage{tikz}
\usetikzlibrary{pgfplots.groupplots} % needs to be loaded exactly like this
\usepgfplotslibrary{fillbetween}
\usepgfplotslibrary{colorbrewer}
\usetikzlibrary{patterns}

\usepackage{csvsimple}
\usepackage{siunitx}
\usepackage{tabularx}
\usepackage{subcaption}
\usepackage{graphicx}
\usepackage{xargs}
\usepackage{url}
\usepackage{amssymb}
\usepackage{pifont}
\usepackage{threeparttable,tablefootnote}
\usepackage{wrapfig}
\usepackage{wasysym}
\usepackage[super]{nth}
\usepackage{listings}
\usepackage{jslistings}

\newcommand{\cmark}{\ding{51}}%
\newcommand{\xmark}{\ding{55}}%

\pgfplotsset{
    table/search path={./../../benchmark},
    every pin edge/.style={solid},
    compat=newest
}
\definecolor{cppColor}{rgb}{0,0,0}
\definecolor{nodeColor}{rgb}{0,0.6,0}
\definecolor{denoColor}{rgb}{0,0,1}
\definecolor{chromeColor}{rgb}{1,0,0}
\definecolor{firefoxColor}{rgb}{1,0.5,0}

\newcommand{\plotBenchmark}[3]{%
    \addplot+ [
        #2,
        #3,
        mark=*,
        mark options={solid,fill=#2, scale=0.5},
        error bars/.cd,
        error bar style={mark size=3pt, solid, #2},
        y dir=both,
        y explicit,
    ]
    table [
            x=sizeTheta,
            y=mean,
            y error=stdev,
            col sep=comma
        ] {#1};
}

\newenvironment{chartBenchmark}
{\begin{tikzpicture}
        \begin{axis} [
                width=\linewidth,
                height=0.45\linewidth,
                legend style={font=\tiny},
                grid,
                grid style=dashed,
                xlabel={$S_\theta$ [pixels per degree]},
                ylabel={Time [ms]},
                mark options={solid},
                legend pos=north west
            ] }
            %
            {
        \end{axis}
    \end{tikzpicture}
}

\newcommandx\groupBenchmark[4][3=1000, 4=800]{
    \begin{tikzpicture}
        \begin{groupplot}[
                group style={group size=2 by 1},
                width=0.52552\linewidth,
                grid,
                grid style=dashed,
                legend style={
                        legend columns=5,
                        font=\tiny,
                    },
                tick label style={font=\tiny},
                every axis title shift=0pt,
                max space between ticks=12,
                ymin=0
            ]

            \nextgroupplot[
                title={SHT Simple},
                ylabel={Time [ms]},
                legend to name={CommonLegend},
                xlabel={$S_\theta$ [pixels per degree]},
                ymax=#3
            ]
            #1

            \nextgroupplot[
                title={SHT Simple Lookup},
                xlabel={$S_\theta$ [pixels per degree]},
                legend to name={CommonLegend},
                ymax=#4
            ]
            #2
        \end{groupplot}
        \path (group c1r1.south east) -- node[below, yshift=-7ex] {\ref{CommonLegend}} (group c2r1.south west);
    \end{tikzpicture}
}


\newcommand\seqReference{
    \plotBenchmark{cpp_theta_SHT_Simple.csv}{cppColor}{dashed}
    \addlegendentry{C++ Sequential};
    \plotBenchmark
    {js-sequential_theta_SHT_Simple_Firefox.csv}
    {firefoxColor}
    {name path=Firefox_Seq,opacity=0}
    \plotBenchmark
    {js-sequential_theta_SHT_Simple_Chrome.csv}
    {chromeColor}
    {name path=Chrome_Seq,opacity=0}
    \addplot [black,opacity=0.1] fill between [of=Chrome_Seq and Firefox_Seq];
}

\newcommand\seqReferenceLookup{
    \plotBenchmark{cpp_theta_SHT_Simple_Lookup.csv}{cppColor}{dashed}
    \addlegendentry{C++ Sequential};
    \plotBenchmark
    {js-sequential_theta_SHT_Simple_Lookup_Firefox.csv}
    {firefoxColor}
    {name path=Firefox_Seq_Lookup,opacity=0}
    \plotBenchmark
    {js-sequential_theta_SHT_Simple_Lookup_Chrome.csv}
    {chromeColor}
    {name path=Chrome_Seq_Lookup,opacity=0}
    \addplot [black,opacity=0.1] fill between [of=Chrome_Seq_Lookup and Firefox_Seq_Lookup];
}


\begin{document}
%
\title{Performance and analysis of acceleration methods in JavaScript environments based on simplified standard Hough transform algorithm}
%
\titlerunning{Performance and analysis of acceleration methods\dots}

\author{Damian Koper \and Marek Woda}
%
% \authorrunning{D. Koper et al.}
% First names are abbreviated in the running head.
% If there are more than two authors, 'et al.' is used.
%
\institute{Faculty of Information and Communication Technology,\\
Wroclaw University of Science and Technology\\
\email{kopernickk@gmail.com, marek.woda@pwr.edu.pl}}
%
\maketitle              % typeset the header of the contribution
%
\begin{abstract}
  In this paper, we present analysis of popular acceleration methods in JavaScript execution environments including Chrome, Firefox, Node and Deno.  We focus evenly on adopting the same codebase to take advantage of every method, benchmarking our solutions and caveats of building libraries compatible with multiple environments. To compare performance, we use a simplified standard Hough transform algorithm. As a reference points of our benchmarks, we use sequential version of the algorithm written in both JavaScript and C++. Our study shows that Chrome is the fastest JS environment in every benchmark and Firefox is the slowest in which we identified optimization problems.  Overall WebGL appeared as the fastest and without parallel execution native C++ addon in Node is the most performant. We hope that this analysis will help to find the most efficient way to speed up execution making JavaScript a more robust environment for CPU-intensive computations.

  \keywords{javascript \and acceleration \and sht \and standard hough transform \and node \and browser \and deno \and webgl \and webpack \and wasm \and simd \and workers \and coldstart}
\end{abstract}

\section{Introduction}\label{sec:introduction}

TODO: JS in general, JS in numerical computing.

TODO: Environments briefly.

TODO: Acceleration methods briefly.

TODO: Building methods and complex architectures.

TODO: Target:
* overall comparison,
* to identify the most promising acceleration method and environment 
* to identify fields and aspects to conduct further research

\section{Benchmarking}\label{sec:benchmarking}

We tested the performance of mentioned acceleration methods in different environments. Implementation status and reason if not implemented is shown in table~\ref{tab:implemented}. Tested JS environments and their versions are described in table~\ref{tab:versions}.


As an algorithm to benchmark, we chose a simplified standard variant of Hough transform(SHT) \cite{mukhopadhyay2015survey}. Choosing a single algorithm over the whole benchmark suite gives us granular control over implementation, building process, and adaptation for each acceleration method. Hough transform, in the standard variant, is used to detect lines in binary images. It maps points to values in an accumulator space, called parameter space. Unlike the original \cite{hough1962method}, modern version maps points to curves $(x, y)$ using polar coordinates $(\theta, \rho)$ according to (\ref{eq:polar_hough}) \cite{duda1972use}.
Peaks in the parameter space can be mapped back to line candidates.
\begin{equation}
    \label{eq:polar_hough}
    f(x,y) = \rho(\theta) = x\cos{\theta}+y\sin{\theta}
\end{equation}


The resolution of the accumulator determines the precision of line detection, the size of the computational problem, and the required memory. The computational complexity of the sequential algorithm equals $O(wh)$ where $w$ and $h$ are dimensions of an input image. It could be also expressed as $O(S_{\theta} S_\rho)$ with constant input dimensions where $S_{\theta}$ and $S_\rho$ denotes angular and pixel sampling respectively. We benchmark each method for various problem sizes keeping everything constant but $S_\theta$. We implemented a version simplified from commonly seen ones. Our implementation defines the anchor point of polar coordinates in the upper left corner of the image instead of its center. It also bases the voting process on a simple threshold instead of analyzing the image space~\cite{palmer1997optimizing}.

\begin{wraptable}{r}{7.5cm}
    \vspace{-0.5cm}
    \begin{threeparttable}
        \caption{Analyzed acceleration methods and environments.}
        \label{tab:implemented}
        \setlength{\tabcolsep}{0.3em}
        \begin{tabularx}{7.5cm}{lllll }%
            \hline
                         & Chrome                            & Firefox                           & Node                              & Deno                              \\
            \hline
            Sequential   & \textcolor{blue}{\cmark}          & \textcolor{blue}{\cmark}          & \textcolor{blue}{\cmark}          & \textcolor{blue}{\cmark}          \\
            Native addon & \textcolor{red}{\xmark} \tnote{1} & \textcolor{red}{\xmark} \tnote{1} & \textcolor{blue}{\cmark}          & \textcolor{red}{\xmark} \tnote{2} \\
            asm.js       & \textcolor{blue}{\cmark}          & \textcolor{blue}{\cmark}          & \textcolor{blue}{\cmark}          & \textcolor{red}{\xmark} \tnote{2} \\
            WASM         & \textcolor{blue}{\cmark}          & \textcolor{blue}{\cmark}          & \textcolor{blue}{\cmark}          & \textcolor{red}{\xmark} \tnote{2} \\
            WASM+SIMD    & \textcolor{blue}{\cmark}          & \textcolor{blue}{\cmark}          & \textcolor{blue}{\cmark}          & \textcolor{red}{\xmark} \tnote{2} \\
            Workers      & \textcolor{blue}{\cmark}          & \textcolor{blue}{\cmark}          & \textcolor{blue}{\cmark}          & \textcolor{blue}{\cmark}          \\
            WebGL        & \textcolor{blue}{\cmark}          & \textcolor{blue}{\cmark}          & \textcolor{red}{\xmark} \tnote{2} & \textcolor{red}{\xmark} \tnote{1} \\
            WebGPU       & \textcolor{red}{\xmark} \tnote{3} & \textcolor{red}{\xmark} \tnote{3} & \textcolor{red}{\xmark} \tnote{1} & \textcolor{red}{\xmark} \tnote{3} \\
            \hline
        \end{tabularx}

        \begin{tablenotes}\footnotesize
            \item [1] Not available in environment
            \item [2] Requires external package or non-C++ codebase
            \item [3] Unstable or under a flag
        \end{tablenotes}

        \vspace{-0.5cm}
    \end{threeparttable}
\end{wraptable}

We believe that the algorithm being a representation of CPU-intensive task is sufficient for performing valuable benchmarks across different environments since it requires many iterations to fill the accumulator and enforces intensified memory usage for input data and the accumulator. Relying heavily on $\sin$ and $\cos$ functions, also allows us to test their performance. Because of that, we implemented two variants of each algorithm - \textit{non-LUT} and \textit{LUT}. The first one uses standard $\sin$ and $\cos$ functions and the second one caches their results in a lookup table. As shown in Figure \ref{fig:sht_example}, we also use the image after threshold operation instead of commonly used edge-detection. It requires more pixels to be processed thus increases problem size.



\begin{wraptable}{r}{7.5cm}
    \vspace{-0.75cm}
    \caption{Versions of environments.}
    \label{tab:versions}
    \setlength{\tabcolsep}{0.5em}
    \begin{tabular}{lllll}%
        \hline
        Env     & Version      & Engine version           \\
        \hline
        Chrome  & 97.0.4692.71 & V8 9.7.106.18            \\
        Firefox & 96.0         & SpiderMonkey  96.0       \\
        Node    & 16.13.2      & V8    9.4.146.24-node.14 \\
        Deno    & 1.18.0       & V8      9.8.177.6        \\
        \hline
    \end{tabular}
    \vspace{-0.5cm}
\end{wraptable}

Each benchmark lasts $5s$ minimum and $30s$ maximum or $50$ runs. We want to rely on the most likely execution scenarios which are influenced by JS engine optimizations, resulting in a shorter execution time. To resolve this cold start problem we use the coefficient of variance metric ($c_v$). We start the actual benchmark after the window of $5$ executions where $c_v \le 1\permil$.

\begin{figure}[]

    \begin{subfigure}{0.3\textwidth}
        \includegraphics[width=\linewidth] {../../test/threshold/1.jpg}
        \caption{Input image}\label{fig:sht_example:a}
    \end{subfigure}\hfill
    \begin{subfigure}{0.3\textwidth}
        \includegraphics[width=\linewidth] {../../packages/js-benchmarks/img/seq.png}
        \caption{Accumulator}\label{fig:sht_example:b}
    \end{subfigure}\hfill
    \begin{subfigure}{0.3\textwidth}
        \includegraphics[width=\linewidth] {img/sht_result.png}
        \caption{Detected lines}\label{fig:sht_example:c}
    \end{subfigure}
    \caption{Input image, accumulator and visualized result of sequential SHT algorithm in non-LUT variant ($S_\theta = 1, S_\rho=1$).}\label{fig:sht_example}
\end{figure}

Benchmarks were performed on a platform equipped with Intel\textsuperscript{\tiny\textregistered} Core\textsuperscript{\tiny\texttrademark} i7-12700KF CPU and Nvidia 970 GTX GPU using Ubuntu 20.04.1. The CPU had frequency scaling turned off and due to its hybrid architecture only 4 P-Cores were enabled for a benchmark using \texttt{taskset} utility.

\section{Results and details}\label{sec:results}
In this section, we present execution times for each method depending on angular sampling $S_\theta$ which should result in linear computational complexity. Section \ref{sec:results:sequential} shows times for sequential execution which is then marked as a grey area for comparison. Every chart contains native C++ times. 

\subsection{Sequential}\label{sec:results:sequential}

% MARK: times math
Analyzing benchmark times shown in Figure \ref{plot:sequential} we can see the advantage of Chrome over Firefox being $1.6\times$ faster in the \textit{non-LUT} variant and $2.08\times$ in the \textit{LUT} one. It it worth to notice the optimization in Firefox for the $S_\theta=5$ and subsequent sampling values in the \textit{LUT} variant which could be an object of further research. The performance of server-side environments, Node and Deno, since they are very similar environments, has only insignificant differences, yet still being $(1.50, 1.45)\times$ slower that Chrome for both variants.



\begin{figure}[ht]
    \groupBenchmark{
        \plotBenchmark{cpp_theta_SHT_Simple.csv}{cppColor}{{}}
        \addlegendentry{C++}

        \plotBenchmark{js-sequential_theta_SHT_Simple_node.csv}{nodeColor}{{}}
        \addlegendentry{Node}

        \plotBenchmark{js-sequential_theta_SHT_Simple_deno.csv}{denoColor}{{}}
        \addlegendentry{Deno}

        \plotBenchmark{js-sequential_theta_SHT_Simple_Firefox.csv}{firefoxColor}{{}}
        \addlegendentry{Firefox}

        \plotBenchmark{js-sequential_theta_SHT_Simple_Chrome.csv}{chromeColor}{{}}
        \addlegendentry{Chrome}

    } {
        \plotBenchmark{cpp_theta_SHT_Simple_Lookup.csv}{cppColor}{{}}
        \addlegendentry{C++}

        \plotBenchmark{js-sequential_theta_SHT_Simple_Lookup_node.csv}{nodeColor}{{}}
        \addlegendentry{Node}

        \plotBenchmark{js-sequential_theta_SHT_Simple_Lookup_deno.csv}{denoColor}{{}}
        \addlegendentry{Deno}

        \plotBenchmark{js-sequential_theta_SHT_Simple_Lookup_Firefox.csv}{firefoxColor}{{}}
        \addlegendentry{Firefox}

        \plotBenchmark{js-sequential_theta_SHT_Simple_Lookup_Chrome.csv}{chromeColor}{{}}
        \addlegendentry{Chrome}
    } {
        \plotBenchmark{cpp_theta_CHT_Simple.csv}{cppColor}{{}}
        \addlegendentry{C++}

        \plotBenchmark{js-sequential_theta_CHT_Simple_node.csv}{nodeColor}{{}}
        \addlegendentry{Node}

        \plotBenchmark{js-sequential_theta_CHT_Simple_deno.csv}{denoColor}{{}}
        \addlegendentry{Deno}

        \plotBenchmark{js-sequential_theta_CHT_Simple_Firefox.csv}{firefoxColor}{{}}
        \addlegendentry{Firefox}

        \plotBenchmark{js-sequential_theta_CHT_Simple_Chrome.csv}{chromeColor}{{}}
        \addlegendentry{Chrome}
    }
    [3400][850][14]
    \caption{Wyniki pomiarów czasu wydajności dla wykonania sekwencyjnego SHT i CHT.}
    \label{plot:sequential}
\end{figure}


We detected one pixel difference in generated accumulator between variants as shown in Figure \ref{fig:diff:seq_lut} (upper right corner). We implemented lookup table for \textit{LUT} variants using \texttt{Float32Array}. JS internally, without optimizations, represents numbers in double-precision and reduced precision of cached values can have significant impact on detection results.

\begin{figure}
    \begin{subfigure}{0.29\textwidth}
        \includegraphics[width=\linewidth] {../../packages/js-benchmarks/img/diff_seq_seq_lookup.png}
        \caption{Sequential \textit{LUT}}\label{fig:diff:seq_lut}
    \end{subfigure}\hfill
    \begin{subfigure}{0.29\textwidth}
        \includegraphics[width=\linewidth] {../../packages/js-benchmarks/img/diff_seq_wasm.png}
        \caption{WASM}\label{fig:diff:wasm}
    \end{subfigure}\hfill
    \begin{subfigure}{0.29\textwidth}
        \includegraphics[width=\linewidth] {../../packages/js-benchmarks/img/diff_seq_gpu.png}
        \caption{WebGL}\label{fig:diff:gpu}
    \end{subfigure}
    \caption{Normalized absolute accumulator difference from sequential \textit{non-LUT} variant. Note the two-pixel difference in the upper right corner for Sequential \textit{LUT}.}\label{fig:diff}
\end{figure}


\subsection{Node C++ addon}\label{sec:results:cpp-addon}

C++ addon for Node was built using the same shared library as the C++ version. Thus the difference in performance between native C++ and the addon arise mostly from handling data transfer between C++ -- JS boundary since the data needs to be copied and transformed to corresponding C++ structures. Results are shown in Figure \ref{plot:cpu-addon}.



\begin{figure}
    \groupBenchmark{
        \plotBenchmark{cpp-addon_theta_SHT_Simple_node.csv}{nodeColor}{{}}
        \addlegendentry{Node}

        \seqReference
    } {
        \plotBenchmark{cpp-addon_theta_SHT_Simple_Lookup_node.csv}{nodeColor}{{}}
        \addlegendentry{Node}

        \seqReferenceLookup
    }[1400][350]
    \label{plot:cpu-addon}
    \caption{Node C++ addon SHT execution benchmark results. Gray area shows sequential JavaScript execution performance range.}
\end{figure}



% MARK: times math
Compilation with optimization of trigonometric functions in \textit{non-LUT} variant allowed to gain more performance ($2.21\times$) than compilation of the \textit{LUT} variant ($1.38\times$) relative to their sequential variants. This case allows us to draw a conclusion that if an algorithm has trigonometric functions and output of which cannot be cached beforehand, the usage of the C++ addon in Node is beneficial.

\subsection{WebAssembly and asm.js}\label{sec:results:asm-wasm}

Benchmark results for asm.js and WASM were shown on figures \ref{plot:asm} and \ref{plot:wasm} respectively. In our case asm.js - a highly optimizable subset of JS instructions, operating only on numeric types and using heap memory - is actually slower in all environments than sequential execution. We suspect that it is caused by the building process. Webpack adds its own module resolution mechanisms that prevent part of the bundle with asm.js code from being recognized and compiled ahead-of-time. Performance flame chart from Chrome DevTools tools shows a lack of \texttt{Compile Code} blocks, unlike any other isolated asm.js sample.

WASM on the other hand improves performance for \textit{non-LUT} variant and has no effect on \textit{LUT} variant besides preventing optimization mentioned in section \ref{sec:results:sequential} in Firefox. Again, it is beneficial to use this method if the output if trigonometric functions cannot be cached.

In our C++ implementation we use single precision floating point variables. This results in accumulator differences shown in Figure \ref{fig:diff:wasm} since WASM distinguishes between \texttt{f32} and \texttt{f64} types.



\begin{figure}[p]
    \groupBenchmark{
        \plotBenchmark{js-asm_theta_SHT_Simple_node.csv}{nodeColor}{{}}
        \addlegendentry{Node}

        \plotBenchmark{js-asm_theta_SHT_Simple_Firefox.csv}{firefoxColor}{{}}
        \addlegendentry{Firefox}

        \plotBenchmark{js-asm_theta_SHT_Simple_Chrome.csv}{chromeColor}{{}}
        \addlegendentry{Chrome}

        \seqReference
    } {
        \plotBenchmark{js-asm_theta_SHT_Simple_Lookup_node.csv}{nodeColor}{{}}
        \addlegendentry{Node}

        \plotBenchmark{js-asm_theta_SHT_Simple_Lookup_Firefox.csv}{firefoxColor}{{}}
        \addlegendentry{Firefox}

        \plotBenchmark{js-asm_theta_SHT_Simple_Lookup_Chrome.csv}{chromeColor}{{}}
        \addlegendentry{Chrome}

        \seqReferenceLookup
    }[8500][2200]
    \caption{asm.js SHT execution benchmark results.}
    \label{plot:asm}
\end{figure}




\begin{figure}
    \groupBenchmark{
        \plotBenchmark{js-wasm_theta_SHT_Simple_node.csv}{nodeColor}{{}}
        \addlegendentry{Node}

        \plotBenchmark{js-wasm_theta_SHT_Simple_Firefox.csv}{firefoxColor}{{}}
        \addlegendentry{Firefox 95}

        \plotBenchmark{js-wasm_theta_SHT_Simple_Chrome.csv}{chromeColor}{{}}
        \addlegendentry{Chrome 97}

        \seqReference
    } {
        \plotBenchmark{js-wasm_theta_SHT_Simple_Lookup_node.csv}{nodeColor}{{}}
        \addlegendentry{Node}

        \plotBenchmark{js-wasm_theta_SHT_Simple_Lookup_Firefox.csv}{firefoxColor}{{}}
        \addlegendentry{Firefox 95}

        \plotBenchmark{js-wasm_theta_SHT_Simple_Lookup_Chrome.csv}{chromeColor}{{}}
        \addlegendentry{Chrome 97}

        \seqReferenceLookup
    }[2500][950]
    \label{plot:wasm}
    \caption{WASM SHT execution benchmark results. Gray area shows sequential JavaScript execution performance range.}
\end{figure}



\subsection{WebAssembly SIMD}

% MARK: times math
SIMD instructions in WASM are available from Chrome 91 and Firefox 89 for all users. The usage of SIMD instructions can be done implicitly by letting the compiler (commonly LLVM) perform the auto-vectorization process or explicitly by using vector instructions in code. We tested both solutions resulting in no difference from sequential benchmarks for the first one. Because of that, we present only an explicit usage attempt. In benchmarks shown in Figure \ref{plot:wasm_simd_explicit} we can see that the performance difference between Chrome and Firefox decreased compared to sequential execution and Chrome is only $1.16\times$ faster than Firefox. Moreover, Firefox overtaken Node in performance, which was not as prone to SIMD optimization as other environments.



\begin{figure}
    \groupBenchmark{
        \plotBenchmark{js-wasm_simd_explicit_theta_SHT_Simple_node.csv}{nodeColor}{{}}
        \addlegendentry{Node}

        \plotBenchmark{js-wasm_simd_explicit_theta_SHT_Simple_Firefox.csv}{firefoxColor}{{}}
        \addlegendentry{Firefox 95}

        \plotBenchmark{js-wasm_simd_explicit_theta_SHT_Simple_Chrome.csv}{chromeColor}{{}}
        \addlegendentry{Chrome 97}

        \seqReference
    } {
        \plotBenchmark{js-wasm_simd_explicit_theta_SHT_Simple_Lookup_node.csv}{nodeColor}{{}}
        \addlegendentry{Node}

        \plotBenchmark{js-wasm_simd_explicit_theta_SHT_Simple_Lookup_Firefox.csv}{firefoxColor}{{}}
        \addlegendentry{Firefox 95}

        \plotBenchmark{js-wasm_simd_explicit_theta_SHT_Simple_Lookup_Chrome.csv}{chromeColor}{{}}
        \addlegendentry{Chrome 97}

        \seqReferenceLookup
    }[2200][650]
    \label{plot:wasm_simd_explicit}
    \caption{WASM SIMD (explicit) SHT execution benchmark results. Gray area shows sequential JavaScript execution performance range.}
\end{figure}


\subsection{Workers}

All worker benchmarks used concurrency $n=4$. Results are shown in Figure \ref{plot:workers}. Because of simplified implementation described in section \ref{sec:benchmarking}, precisely the center of polar coordinate system in image space, the \nth{3} worker is redundant. The \nth{3} vertical quarter of the accumulator will always be empty (Fig. \ref{fig:sht_example:b}). This was not optimized in our implementation.

\begin{wraptable}{r}{7.9cm}
    \caption{Przyspieszenie i jego efektywność dla metody akceleracji z wykorzystaniem Worker'ów \mbox{($S_\theta = 1, n=1, p = 4$)}.}
    \label{tab:speedup}
    \begin{tabular}{rlrr}%
        \hline
        Alg.                 & Środ.   & Przysp. & Efek. \\
        \hline
        SHT \textit{non-LUT} & Chrome  & 2.78    & 0.69  \\
        SHT \textit{non-LUT} & Firefox & 2.89    & 0.72  \\
        SHT \textit{non-LUT} & Node    & 3.19    & \textcolor{green!70!black}{0.80}  \\
        SHT \textit{non-LUT} & Deno    & 2.56    & \textcolor{red!70!black}{0.64}  \\
        \hline
        SHT \textit{LUT}     & Chrome  & 1.84    & 0.46  \\
        SHT \textit{LUT}     & Firefox & 1.83    & 0.46  \\
        SHT \textit{LUT}     & Node    & 1.66    & \textcolor{red!70!black}{0.41}  \\
        SHT \textit{LUT}     & Deno    & 1.94    & \textcolor{green!70!black}{0.48}  \\
        \hline
        CHT                  & Chrome  & 1.19    & \textcolor{red!70!black}{0.30} \\
        CHT                  & Firefox & 1.29    & \textcolor{green!70!black}{0.32}  \\
        CHT                  & Node    & 1.20    & \textcolor{red!70!black}{0.30}  \\
        CHT                  & Deno    & 1.22    & 0.31  \\
        \hline
    \end{tabular}
\end{wraptable}
 
The table \ref{tab:worker_speedup} shows the speedup and its efficiency for environments and variants. The big difference in speedup efficiency between variants again shows us how demanding trigonometric functions are. Only the accumulator filling process was parallelized thus the speedup difference between environments is expected since the worker calculations take less time due to the lookup tables. Our implementation can be improved to achieve better performance because the voting process is not parallelized. Even though, the current state of implementation still allows us to compare this method across environments.

We share the accumulator array between workers using \texttt{SharedArrayBuffer} and it is important to mention that our implementation does not use \texttt{Atomics} since every worker operates on a different part of the array. According to our benchmarks, \texttt{Atomics} tends to slow down performance and were not necessary in this case.



\begin{figure}[h]
    \groupBenchmark{
        \plotBenchmark{js-workers_theta_SHT_Simple_node.csv}{nodeColor}{{}}
        \addlegendentry{Node}

        \plotBenchmark{js-workers_theta_SHT_Simple_deno.csv}{denoColor}{{}}
        \addlegendentry{Deno}

        \plotBenchmark{js-workers_theta_SHT_Simple_Firefox.csv}{firefoxColor}{{}}
        \addlegendentry{Firefox}

        \plotBenchmark{js-workers_theta_SHT_Simple_Chrome.csv}{chromeColor}{{}}
        \addlegendentry{Chrome}

        \seqReference
    } {
        \plotBenchmark{js-workers_theta_SHT_Simple_Lookup_node.csv}{nodeColor}{{}}
        \addlegendentry{Node}

        \plotBenchmark{js-workers_theta_SHT_Simple_Lookup_deno.csv}{denoColor}{{}}
        \addlegendentry{Deno}

        \plotBenchmark{js-workers_theta_SHT_Simple_Lookup_Firefox.csv}{firefoxColor}{{}}
        \addlegendentry{Firefox}

        \plotBenchmark{js-workers_theta_SHT_Simple_Lookup_Chrome.csv}{chromeColor}{{}}
        \addlegendentry{Chrome}

        \seqReferenceLookup
    }{
        \plotBenchmark{js-workers_theta_CHT_Simple_node.csv}{nodeColor}{{}}
        \addlegendentry{Node}

        \plotBenchmark{js-workers_theta_CHT_Simple_deno.csv}{denoColor}{{}}
        \addlegendentry{Deno}

        \plotBenchmark{js-workers_theta_CHT_Simple_Firefox.csv}{firefoxColor}{{}}
        \addlegendentry{Firefox}

        \plotBenchmark{js-workers_theta_CHT_Simple_Chrome.csv}{chromeColor}{{}}
        \addlegendentry{Chrome}

        \seqReferenceCircle
    }[1700][500][10]
    \caption{Wyniki pomiarów czasu wydajności dla wykonania SHT i CHT z wykorzystaniem czterech Worker'ów.}
    \label{plot:workers}
\end{figure}


\subsection{WebGL}

Our last acceleration method uses a GPGPU to fill the accumulator array. With help of the WebGL and the gpu.js library, we implemented kernel functions calculating every pixel separately. It is the only possible solution since the WebGL pipeline does not provide shared memory. This results in a bigger accumulator difference shown in Figure \ref{fig:diff:gpu}. First of all, the pipeline provides only single-precision operations. Secondly, for every accumulator value - pair ($\theta$, $\rho$), we had to sum image pixels laying on a possible line. This operation is prone to rounding errors. Additionally, the minification of output bundle provided by Webpack was interfering with the way the gpu.js library transpiles code to a GLSL language. We had to construct the function from string to prevent minification of the kernel function -- \lstinline[language=JavaScript]|new Function('return function (testImage) {...}')()|.

This method has the biggest result variance which comes directly from communication between CPU and GPU. It has also the biggest cold start times since the kernel has to be compiled by an environment on the first run. There is no big difference between both variants because in the \textit{non-LUT} variant each thread on the GPU has to calculate $sin$ and $cos$ functions once which is not a significant overhead.


\begin{figure}
    \groupBenchmark{
        %\plotBenchmark{js-gpu_theta_SHT_Simple_node.csv}{nodeColor}{{}}
        %\addlegendentry{Node}

        %\plotBenchmark{js-gpu_theta_SHT_Simple_deno.csv}{denoColor}{{}}
        %\addlegendentry{Deno}

        \plotBenchmark{js-gpu_theta_SHT_Simple_Firefox.csv}{firefoxColor}{{}}
        \addlegendentry{Firefox}

        \plotBenchmark{js-gpu_theta_SHT_Simple_Chrome.csv}{chromeColor}{{}}
        \addlegendentry{Chrome}

        \seqReference
    } {
        %\plotBenchmark{js-gpu_theta_SHT_Simple_Lookup_node.csv}{nodeColor}{{}}
        %\addlegendentry{Node}

        %\plotBenchmark{js-gpu_theta_SHT_Simple_Lookup_deno.csv}{denoColor}{{}}
        %\addlegendentry{Deno}

        \plotBenchmark{js-gpu_theta_SHT_Simple_Lookup_Firefox.csv}{firefoxColor}{{}}
        \addlegendentry{Firefox}

        \plotBenchmark{js-gpu_theta_SHT_Simple_Lookup_Chrome.csv}{chromeColor}{{}}
        \addlegendentry{Chrome}

        \seqReferenceLookup
    }[550][200]
    \label{plot:gpu}
    \caption{WebGL SHT execution benchmark results. Gray area shows sequential JavaScript execution performance range.}
\end{figure}


\section{Conclusions}\label{sec:conclusions}
We performed various benchmarks of the same algorithm in Chrome, Firefox, Node, and Deno environments listed in Table \ref{tab:versions}. In each one, we tested available and popular acceleration methods including a native addon, WASM alone, WASM with SIMD instructions, multi-threading with workers, and GPGPU using \mbox{WebGL} graphics pipeline. We did not test every method on every environment because of some being unavailable, unstable, under a flag, or based on non-C++ codebase as shown in Table \ref{tab:implemented}. Summarized results for the same problem size are shown in Table \ref{tab:envs}.
In every benchmark Chrome appears as the fastest environment with Firefox being $2.40\times$, Node $1.45\times$, and Deno $1.46\times$ slower in general. As expected, in the case without involving parallel execution, the Node C++ native addon brings the best results across all environments. Server-side environments performed similarly with a slight predominance of Node over Deno.

According to our results, the performance of \textit{LUT} variant was always better than its \textit{non-LUT} counterpart. Trigonometric functions are demanding but our study shows that using native addon or compiling code to WASM can prevent significant performance loss, especially in the Firefox environment. We can see that Firefox is not able to optimize code as well as other environments, where using vector instructions explicitly actually lowers performance, increasing it in Firefox.
When using lookup tables results may be different. Looking at Chrome performance of all WASM methods with \textit{LUT} variant we can see that performance is roughly the same with the sequential. Data exchanged between JS and WASM must be transformed and copied, which takes time, so it is not safe to assume performance benefit with intensive memory input and output when adopting WASM.
We also identified a problem with asm.js. which wasn't compiled ahead-of-time. We suspect that the bundling system and minification process prevented environments from recognizing asm.js specific code. 

To sum up, all environments serve similar cases but differ in terms of the performance of various acceleration methods. It is important to analyze which method best suits our needs depending on requirements. 
With all this, it is important to remember that no acceleration method can increase the performance of an algorithm like improvement of the computational complexity of the algorithm itself.

\newcommand{\comma}{, }
\begin{table}
    \label{tab:envs}
    \caption{Comparison of implemented methods in analyzed environments. The general comparison was done using Chrome as a reference point and geometric mean for times comparable with Chrome.}

    \begin{tabularx}{\linewidth}{X r r r r}%
        \hline
                                 & \multicolumn{4}{c}{\bfseries Execution time[ms]}                                                             \\
        \bfseries Method         & \bfseries Chrome                                 & \bfseries Firefox     & \bfseries Node  & \bfseries Deno

        \csvreader[
            head to column names,
            before first line=                                                                                                                  \\\hline,
            late after line=                                                                                                                    \\,
            late after last line=                                                                                                               \\\hline
        ]{../../benchmark/environments/envs.csv}{}% use head of csv as column names
        {
        \name\                   & \chrome\ (\chromeF)                              & \firefox\ (\firefoxF) & \node\ (\nodeF) & \deno\ (\denoF)
        }% specify your coloumns here 
        \bfseries Geometric mean &
        \csvreader[
            head to column names,
            late after last line=                                                                                                               \\\hline

        ]{../../benchmark/environments/geoMeans.csv}{}% use head of csv as column names
        {
        (\chrome)                & (\firefox)                                       & (\node)               & (\deno)
        }% specify your coloumns here 
    \end{tabularx}

\end{table}


% TODO: identified interesting aspects

% TODO: further research

% TODO: state of multi-environment numerical computing in JS


\clearpage
 
\bibliographystyle{splncs04}
\bibliography{bibliography}

\end{document}
