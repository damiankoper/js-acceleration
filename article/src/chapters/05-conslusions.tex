\section{Conclusions}\label{sec:conclusions}
We performed various benchmarks of the same algorithm in Chrome, Firefox, Node, and Deno environments listed in Table \ref{tab:versions}. In each one, we tested available and popular acceleration methods including a native addon, WASM alone, WASM with SIMD instructions, multi-threading with workers, and GPGPU using \mbox{WebGL} graphics pipeline. We did not test every method on every environment because of some being unavailable, unstable, under a flag, or based on non-C++ codebase as shown in Table \ref{tab:implemented}. Summarized results for the same problem size are shown in Table \ref{tab:envs}.
In every benchmark Chrome appears as the fastest environment with Firefox being $2.40\times$, Node $1.45\times$, and Deno $1.46\times$ slower in general. As expected, in the case without involving parallel execution, the Node C++ native addon brings the best results across all environments. Server-side environments performed similarly with a slight predominance of Node over Deno.

According to our results, the performance of \textit{LUT} variant was always better than its \textit{non-LUT} counterpart. Trigonometric functions are demanding but our study shows that using native addon or compiling code to WASM can prevent significant performance loss, especially in the Firefox environment. We can see that Firefox is not able to optimize code as well as other environments, where using vector instructions explicitly actually lowers performance, increasing it in Firefox.
When using lookup tables results may be different. Looking at Chrome performance of all WASM methods with \textit{LUT} variant we can see that performance is roughly the same with the sequential. Data exchanged between JS and WASM must be transformed and copied, which takes time, so it is not safe to assume performance benefit with intensive memory input and output when adopting WASM.
We also identified a problem with asm.js. which wasn't compiled ahead-of-time. We suspect that the bundling system and minification process prevented environments from recognizing asm.js specific code. 

To sum up, all environments serve similar cases but differ in terms of the performance of various acceleration methods. It is important to analyze which method best suits our needs depending on requirements. 
With all this, it is important to remember that no acceleration method can increase the performance of an algorithm like improvement of the computational complexity of the algorithm itself.

\newcommand{\comma}{, }
\begin{table}
    \label{tab:envs}
    \caption{Comparison of implemented methods in analyzed environments. The general comparison was done using Chrome as a reference point and geometric mean for times comparable with Chrome.}

    \begin{tabularx}{\linewidth}{X r r r r}%
        \hline
                                 & \multicolumn{4}{c}{\bfseries Execution time[ms]}                                                             \\
        \bfseries Method         & \bfseries Chrome                                 & \bfseries Firefox     & \bfseries Node  & \bfseries Deno

        \csvreader[
            head to column names,
            before first line=                                                                                                                  \\\hline,
            late after line=                                                                                                                    \\,
            late after last line=                                                                                                               \\\hline
        ]{../../benchmark/environments/envs.csv}{}% use head of csv as column names
        {
        \name\                   & \chrome\ (\chromeF)                              & \firefox\ (\firefoxF) & \node\ (\nodeF) & \deno\ (\denoF)
        }% specify your coloumns here 
        \bfseries Geometric mean &
        \csvreader[
            head to column names,
            late after last line=                                                                                                               \\\hline

        ]{../../benchmark/environments/geoMeans.csv}{}% use head of csv as column names
        {
        (\chrome)                & (\firefox)                                       & (\node)               & (\deno)
        }% specify your coloumns here 
    \end{tabularx}

\end{table}


% TODO: identified interesting aspects

% TODO: further research

% TODO: state of multi-environment numerical computing in JS
